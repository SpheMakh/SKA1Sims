% Auto generated table
 \begin{table}[!htp]
 \tiny{
\subfloat[DEC=-30, natural weighting]{\begin{tabular}{|lcccc|} \hline 
 resbin & 1 & 2 & 3 & 4 \tabularnewline \hline
SKA1REF2 & 1.42 \cellcolor{blue!60.00} & 0.42 \cellcolor{red!60.00} & 0.11 \cellcolor{green!18.00} & 0.17 \cellcolor{orange!18.00}\\ \hline 
SKA1W9-12A72B120 & 0.70 \cellcolor{blue!18.00} & 0.19 \cellcolor{red!18.00} & 0.11 \cellcolor{green!18.00} & 0.21 \cellcolor{orange!60.00}\\ \hline 
SKA1W9-0A72B120 & 0.81 \cellcolor{blue!24.42} & 0.22 \cellcolor{red!23.48} & 0.13 \cellcolor{green!60.00} & 0.20 \cellcolor{orange!49.50}\tabularnewline \hline 
\end{tabular}}\hfil 

\caption{The hours required to reach a mean SNR of 10 (average over 8, 12 and 13.8GHz), relative to a 10$\mu$Jy source at 13.8GHz with a spectral index of -0.7 for the different layouts at different angular scales. These values are generated for angular scales \{0.04-0.05, 0.05-0.1, 0.1-1, 1-12\} arcsec and are labeled {\it resbin} \{1, 2, 3, 4\} respectively. This is done for natural weighting at a declination of -30 and degrees. For each column, the intensity of the color increases with the value.}\label{tab:hours}}
 \end{table}