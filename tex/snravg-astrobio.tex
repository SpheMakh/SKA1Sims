% Auto generated table
 \begin{table}[!htp]
 \tiny{
\subfloat[DEC=-30, natural weighting]{\begin{tabular}{|lcccc|} \hline 
 resbin & 1 & 2 & 3 & 4 \tabularnewline \hline
SKA1REF2 & 23.72 \cellcolor{blue!18.00} & 43.56 \cellcolor{red!18.00} & 85.76 \cellcolor{green!52.51} & 68.87 \cellcolor{orange!60.00}\\ \hline 
SKA1W9-12A72B120 & 33.92 \cellcolor{blue!60.00} & 65.60 \cellcolor{red!60.00} & 87.18 \cellcolor{green!60.00} & 62.18 \cellcolor{orange!18.00}\\ \hline 
SKA1W9-0A72B120 & 31.52 \cellcolor{blue!50.12} & 60.32 \cellcolor{red!49.94} & 79.22 \cellcolor{green!18.00} & 62.74 \cellcolor{orange!21.52}\tabularnewline \hline 
\end{tabular}}\hfil 

\caption{SNR after 8 hours relative to a 10$\mu$Jy source at 13.8GHz (2.5GHz band) with a spectral index of -0.7 averaged over 8, 12 and 13.8GHz, for the different layouts at different angular scales. These values are generated for angular scales \{0.04-0.05, 0.05-0.1, 0.1-1, 1-12\} arcsec and are labelled {\it resbin} \{1, 2, 3, 4\} respectively. This is done for natural weighting at a declination of -30 degrees. For each column, the intensity of the color increases with the value.}\label{tab:snravg-astrobio}}
 \end{table}