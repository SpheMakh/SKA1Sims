% Auto generated table
 \begin{table}[!htp]
 \tiny{
\subfloat[DEC=-30, natural weighting]{\begin{tabular}{|lcccc|} \hline 
 resbin & 1 & 2 & 3 & 4 \tabularnewline \hline
SKA1REF2 & 1.00 \cellcolor{blue!18.00} & 1.00 \cellcolor{red!18.00} & 1.00 \cellcolor{green!43.20} & 1.00 \cellcolor{orange!60.00}\\ \hline 
SKA1W9-12A72B120 & 1.24 \cellcolor{blue!60.00} & 2.54 \cellcolor{red!60.00} & 1.09 \cellcolor{green!60.00} & 0.82 \cellcolor{orange!18.00}\\ \hline 
SKA1W9-0A72B120 & 1.06 \cellcolor{blue!28.50} & 2.20 \cellcolor{red!50.77} & 0.87 \cellcolor{green!18.00} & 0.82 \cellcolor{orange!18.00}\tabularnewline \hline 
\end{tabular}}\hfil 

\caption{Relative (w.r.t REF2) average survey speeds for the different layouts, calculated using the FOV (PAF FOV for SKASUR) values given in the SRD \cite{srd} and the values in table \ref{tab:snr10-astrobio}. These values are generated for angular scales \{0.04-0.05, 0.05-0.1, 0.1-1, 1-12\} arcsec and are labeled {\it resbin} \{1, 2, 3, 4\} respectively. This is done for natural weighting at a declenation of -30 degrees. For each column, the intensity of the color increases with the value.}\label{tab:speed_avg-astrobio}}
 \end{table}