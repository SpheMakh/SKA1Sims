% Auto generated table
 \begin{table}[!htp]
 \tiny{ 
\begin{tabular}{|lccccc|} \hline 
 resbin & 1 & 2 & 3 & 4 & 5 \tabularnewline \hline
SKA1REF2 & 2.10 \cellcolor{blue!30.83} & 1.70 \cellcolor{red!39.00} & 1.40 \cellcolor{green!51.60} & 0.92 \cellcolor{orange!48.41}
& 12.00 \cellcolor{purple!60.00}\\ \hline 
SKA1W9-12A72B120 & 4.60 \cellcolor{blue!60.00} & 2.40 \cellcolor{red!60.00} & 1.50 \cellcolor{green!60.00} & 0.86
\cellcolor{orange!39.72} & 10.00 \cellcolor{purple!52.36}\\ \hline 
SKA1W9-0A72B120 & 4.00 \cellcolor{blue!53.00} & 1.90 \cellcolor{red!45.00} & 1.20 \cellcolor{green!34.80} & 0.71
\cellcolor{orange!18.00} & 11.00 \cellcolor{purple!56.18}\\ \hline 
SKASUR & 1.00 \cellcolor{blue!18.00} & 1.00 \cellcolor{red!18.00} & 1.00 \cellcolor{green!18.00} & 1.00 \cellcolor{orange!60.00} &
1.00 \cellcolor{purple!18.00}\tabularnewline \hline 
\end{tabular}\hfil
\caption{Relative (w.r.t SKASUR) survey speeds for the different layouts, calculated using the FOV values given in the SRD
\cite{srd}. These values are generated for angular scales \{0.4-1, 1-2, 2-3, 3-4, 600-3600\} arcsec and are labeled {\it resbin}
\{1, 2, 3, 4, 5\} respectively. This is done for natural weighting at declination -30 degrees. For each column, the
intensity of the color increases with the value.}\label{tab:speed-mix_sefd}}
\end{table}