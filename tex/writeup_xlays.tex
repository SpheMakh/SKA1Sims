\documentclass[sfheadings,a4paper,10pt,floats,floatfix]{article}
\usepackage[margin=0.7in]{geometry}
\usepackage{graphicx}
\usepackage{verbatim}
\usepackage{bookmark}
\usepackage{float,url}% to fix the figure at the exact position
% \restylefloat{figure}
%\usepackage{hyperref}
% \usepackage{latin1}%{inputenc}
\usepackage{mathrsfs}
\usepackage{amssymb}
\usepackage{amsmath}
\usepackage{bm}
\usepackage{color}
\usepackage[table]{xcolor}% http://ctan.org/pkg/xcolor
\usepackage{multirow}
% \usepackage{caption}
\usepackage{subfig}%
% \usepackage{subcaption}
\newcommand{\bra}[1]{\left(#1\right)}
\newcommand{\loge}[1]{\text{#1}}
\newcommand{\bras}[1]{\left[#1\right]}
\newcommand{\brac}[1]{\left\{#1\right\}}
\newcommand{\bral}[1]{\left|#1\right|}
\newcommand{\ber}{\begin{eqnarray}}
\newcommand{\eer}{\end{eqnarray}}
\newcommand{\ETAL}{{\it et al.}}  
\newcommand{\be}{\begin{equation}}
\newcommand{\ee}{\end{equation}}
\newcommand{\ba}{\begin{eqnarray}}
\newcommand{\ea}{\end{eqnarray}}
\newcommand{\hld}{\hspace{0.25cm}\cdots}
\newcommand{\fns}{\footnotesize}
\newcommand{\fnsc}{\scriptsize}
\newcommand{\mbs}[1]{\mbox{\small #1}}
\newcommand{\la}{\langle}
\newcommand{\ra}{\rangle}
\usepackage{enumitem}% http://ctan.org/pkg/enumitem

 

\voffset .5in
\title{SKA1-Mid Simulations}
\author{S. Makhathini$^{1,2}$, O. M. Smirnov$^{1,2}$, M. Jarvis$^{3,4}$, F. B. Abdalla$^5$ \\{\footnotesize \it $^1$Rhodes
University, South Africa}\\ {\footnotesize \it  $^2$SKA South Africa}\\ 
{\footnotesize \it $^3$ University of the Western Cape, South Africa} \\ {\footnotesize \it $^4$University of Oxford} \\
{\footnotesize \it $^5$Department of Physics and Astronomy, University College London}} 
% \date{}
\begin{document}
\maketitle
\section{Introduction}
The SKA at mid frequencies (SKA-Mid) will be built in two distinct phases, SKA-Mid phase one (SKA1-Mid) and phase two (SKA2-Mid)
in South Africa, and SKA-SUR in phase 1 in Australia. The key science goals for SKA1-Mid include the study of the history and role
of neutral hydrogen in the Universe from the dark ages to the present-day, the use of pulsars as probes of fundamental physics
\cite{bd} and continuum and H{\sc i} surveys to pin down the cosmological model. \\In this document, we attempt to gauge the
scale-dependent sensitivity of the mid-frequency array described in the {\it SKA1 System Baseline Design} (BD) document \cite{bd}.
The performance of this configuration is then compared to four alternative configurations. We also explore different weighting and
tapering schemes as a way of increasing performance.\\ The rest of this document is laid out as follows: in \autoref{sec:sci-req}
we discuss the science requirements for SKA1-Mid, then in \autoref{sec:layouts} we describe layouts that we will be considering.
In \autoref{sec:exp} we describe our simulation techniques and the metrics we generated from the experiment. Finally, the results,
discussions and concluding remarks are in sections \autoref{sec:results} and \autoref{sec:conclusion}.

\subsection{SKA1-Mid Baseline Design}\label{sec:BL}
The mid-frequency array described in the BD is a 254 dish array with about $36\%$ of the dishes within a radius of 400m (core),
40\% of the dishes are between 400m and 4km (outer-core) with the remaining 24\% in 3 three spiral arm-like structures starting
from abot 2km and stretching out to a radius around 100km. Figures \ref{fig:lay} and \ref{fig:hist} show the BL layout and
baseline distribution histogram ($log_{10}$ space). With about 200 fifteen metre dishes (bar the 64 13.5~metre MeerKAT dishes)
within a radius of 4km, this array promises high sensitivity, with noise values around 63$\mu$Jy beam$^{-1}$ hr$^{1/2}$ \cite{bd}.
However, with three distinct dish-density regions, the resulting full sensitivity (natural weighting) PSF has two pedestals
corresponding to the abrupt changes in dish densities from the core to the outer-core and from the outer-core to the spiral arms
(see Appendix \ref{app:psf}). With such a configuration, a uv-weighting that tends towards uniform is required to obtain high
resolution and uv-tapering might be also be required to get a well behaved PSF. Leading to lower sensitivity due to the
down-weighting of baselines\footnote{uniform uv-weighting down-weights the uv-points corresponding to shorter baselines, and hence
gives better resolution, while a uv-taper down-weights the edges of the uv-coverage (longer baselines) to decrease sidelobe
levels}. It is therefore important to quantify the scale-dependent sensitivity of this configuration, i.e., what is the
sensitivity at different angular scales and how this sensitivity is affected by different uv-weighting and tapering schemes. 

\subsection{SKA1-Mid science requirements}\label{sec:sci-req}
The general scientific requirements for SKA1\cite{srd} (SRD) published by the SPO in March 2014 suggest that (at least for
SKA1-Mid), an array with a maximum baseline of around ~100km is required.
% \begin{table}[H]
%  \centering
%  \tiny{\begin{tabular}{l|c|c|c}\hline
%         Science goal & Frequency range [MHz] & Resolution & Max baseline required [km] \\ \hline \hline
%         SKA1-SCI-5 & 450 - 900 & 2$''$ & 41.91 \\ \hline
%         SKA1-SCI-8 & 900 - 1800 & 0.5$''$ & 83.83 \\ \hline
%         SKA1-SCI-27 & 8000 - 13000 & 0.1$''$ to 1$''$ & 47.15 \\ \hline
%         SKA1-SCI-12 & 1600 - 3000 & 0.5$''$ & 47.15 \\ \hline
%         SKA1-SCI-14 & 1400 - 1700 & 0.7$''$ & 38.49 \\ \hline
%         SKA1-SCI-16 & near 12000 & 0.04$''$ & 78.59 \\ \hline
%        \end{tabular}
%  }
% \end{table}
Therefore, we seek a layout with the shortest possible maximum baseline that does at least as well as the
``second generation'' baseline design in the resolution range 0.4-1 arcsec over 650, 800 and 1100MHz while not significantly
compromising the performance at the larger angular scales. Moreover, having a layout which performs just as well as (or better)
than the baseline layout but which covers significantly less space translates to less trenching, which may present the opportunity
to re-invest the funds somewhere else, therefore we consider the conservative addition of 12 dishes. However, we note that
improvements on the scales of interest are still possible without these 12 additional dishes. In the next section we present 4
alternate layouts, these layouts have maximum baselines of 90,100,120 and 133 km. The scale dependent sensitivity of these layouts
is compared to the baseline layout in section \ref{sec:exp}.

\subsection{Background on Layouts}\label{sec:layouts}
The following SKA1-Mid layouts are under consideration here:
\begin{description}
% \item[{\bf REF1}] Original baseline design layout (254 dishes) released by the SPO (April 2013).
\item[{\bf REF2A100B173 :}] This is the ”Second-generation” layout (254 dishes) produced by Robert Braun (September
2013)\footnote{We assume this to be the baseline layout.}. The spiral arms of this layout stretch out to ~100km, and the maximum
baseline is 173km. We also refere to this layout as REF2. 
\item[{\bf X$i$-$j$A$k$B$l$ :}] This is the REF2 layout with the core ``puffed up'' by 10\%, with $i$ dishes moved from
the outer core to the spiral arms and $j$ extra dishes added to the spiral arms. The spacing in the arms is then optimized to get
more sensitivity on the longer ($>50$km) baselines (See baseline distribution histograms in Figure \ref{fig:hist} ). Each spiral
arm stretches out to ~$k$ kilometres and the maximum baseline length is about $l$ kilometres.
\end{description}
The details of the layouts are tabulated in Table \ref{fig:lay}, and Figure \ref{fig:lay} shows the layouts and Figure
\ref{fig:hist} shows the baseline distribution histogram for the different layouts. Figure \ref{fig:uvcov} shows the uv-coverage
for the different layouts at 1.1GHz at declinations -50,-30,-10 degrees for 8hr tracks. At this point no optimisation has been
done on the antenna distributions and we emphasise that further improvements can be and should be made.
\begin{table}[H]
\centering
 \tiny{
 \begin{tabular}{l|cccc}\hline
 {\bf REF2A100B173} [254 dishes] & SKA dishes&  MeerKAT dishes & Both & \% \\\hline\hline
  Core & 63 & 30 & 93 & 36 \\
 Outer-core & 67 & 34 & 101 & 40 \\
 Spiral-arms & 60 & 0 & 60 & 24 \\\hline\hline
  {\bf X9-12A54B90} [266 dishes] &  & &  & \\\hline\hline
  Core & 70 & 30 & 93 & 36 \\
 Outer-core & 58 & 34 & 92 & 34 \\
 Spiral-arms & 81 & 0 & 81 & 30 \\\hline\hline
  {\bf X9-12A60B100} [266 dishes] &  & &  & \\\hline\hline
  Core & 70 & 30 & 93 & 36 \\
 Outer-core & 58 & 34 & 92 & 34 \\
 Spiral-arms & 81 & 0 & 81 & 30 \\\hline\hline
  {\bf X9-12A72B120} [266 dishes] &  & &  & \\\hline\hline
  Core & 70 & 30 & 93 & 36 \\
 Outer-core & 58 & 34 & 92 & 34 \\
 Spiral-arms & 81 & 0 & 81 & 30 \\\hline\hline
   {\bf X9-12A80B133} [266 dishes] &  & &  & \\\hline\hline
  Core & 70 & 30 & 93 & 36 \\
 Outer-core & 58 & 34 & 92 & 34 \\
 Spiral-arms & 81 & 0 & 81 & 30 \\\hline
 \end{tabular}}
 \caption{Breakdown of the layouts under consideration.}\label{tab:lay}
\end{table}

% layouts
\begin{figure}[H]
 \tiny{%%% autogen
 \begin{tabular}{lll}
\includegraphics[width=0.300000\textwidth,trim= 0 .05cm 0 0.05cm]{{images/lay*REF2*}.png} &\includegraphics[width=0.300000\textwidth,trim= 0 .05cm 0 0.05cm]{{images/outer*REF2*}.png} &\includegraphics[width=0.300000\textwidth,trim= 0 .05cm 0 0.05cm]{{images/core*REF2*}.png} 
 \\ \hfill\includegraphics[width=0.300000\textwidth,trim= 0 .05cm 0 0.05cm]{{images/lay_SKA1V8}.png} &\includegraphics[width=0.300000\textwidth,trim= 0 .05cm 0 0.05cm]{{images/lay_SKA1W8}.png} &\includegraphics[width=0.300000\textwidth,trim= 0 .05cm 0 0.05cm]{{images/outer_core_SKA1V8}.png} 
 \\ \hfill\includegraphics[width=0.300000\textwidth,trim= 0 .05cm 0 0.05cm]{{images/outer_core_SKA1W8}.png} &\includegraphics[width=0.300000\textwidth,trim= 0 .05cm 0 0.05cm]{{images/core_SKA1V8}.png} &\includegraphics[width=0.300000\textwidth,trim= 0 .05cm 0 0.05cm]{{images/core_SKA1W8}.png} 
 \\ \hfill\end{tabular}}
 \caption{Antenna layouts, REF2 ploted as a reference (red crosses)}\label{fig:lay}
\end{figure}
% baseline distribution histograms
\begin{figure}[H]
 \tiny{%%% autogen
 \begin{tabular}{cccccc}
\includegraphics[width=0.150000\textwidth,trim= 0 .05cm 0 0.05cm]{{images/hist_SKA1REF2}.png} &\includegraphics[width=0.150000\textwidth,trim= 0 .05cm 0 0.05cm]{{images/hist_SKA1W9-12A54B90}.png} &\includegraphics[width=0.150000\textwidth,trim= 0 .05cm 0 0.05cm]{{images/hist_SKA1W9-12A60B100}.png} &\includegraphics[width=0.150000\textwidth,trim= 0 .05cm 0 0.05cm]{{images/hist_SKA1W9-0A60B100}.png} &\includegraphics[width=0.150000\textwidth,trim= 0 .05cm 0 0.05cm]{{images/hist_SKA1W9-12A72B120}.png} &\includegraphics[width=0.150000\textwidth,trim= 0 .05cm 0 0.05cm]{{images/hist_SKA1W9-12A80B133}.png} 
 \\\end{tabular}}
 \caption{Baseline distribution with the uv-distance in $log_{10}$ km . Yellow and green dashed lines mark 10 and 120
kilometres respectively, and the pink strip represents baselines from 30-80km.}\label{fig:hist}
\end{figure}

% uv-coverage plots 
\begin{figure}[H]
 \tiny{%%% autogen
 \begin{tabular}{ccccc}
\includegraphics[width=0.180000\textwidth,trim= 0 .05cm 0 0.05cm]{{images/uvcov_SKA1REF2_-50_1100}.png} &\includegraphics[width=0.180000\textwidth,trim= 0 .05cm 0 0.05cm]{{images/uvcov_SKA1X9-12A54B90_-50_1100}.png} &\includegraphics[width=0.180000\textwidth,trim= 0 .05cm 0 0.05cm]{{images/uvcov_SKA1X9-12A60B100_-50_1100}.png} &\includegraphics[width=0.180000\textwidth,trim= 0 .05cm 0 0.05cm]{{images/uvcov_SKA1X9-12A72B120_-50_1100}.png} &\includegraphics[width=0.180000\textwidth,trim= 0 .05cm 0 0.05cm]{{images/uvcov_SKA1X9-12A80B133_-50_1100}.png} 
 \\\includegraphics[width=0.180000\textwidth,trim= 0 .05cm 0 0.05cm]{{images/uvcov_SKA1REF2_-30_1100}.png} &\includegraphics[width=0.180000\textwidth,trim= 0 .05cm 0 0.05cm]{{images/uvcov_SKA1X9-12A54B90_-30_1100}.png} &\includegraphics[width=0.180000\textwidth,trim= 0 .05cm 0 0.05cm]{{images/uvcov_SKA1X9-12A60B100_-30_1100}.png} &\includegraphics[width=0.180000\textwidth,trim= 0 .05cm 0 0.05cm]{{images/uvcov_SKA1X9-12A72B120_-30_1100}.png} &\includegraphics[width=0.180000\textwidth,trim= 0 .05cm 0 0.05cm]{{images/uvcov_SKA1X9-12A80B133_-30_1100}.png} 
 \\\includegraphics[width=0.180000\textwidth,trim= 0 .05cm 0 0.05cm]{{images/uvcov_SKA1REF2_-10_1100}.png} &\includegraphics[width=0.180000\textwidth,trim= 0 .05cm 0 0.05cm]{{images/uvcov_SKA1X9-12A54B90_-10_1100}.png} &\includegraphics[width=0.180000\textwidth,trim= 0 .05cm 0 0.05cm]{{images/uvcov_SKA1X9-12A60B100_-10_1100}.png} &\includegraphics[width=0.180000\textwidth,trim= 0 .05cm 0 0.05cm]{{images/uvcov_SKA1X9-12A72B120_-10_1100}.png} &\includegraphics[width=0.180000\textwidth,trim= 0 .05cm 0 0.05cm]{{images/uvcov_SKA1X9-12A80B133_-10_1100}.png} 
 \\\end{tabular}}
 \caption{UV-Coverage for 8-hr tracks at 1.1 GHz (50MHz bandwidth) at declinations -50,-30,-10 for the different layouts. Blue
indicates uv-points, red indicates conjugate uv-points.}\label{fig:uvcov}
\end{figure}

\section{The Experiment}\label{sec:exp}
Our aim is to investigate the scale-dependent sensitivity of the layouts described in the previous section.
We use the \texttt{makems} tool make simulated measurement sets of a an 8hr track with a 60s integration time on declinations
\{-50, -30, -10\} degrees at frequencies of \{650, 800, 1100\}MHz with a single 50MHz channel. The expected rms noise per real
and imaginary part for each visibility is calculated as 
\begin{equation}
\sigma_{\text{vis}} = \frac{\text{SEFD}}{\sqrt{2\Delta t\Delta \nu}}.
\end{equation}
We use the baseline designs SEFD value of 400 corresponding to the 15 m dishes. We then fill the MS with random Gaussian noise
using the computed value of the noise for a given integration and bandwidth. We then use the (CASA-derived) \texttt{lwimager} tool
to make maps of the PSF, and dirty maps of the noise using various weighting schemes. Note that for uniform and robust weighting,
a crucial parameter is the size of the uv-bin over which weights are “uniformized”. By default this is determined from the full
image size, but \texttt{lwimager} allows one to uniformize the weights over bins corresponding to a user-defined FoV instead. For
these simulations uv-bins corresponding to a FoV of 10 arcmin were used. The following metrics were generated:\\ {\bf Note:} These
metrics are generated at different angular scales, this is done by applying an inner-taper\footnote{The weights for the taper
are generated using a Butterworth function.} to taper out baselines
that do not fall with a given resolution range, i.e., only
considering uv-points that correspond to a given resolution.
\begin{itemize}
 \item PSF FWHM size (mean of the FWHM dimensions). This was measured by making high-resolution images of the PSF (0.05 arcsec
resolution), and fitting a Gaussian to the PSF. Note that for the highly non-Gaussian PSFs corresponding to natural and (some)
robust weighting schemes, the fit is very poor, so the size parameter becomes somewhat ill-defined (Table \ref{tab:psf_mean}).
A catalog of PSF cross-sections is provided in the Appendix \ref{app:psf}

 \item PSF symmetry (PSF size parameters are obtained as explained above). As a measure of PSF symmetry, we define 
$\text{PSF}_{sym}=1-\text{FWHM}_{min}/\text{FWHM}_{maj}$, then $\text{PSF}_{sym} = 0$ is perfect symmetry, and the symmetry
degenerates as $\text{PSF}_{sym}\,\,\, \rightarrow\,\,1$ (Table \ref{tab:psf_sym}).

 \item Rms pixel noise at different angular scales for 50 and 166MHz wide bands (Tables \ref{tab:noise50} and \ref{tab:noise166}).
 
 \item SNR for a 10$\mu$Jy source at 1100MHz with a spectral index of -0.7 after 8hrs for a 166MHz band (Table \ref{tab:snr10}).
 \item Average SNR over frequencies 650, 800 and 1100MHz (166MHz band)
   after 8 hours, for a 10$\mu$Jy source at 1100MHz
with an spectral index of -0.7 (Table \ref{tab:snravg}). {$\overline{SNR10}=\sqrt{SNR10_{650}^2 + SNR10_{800}^2 + SNR10_{1100}^2}$}.
 \item Hours required to reach a mean SNR of 10 (Table \ref{tab:hours}).
\end{itemize}
\section{Results}\label{sec:results}
%===========================================================Performance Stats===============================================
% Auto generated table
 \vspace{-2cm}\begin{table}[H]
 \tiny{\subfloat[DEC=-10, natural weighting]{\begin{tabular}{|lccccc||ccccc||ccccc|} 
 \\ \cline{2-16} \multicolumn{1}{c}{ } & \multicolumn{5}{|c}{650MHz}  & \multicolumn{5}{c}{800MHz}  & \multicolumn{5}{c|}{1100MHz} \\ \cline{1-16} 
 resbin  &1 & 2 & 3 & 4 & 5 & 1 & 2 & 3 & 4 & 5 & 1 & 2 & 3 & 4 & 5 \\ \hline
SKA1REF2 & 0.65 \cellcolor{blue!18.00} & 1.33 \cellcolor{red!60.00} & 2.33 \cellcolor{green!39.98} & 3.35 \cellcolor{orange!60.00} & 803.08 \cellcolor{purple!60.00} & 0.60 \cellcolor{blue!18.00} & 1.33 \cellcolor{red!60.00} & 2.35 \cellcolor{green!25.87} & 3.34 \cellcolor{orange!59.86} & 792.32 \cellcolor{purple!60.00} & 0.59 \cellcolor{blue!31.03} & 1.33 \cellcolor{red!60.00} & 2.35 \cellcolor{green!60.00} & 3.35 \cellcolor{orange!60.00} & 763.61 \cellcolor{purple!21.13}\\ \hline 
SKA1W9-12A54B90 & 0.85 \cellcolor{blue!60.00} & 1.29 \cellcolor{red!43.04} & 2.32 \cellcolor{green!18.00} & 3.34 \cellcolor{orange!42.46} & 794.44 \cellcolor{purple!18.00} & 0.81 \cellcolor{blue!60.00} & 1.24 \cellcolor{red!20.21} & 2.35 \cellcolor{green!33.00} & 3.33 \cellcolor{orange!51.52} & 790.43 \cellcolor{purple!26.76} & 0.69 \cellcolor{blue!60.00} & 1.28 \cellcolor{red!18.00} & 2.34 \cellcolor{green!50.17} & 3.33 \cellcolor{orange!20.51} & 767.45 \cellcolor{purple!56.57}\\ \hline 
SKA1W9-12A60B100 & 0.84 \cellcolor{blue!56.59} & 1.25 \cellcolor{red!27.98} & 2.32 \cellcolor{green!21.46} & 3.33 \cellcolor{orange!22.95} & 796.90 \cellcolor{purple!29.96} & 0.77 \cellcolor{blue!53.01} & 1.23 \cellcolor{red!18.00} & 2.36 \cellcolor{green!60.00} & 3.34 \cellcolor{orange!60.00} & 789.93 \cellcolor{purple!18.00} & 0.64 \cellcolor{blue!47.03} & 1.30 \cellcolor{red!34.41} & 2.34 \cellcolor{green!54.60} & 3.34 \cellcolor{orange!28.03} & 767.82 \cellcolor{purple!60.00}\\ \hline 
SKA1W9-12A72B120 & 0.78 \cellcolor{blue!44.81} & 1.22 \cellcolor{red!18.00} & 2.35 \cellcolor{green!60.00} & 3.35 \cellcolor{orange!47.70} & 798.73 \cellcolor{purple!38.85} & 0.70 \cellcolor{blue!39.05} & 1.28 \cellcolor{red!37.90} & 2.35 \cellcolor{green!18.00} & 3.32 \cellcolor{orange!43.73} & 790.38 \cellcolor{purple!26.03} & 0.57 \cellcolor{blue!25.70} & 1.32 \cellcolor{red!52.95} & 2.32 \cellcolor{green!38.48} & 3.33 \cellcolor{orange!18.00} & 763.27 \cellcolor{purple!18.00}\\ \hline 
SKA1W9-12A80B133 & 0.74 \cellcolor{blue!36.60} & 1.25 \cellcolor{red!27.03} & 2.34 \cellcolor{green!42.99} & 3.32 \cellcolor{orange!18.00} & 797.36 \cellcolor{purple!32.22} & 0.65 \cellcolor{blue!29.03} & 1.30 \cellcolor{red!45.94} & 2.36 \cellcolor{green!57.00} & 3.28 \cellcolor{orange!18.00} & 790.65 \cellcolor{purple!30.63} & 0.54 \cellcolor{blue!18.00} & 1.32 \cellcolor{red!54.72} & 2.30 \cellcolor{green!18.00} & 3.33 \cellcolor{orange!19.04} & 767.22 \cellcolor{purple!54.47}\\ \hline 
\end{tabular}}
\vspace{-0.300000cm}
\hspace{1cm} 
\subfloat[DEC=-10, robust-2 weighting ]{\begin{tabular}{|lccccc||ccccc||ccccc|} 
 \\ \cline{2-16} \multicolumn{1}{c}{ } & \multicolumn{5}{|c}{650MHz}  & \multicolumn{5}{c}{800MHz}  & \multicolumn{5}{c|}{1100MHz} \\ \cline{1-16} 
 resbin  &1 & 2 & 3 & 4 & 5 & 1 & 2 & 3 & 4 & 5 & 1 & 2 & 3 & 4 & 5 \\ \hline
SKA1REF2 & 0.73 \cellcolor{blue!20.97} & 1.22 \cellcolor{red!60.00} & 2.25 \cellcolor{green!60.00} & 3.26 \cellcolor{orange!60.00} & 745.23 \cellcolor{purple!60.00} & 0.62 \cellcolor{blue!19.20} & 1.21 \cellcolor{red!60.00} & 2.24 \cellcolor{green!60.00} & 3.25 \cellcolor{orange!59.39} & 791.85 \cellcolor{purple!60.00} & 0.53 \cellcolor{blue!25.74} & 1.19 \cellcolor{red!60.00} & 2.24 \cellcolor{green!60.00} & 3.26 \cellcolor{orange!42.50} & 761.50 \cellcolor{purple!18.00}\\ \hline 
SKA1W9-12A54B90 & 0.96 \cellcolor{blue!60.00} & 1.22 \cellcolor{red!54.94} & 2.23 \cellcolor{green!18.00} & 3.25 \cellcolor{orange!37.87} & 732.34 \cellcolor{purple!18.00} & 0.78 \cellcolor{blue!60.00} & 1.16 \cellcolor{red!30.00} & 2.23 \cellcolor{green!18.00} & 3.25 \cellcolor{orange!58.78} & 790.56 \cellcolor{purple!33.85} & 0.59 \cellcolor{blue!60.00} & 1.15 \cellcolor{red!18.00} & 2.24 \cellcolor{green!43.54} & 3.25 \cellcolor{orange!33.27} & 766.98 \cellcolor{purple!57.61}\\ \hline 
SKA1W9-12A60B100 & 0.88 \cellcolor{blue!46.78} & 1.19 \cellcolor{red!36.63} & 2.23 \cellcolor{green!28.20} & 3.25 \cellcolor{orange!38.62} & 735.91 \cellcolor{purple!29.65} & 0.72 \cellcolor{blue!45.46} & 1.16 \cellcolor{red!23.49} & 2.24 \cellcolor{green!32.73} & 3.25 \cellcolor{orange!60.00} & 789.78 \cellcolor{purple!18.00} & 0.57 \cellcolor{blue!47.57} & 1.16 \cellcolor{red!24.36} & 2.24 \cellcolor{green!52.62} & 3.25 \cellcolor{orange!32.00} & 767.31 \cellcolor{purple!60.00}\\ \hline 
SKA1W9-12A72B120 & 0.77 \cellcolor{blue!28.28} & 1.16 \cellcolor{red!21.39} & 2.24 \cellcolor{green!42.02} & 3.25 \cellcolor{orange!39.38} & 736.00 \cellcolor{purple!29.93} & 0.64 \cellcolor{blue!25.58} & 1.15 \cellcolor{red!18.00} & 2.24 \cellcolor{green!34.61} & 3.25 \cellcolor{orange!28.96} & 790.19 \cellcolor{purple!26.29} & 0.53 \cellcolor{blue!29.37} & 1.17 \cellcolor{red!36.24} & 2.23 \cellcolor{green!18.00} & 3.25 \cellcolor{orange!18.00} & 761.79 \cellcolor{purple!20.11}\\ \hline 
SKA1W9-12A80B133 & 0.71 \cellcolor{blue!18.00} & 1.16 \cellcolor{red!18.00} & 2.24 \cellcolor{green!45.31} & 3.25 \cellcolor{orange!18.00} & 734.91 \cellcolor{purple!26.39} & 0.61 \cellcolor{blue!18.00} & 1.15 \cellcolor{red!20.31} & 2.23 \cellcolor{green!30.54} & 3.25 \cellcolor{orange!18.00} & 790.41 \cellcolor{purple!30.68} & 0.51 \cellcolor{blue!18.00} & 1.17 \cellcolor{red!39.68} & 2.23 \cellcolor{green!21.41} & 3.26 \cellcolor{orange!60.00} & 766.24 \cellcolor{purple!52.27}\\ \hline 
\end{tabular}}
\vspace{-0.300000cm}
\hspace{1cm} 
\subfloat[DEC=-10, robust-2 weighting with a 1 arcsec Gaussian taper]{\begin{tabular}{|lccccc||ccccc||ccccc|} 
 \\ \cline{2-16} \multicolumn{1}{c}{ } & \multicolumn{5}{|c}{650MHz}  & \multicolumn{5}{c}{800MHz}  & \multicolumn{5}{c|}{1100MHz} \\ \cline{1-16} 
 resbin  &1 & 2 & 3 & 4 & 5 & 1 & 2 & 3 & 4 & 5 & 1 & 2 & 3 & 4 & 5 \\ \hline
SKA1REF2 & 1.03 \cellcolor{blue!32.23} & 1.33 \cellcolor{red!60.00} & 2.27 \cellcolor{green!60.00} & 3.27 \cellcolor{orange!60.00} & 745.23 \cellcolor{purple!60.00} & 0.98 \cellcolor{blue!46.77} & 1.31 \cellcolor{red!60.00} & 2.27 \cellcolor{green!60.00} & 3.26 \cellcolor{orange!58.78} & 791.85 \cellcolor{purple!60.00} & 0.94 \cellcolor{blue!60.00} & 1.29 \cellcolor{red!60.00} & 2.26 \cellcolor{green!60.00} & 3.26 \cellcolor{orange!42.69} & 761.50 \cellcolor{purple!18.00}\\ \hline 
SKA1W9-12A54B90 & 1.15 \cellcolor{blue!60.00} & 1.31 \cellcolor{red!48.60} & 2.25 \cellcolor{green!18.00} & 3.26 \cellcolor{orange!38.75} & 732.34 \cellcolor{purple!18.00} & 1.02 \cellcolor{blue!60.00} & 1.26 \cellcolor{red!28.29} & 2.25 \cellcolor{green!18.00} & 3.26 \cellcolor{orange!59.39} & 790.56 \cellcolor{purple!33.84} & 0.91 \cellcolor{blue!43.52} & 1.25 \cellcolor{red!18.00} & 2.26 \cellcolor{green!44.68} & 3.26 \cellcolor{orange!33.71} & 766.98 \cellcolor{purple!57.61}\\ \hline 
SKA1W9-12A60B100 & 1.09 \cellcolor{blue!44.85} & 1.28 \cellcolor{red!32.87} & 2.25 \cellcolor{green!28.36} & 3.26 \cellcolor{orange!39.00} & 735.91 \cellcolor{purple!29.65} & 0.98 \cellcolor{blue!44.83} & 1.25 \cellcolor{red!23.18} & 2.26 \cellcolor{green!32.86} & 3.26 \cellcolor{orange!60.00} & 789.78 \cellcolor{purple!18.00} & 0.89 \cellcolor{blue!32.89} & 1.26 \cellcolor{red!23.72} & 2.26 \cellcolor{green!52.62} & 3.26 \cellcolor{orange!32.43} & 767.31 \cellcolor{purple!60.00}\\ \hline 
SKA1W9-12A72B120 & 1.01 \cellcolor{blue!26.99} & 1.26 \cellcolor{red!20.14} & 2.26 \cellcolor{green!42.18} & 3.26 \cellcolor{orange!39.75} & 736.00 \cellcolor{purple!29.93} & 0.93 \cellcolor{blue!26.08} & 1.25 \cellcolor{red!18.00} & 2.26 \cellcolor{green!33.51} & 3.26 \cellcolor{orange!28.35} & 790.19 \cellcolor{purple!26.29} & 0.87 \cellcolor{blue!20.55} & 1.26 \cellcolor{red!33.56} & 2.25 \cellcolor{green!18.00} & 3.26 \cellcolor{orange!18.00} & 761.79 \cellcolor{purple!20.11}\\ \hline 
SKA1W9-12A80B133 & 0.98 \cellcolor{blue!18.00} & 1.25 \cellcolor{red!18.00} & 2.26 \cellcolor{green!45.45} & 3.25 \cellcolor{orange!18.00} & 734.91 \cellcolor{purple!26.39} & 0.91 \cellcolor{blue!18.00} & 1.25 \cellcolor{red!19.92} & 2.26 \cellcolor{green!28.02} & 3.25 \cellcolor{orange!18.00} & 790.41 \cellcolor{purple!30.68} & 0.87 \cellcolor{blue!18.00} & 1.27 \cellcolor{red!36.54} & 2.25 \cellcolor{green!20.84} & 3.27 \cellcolor{orange!60.00} & 766.24 \cellcolor{purple!52.27}\\ \hline 
\end{tabular}}
\vspace{-0.300000cm}
\hspace{1cm} 
\subfloat[DEC=-30, natural weighting]{\begin{tabular}{|lccccc||ccccc||ccccc|} 
 \\ \cline{2-16} \multicolumn{1}{c}{ } & \multicolumn{5}{|c}{650MHz}  & \multicolumn{5}{c}{800MHz}  & \multicolumn{5}{c|}{1100MHz} \\ \cline{1-16} 
 resbin  &1 & 2 & 3 & 4 & 5 & 1 & 2 & 3 & 4 & 5 & 1 & 2 & 3 & 4 & 5 \\ \hline
SKA1REF2 & 0.65 \cellcolor{blue!18.00} & 1.34 \cellcolor{red!60.00} & 2.35 \cellcolor{green!32.09} & 3.37 \cellcolor{orange!60.00} & 797.29 \cellcolor{purple!60.00} & 0.60 \cellcolor{blue!18.00} & 1.33 \cellcolor{red!60.00} & 2.35 \cellcolor{green!41.86} & 3.34 \cellcolor{orange!53.46} & 790.58 \cellcolor{purple!33.54} & 0.59 \cellcolor{blue!33.92} & 1.33 \cellcolor{red!60.00} & 2.35 \cellcolor{green!60.00} & 3.35 \cellcolor{orange!60.00} & 759.20 \cellcolor{purple!47.21}\\ \hline 
SKA1W9-12A54B90 & 0.87 \cellcolor{blue!60.00} & 1.27 \cellcolor{red!35.33} & 2.35 \cellcolor{green!39.77} & 3.35 \cellcolor{orange!38.66} & 788.73 \cellcolor{purple!18.00} & 0.81 \cellcolor{blue!60.00} & 1.22 \cellcolor{red!18.00} & 2.35 \cellcolor{green!31.94} & 3.33 \cellcolor{orange!45.90} & 791.56 \cellcolor{purple!60.00} & 0.68 \cellcolor{blue!60.00} & 1.30 \cellcolor{red!18.00} & 2.34 \cellcolor{green!52.65} & 3.34 \cellcolor{orange!35.52} & 761.56 \cellcolor{purple!57.08}\\ \hline 
SKA1W9-12A60B100 & 0.85 \cellcolor{blue!55.61} & 1.24 \cellcolor{red!22.48} & 2.34 \cellcolor{green!18.00} & 3.33 \cellcolor{orange!19.79} & 790.37 \cellcolor{purple!26.07} & 0.77 \cellcolor{blue!52.36} & 1.22 \cellcolor{red!19.91} & 2.36 \cellcolor{green!60.00} & 3.35 \cellcolor{orange!60.00} & 790.15 \cellcolor{purple!21.74} & 0.63 \cellcolor{blue!46.41} & 1.31 \cellcolor{red!34.37} & 2.34 \cellcolor{green!53.14} & 3.33 \cellcolor{orange!18.00} & 762.26 \cellcolor{purple!60.00}\\ \hline 
SKA1W9-12A72B120 & 0.78 \cellcolor{blue!43.43} & 1.22 \cellcolor{red!18.00} & 2.37 \cellcolor{green!60.00} & 3.35 \cellcolor{orange!39.11} & 792.88 \cellcolor{purple!38.36} & 0.69 \cellcolor{blue!37.03} & 1.29 \cellcolor{red!44.74} & 2.34 \cellcolor{green!18.00} & 3.31 \cellcolor{orange!34.88} & 790.22 \cellcolor{purple!23.64} & 0.56 \cellcolor{blue!25.14} & 1.33 \cellcolor{red!60.00} & 2.32 \cellcolor{green!32.54} & 3.34 \cellcolor{orange!45.26} & 752.20 \cellcolor{purple!18.00}\\ \hline 
SKA1W9-12A80B133 & 0.74 \cellcolor{blue!35.36} & 1.27 \cellcolor{red!33.64} & 2.34 \cellcolor{green!28.37} & 3.33 \cellcolor{orange!18.00} & 791.02 \cellcolor{purple!29.27} & 0.64 \cellcolor{blue!26.86} & 1.31 \cellcolor{red!53.84} & 2.36 \cellcolor{green!52.78} & 3.28 \cellcolor{orange!18.00} & 790.01 \cellcolor{purple!18.00} & 0.53 \cellcolor{blue!18.00} & 1.33 \cellcolor{red!51.03} & 2.30 \cellcolor{green!18.00} & 3.34 \cellcolor{orange!41.64} & 762.18 \cellcolor{purple!59.66}\\ \hline 
\end{tabular}}
\vspace{-0.300000cm}
\hspace{1cm} 
\subfloat[DEC=-30, robust-2 weighting ]{\begin{tabular}{|lccccc||ccccc||ccccc|} 
 \\ \cline{2-16} \multicolumn{1}{c}{ } & \multicolumn{5}{|c}{650MHz}  & \multicolumn{5}{c}{800MHz}  & \multicolumn{5}{c|}{1100MHz} \\ \cline{1-16} 
 resbin  &1 & 2 & 3 & 4 & 5 & 1 & 2 & 3 & 4 & 5 & 1 & 2 & 3 & 4 & 5 \\ \hline
SKA1REF2 & 0.71 \cellcolor{blue!20.51} & 1.21 \cellcolor{red!60.00} & 2.24 \cellcolor{green!60.00} & 3.26 \cellcolor{orange!60.00} & 737.52 \cellcolor{purple!60.00} & 0.60 \cellcolor{blue!18.00} & 1.19 \cellcolor{red!60.00} & 2.24 \cellcolor{green!60.00} & 3.24 \cellcolor{orange!33.98} & 789.76 \cellcolor{purple!19.45} & 0.51 \cellcolor{blue!26.97} & 1.18 \cellcolor{red!60.00} & 2.24 \cellcolor{green!60.00} & 3.25 \cellcolor{orange!26.91} & 757.10 \cellcolor{purple!43.02}\\ \hline 
SKA1W9-12A54B90 & 0.93 \cellcolor{blue!60.00} & 1.19 \cellcolor{red!51.44} & 2.23 \cellcolor{green!18.00} & 3.25 \cellcolor{orange!52.61} & 729.39 \cellcolor{purple!18.00} & 0.76 \cellcolor{blue!60.00} & 1.15 \cellcolor{red!25.19} & 2.24 \cellcolor{green!33.21} & 3.25 \cellcolor{orange!49.22} & 791.60 \cellcolor{purple!60.00} & 0.58 \cellcolor{blue!60.00} & 1.14 \cellcolor{red!18.00} & 2.23 \cellcolor{green!29.70} & 3.25 \cellcolor{orange!32.51} & 760.62 \cellcolor{purple!57.06}\\ \hline 
SKA1W9-12A60B100 & 0.86 \cellcolor{blue!46.93} & 1.17 \cellcolor{red!31.82} & 2.23 \cellcolor{green!21.20} & 3.25 \cellcolor{orange!43.73} & 729.56 \cellcolor{purple!18.87} & 0.70 \cellcolor{blue!45.59} & 1.14 \cellcolor{red!20.53} & 2.24 \cellcolor{green!52.40} & 3.25 \cellcolor{orange!60.00} & 789.93 \cellcolor{purple!23.05} & 0.55 \cellcolor{blue!48.79} & 1.15 \cellcolor{red!30.69} & 2.23 \cellcolor{green!42.30} & 3.25 \cellcolor{orange!18.00} & 761.36 \cellcolor{purple!60.00}\\ \hline 
SKA1W9-12A72B120 & 0.75 \cellcolor{blue!28.19} & 1.15 \cellcolor{red!20.36} & 2.24 \cellcolor{green!55.37} & 3.25 \cellcolor{orange!48.17} & 732.45 \cellcolor{purple!33.81} & 0.63 \cellcolor{blue!25.59} & 1.14 \cellcolor{red!18.00} & 2.23 \cellcolor{green!18.00} & 3.24 \cellcolor{orange!18.00} & 789.92 \cellcolor{purple!22.84} & 0.52 \cellcolor{blue!29.87} & 1.16 \cellcolor{red!41.14} & 2.22 \cellcolor{green!18.00} & 3.25 \cellcolor{orange!28.95} & 750.82 \cellcolor{purple!18.00}\\ \hline 
SKA1W9-12A80B133 & 0.70 \cellcolor{blue!18.00} & 1.14 \cellcolor{red!18.00} & 2.23 \cellcolor{green!41.49} & 3.24 \cellcolor{orange!18.00} & 729.42 \cellcolor{purple!18.11} & 0.60 \cellcolor{blue!18.20} & 1.15 \cellcolor{red!27.89} & 2.24 \cellcolor{green!32.84} & 3.24 \cellcolor{orange!26.92} & 789.70 \cellcolor{purple!18.00} & 0.50 \cellcolor{blue!18.00} & 1.16 \cellcolor{red!45.53} & 2.23 \cellcolor{green!24.60} & 3.26 \cellcolor{orange!60.00} & 761.07 \cellcolor{purple!58.84}\\ \hline 
\end{tabular}}
\vspace{-0.300000cm}
\hspace{1cm} 
\subfloat[DEC=-30, robust-2 weighting with a 1 arcsec Gaussian taper]{\begin{tabular}{|lccccc||ccccc||ccccc|} 
 \\ \cline{2-16} \multicolumn{1}{c}{ } & \multicolumn{5}{|c}{650MHz}  & \multicolumn{5}{c}{800MHz}  & \multicolumn{5}{c|}{1100MHz} \\ \cline{1-16} 
 resbin  &1 & 2 & 3 & 4 & 5 & 1 & 2 & 3 & 4 & 5 & 1 & 2 & 3 & 4 & 5 \\ \hline
SKA1REF2 & 1.02 \cellcolor{blue!33.51} & 1.31 \cellcolor{red!60.00} & 2.26 \cellcolor{green!60.00} & 3.26 \cellcolor{orange!60.00} & 737.52 \cellcolor{purple!60.00} & 0.97 \cellcolor{blue!47.66} & 1.29 \cellcolor{red!60.00} & 2.27 \cellcolor{green!60.00} & 3.25 \cellcolor{orange!34.50} & 789.76 \cellcolor{purple!19.45} & 0.93 \cellcolor{blue!60.00} & 1.28 \cellcolor{red!60.00} & 2.26 \cellcolor{green!60.00} & 3.26 \cellcolor{orange!26.96} & 757.10 \cellcolor{purple!43.02}\\ \hline 
SKA1W9-12A54B90 & 1.13 \cellcolor{blue!60.00} & 1.29 \cellcolor{red!44.18} & 2.25 \cellcolor{green!18.00} & 3.26 \cellcolor{orange!52.95} & 729.39 \cellcolor{purple!18.00} & 1.00 \cellcolor{blue!60.00} & 1.25 \cellcolor{red!22.96} & 2.26 \cellcolor{green!34.07} & 3.26 \cellcolor{orange!50.25} & 791.60 \cellcolor{purple!60.00} & 0.90 \cellcolor{blue!42.01} & 1.25 \cellcolor{red!18.00} & 2.25 \cellcolor{green!31.09} & 3.26 \cellcolor{orange!32.85} & 760.62 \cellcolor{purple!57.06}\\ \hline 
SKA1W9-12A60B100 & 1.07 \cellcolor{blue!45.51} & 1.26 \cellcolor{red!27.65} & 2.25 \cellcolor{green!20.67} & 3.26 \cellcolor{orange!44.43} & 729.56 \cellcolor{purple!18.87} & 0.97 \cellcolor{blue!45.72} & 1.25 \cellcolor{red!19.51} & 2.26 \cellcolor{green!52.70} & 3.26 \cellcolor{orange!60.00} & 789.93 \cellcolor{purple!23.05} & 0.89 \cellcolor{blue!32.05} & 1.25 \cellcolor{red!29.90} & 2.26 \cellcolor{green!42.04} & 3.25 \cellcolor{orange!18.00} & 761.36 \cellcolor{purple!60.00}\\ \hline 
SKA1W9-12A72B120 & 1.00 \cellcolor{blue!27.21} & 1.25 \cellcolor{red!18.85} & 2.26 \cellcolor{green!55.04} & 3.26 \cellcolor{orange!48.25} & 732.45 \cellcolor{purple!33.81} & 0.92 \cellcolor{blue!26.64} & 1.24 \cellcolor{red!18.00} & 2.25 \cellcolor{green!18.00} & 3.25 \cellcolor{orange!18.00} & 789.92 \cellcolor{purple!22.84} & 0.87 \cellcolor{blue!20.53} & 1.26 \cellcolor{red!37.93} & 2.25 \cellcolor{green!18.00} & 3.26 \cellcolor{orange!29.27} & 750.82 \cellcolor{purple!18.00}\\ \hline 
SKA1W9-12A80B133 & 0.96 \cellcolor{blue!18.00} & 1.25 \cellcolor{red!18.00} & 2.26 \cellcolor{green!40.91} & 3.25 \cellcolor{orange!18.00} & 729.42 \cellcolor{purple!18.12} & 0.90 \cellcolor{blue!18.00} & 1.25 \cellcolor{red!27.49} & 2.26 \cellcolor{green!30.78} & 3.25 \cellcolor{orange!26.62} & 789.70 \cellcolor{purple!18.00} & 0.87 \cellcolor{blue!18.00} & 1.26 \cellcolor{red!41.81} & 2.25 \cellcolor{green!24.39} & 3.27 \cellcolor{orange!60.00} & 761.07 \cellcolor{purple!58.84}\\ \hline 
\end{tabular}}
\vspace{-0.300000cm}
\hspace{1cm} 
\subfloat[DEC=-50, natural weighting]{\begin{tabular}{|lccccc||ccccc||ccccc|} 
 \\ \cline{2-16} \multicolumn{1}{c}{ } & \multicolumn{5}{|c}{650MHz}  & \multicolumn{5}{c}{800MHz}  & \multicolumn{5}{c|}{1100MHz} \\ \cline{1-16} 
 resbin  &1 & 2 & 3 & 4 & 5 & 1 & 2 & 3 & 4 & 5 & 1 & 2 & 3 & 4 & 5 \\ \hline
SKA1REF2 & 0.65 \cellcolor{blue!18.00} & 1.34 \cellcolor{red!60.00} & 2.34 \cellcolor{green!41.29} & 3.35 \cellcolor{orange!60.00} & 799.83 \cellcolor{purple!60.00} & 0.60 \cellcolor{blue!18.00} & 1.33 \cellcolor{red!60.00} & 2.36 \cellcolor{green!46.69} & 3.34 \cellcolor{orange!60.00} & 789.57 \cellcolor{purple!60.00} & 0.58 \cellcolor{blue!32.25} & 1.33 \cellcolor{red!60.00} & 2.34 \cellcolor{green!58.72} & 3.36 \cellcolor{orange!60.00} & 763.91 \cellcolor{purple!60.00}\\ \hline 
SKA1W9-12A54B90 & 0.87 \cellcolor{blue!60.00} & 1.27 \cellcolor{red!37.00} & 2.34 \cellcolor{green!31.36} & 3.33 \cellcolor{orange!36.54} & 793.72 \cellcolor{purple!18.00} & 0.81 \cellcolor{blue!60.00} & 1.22 \cellcolor{red!18.00} & 2.35 \cellcolor{green!35.44} & 3.33 \cellcolor{orange!51.45} & 786.93 \cellcolor{purple!26.30} & 0.68 \cellcolor{blue!60.00} & 1.31 \cellcolor{red!18.00} & 2.34 \cellcolor{green!55.30} & 3.34 \cellcolor{orange!18.00} & 761.27 \cellcolor{purple!44.82}\\ \hline 
SKA1W9-12A60B100 & 0.85 \cellcolor{blue!56.48} & 1.24 \cellcolor{red!23.60} & 2.33 \cellcolor{green!18.00} & 3.33 \cellcolor{orange!26.37} & 796.94 \cellcolor{purple!40.15} & 0.77 \cellcolor{blue!52.66} & 1.22 \cellcolor{red!18.11} & 2.36 \cellcolor{green!60.00} & 3.34 \cellcolor{orange!58.50} & 787.46 \cellcolor{purple!33.00} & 0.64 \cellcolor{blue!46.86} & 1.32 \cellcolor{red!35.04} & 2.34 \cellcolor{green!60.00} & 3.34 \cellcolor{orange!27.33} & 761.29 \cellcolor{purple!44.91}\\ \hline 
SKA1W9-12A72B120 & 0.79 \cellcolor{blue!44.29} & 1.22 \cellcolor{red!18.00} & 2.35 \cellcolor{green!60.00} & 3.34 \cellcolor{orange!40.97} & 796.68 \cellcolor{purple!38.34} & 0.70 \cellcolor{blue!37.95} & 1.28 \cellcolor{red!42.98} & 2.35 \cellcolor{green!18.00} & 3.32 \cellcolor{orange!44.88} & 786.41 \cellcolor{purple!19.67} & 0.56 \cellcolor{blue!26.13} & 1.33 \cellcolor{red!55.52} & 2.32 \cellcolor{green!37.21} & 3.34 \cellcolor{orange!26.94} & 756.61 \cellcolor{purple!18.00}\\ \hline 
SKA1W9-12A80B133 & 0.74 \cellcolor{blue!36.07} & 1.26 \cellcolor{red!31.52} & 2.33 \cellcolor{green!26.40} & 3.32 \cellcolor{orange!18.00} & 796.20 \cellcolor{purple!35.05} & 0.65 \cellcolor{blue!27.96} & 1.31 \cellcolor{red!53.75} & 2.36 \cellcolor{green!52.20} & 3.27 \cellcolor{orange!18.00} & 786.28 \cellcolor{purple!18.00} & 0.53 \cellcolor{blue!18.00} & 1.33 \cellcolor{red!51.48} & 2.29 \cellcolor{green!18.00} & 3.34 \cellcolor{orange!26.36} & 762.58 \cellcolor{purple!52.35}\\ \hline 
\end{tabular}}
\vspace{-0.300000cm}
\hspace{1cm} 
\subfloat[DEC=-50, robust-2 weighting ]{\begin{tabular}{|lccccc||ccccc||ccccc|} 
 \\ \cline{2-16} \multicolumn{1}{c}{ } & \multicolumn{5}{|c}{650MHz}  & \multicolumn{5}{c}{800MHz}  & \multicolumn{5}{c|}{1100MHz} \\ \cline{1-16} 
 resbin  &1 & 2 & 3 & 4 & 5 & 1 & 2 & 3 & 4 & 5 & 1 & 2 & 3 & 4 & 5 \\ \hline
SKA1REF2 & 0.71 \cellcolor{blue!19.71} & 1.21 \cellcolor{red!60.00} & 2.24 \cellcolor{green!60.00} & 3.26 \cellcolor{orange!60.00} & 743.31 \cellcolor{purple!60.00} & 0.60 \cellcolor{blue!18.00} & 1.19 \cellcolor{red!60.00} & 2.24 \cellcolor{green!60.00} & 3.25 \cellcolor{orange!36.07} & 788.86 \cellcolor{purple!60.00} & 0.51 \cellcolor{blue!25.52} & 1.17 \cellcolor{red!60.00} & 2.24 \cellcolor{green!60.00} & 3.25 \cellcolor{orange!32.46} & 762.04 \cellcolor{purple!59.29}\\ \hline 
SKA1W9-12A54B90 & 0.94 \cellcolor{blue!60.00} & 1.19 \cellcolor{red!51.73} & 2.23 \cellcolor{green!18.00} & 3.25 \cellcolor{orange!51.03} & 735.05 \cellcolor{purple!18.00} & 0.77 \cellcolor{blue!60.00} & 1.15 \cellcolor{red!25.70} & 2.23 \cellcolor{green!37.87} & 3.25 \cellcolor{orange!55.12} & 786.91 \cellcolor{purple!30.51} & 0.58 \cellcolor{blue!60.00} & 1.14 \cellcolor{red!18.00} & 2.23 \cellcolor{green!36.05} & 3.25 \cellcolor{orange!30.85} & 760.90 \cellcolor{purple!51.16}\\ \hline 
SKA1W9-12A60B100 & 0.87 \cellcolor{blue!46.73} & 1.16 \cellcolor{red!32.53} & 2.23 \cellcolor{green!25.07} & 3.25 \cellcolor{orange!43.96} & 738.92 \cellcolor{purple!37.67} & 0.71 \cellcolor{blue!45.91} & 1.15 \cellcolor{red!21.43} & 2.23 \cellcolor{green!49.61} & 3.25 \cellcolor{orange!60.00} & 787.29 \cellcolor{purple!36.28} & 0.56 \cellcolor{blue!48.33} & 1.15 \cellcolor{red!29.39} & 2.24 \cellcolor{green!54.84} & 3.25 \cellcolor{orange!26.49} & 760.87 \cellcolor{purple!50.98}\\ \hline 
SKA1W9-12A72B120 & 0.76 \cellcolor{blue!28.09} & 1.15 \cellcolor{red!20.87} & 2.24 \cellcolor{green!59.56} & 3.25 \cellcolor{orange!49.38} & 741.45 \cellcolor{purple!50.58} & 0.63 \cellcolor{blue!26.01} & 1.14 \cellcolor{red!18.00} & 2.23 \cellcolor{green!24.77} & 3.24 \cellcolor{orange!27.28} & 786.27 \cellcolor{purple!20.84} & 0.52 \cellcolor{blue!29.78} & 1.16 \cellcolor{red!42.80} & 2.23 \cellcolor{green!18.00} & 3.25 \cellcolor{orange!18.00} & 756.20 \cellcolor{purple!18.00}\\ \hline 
SKA1W9-12A80B133 & 0.70 \cellcolor{blue!18.00} & 1.14 \cellcolor{red!18.00} & 2.23 \cellcolor{green!40.55} & 3.24 \cellcolor{orange!18.00} & 737.83 \cellcolor{purple!32.15} & 0.60 \cellcolor{blue!18.56} & 1.15 \cellcolor{red!26.99} & 2.23 \cellcolor{green!18.00} & 3.24 \cellcolor{orange!18.00} & 786.08 \cellcolor{purple!18.00} & 0.50 \cellcolor{blue!18.00} & 1.16 \cellcolor{red!46.59} & 2.23 \cellcolor{green!35.32} & 3.26 \cellcolor{orange!60.00} & 762.15 \cellcolor{purple!60.00}\\ \hline 
\end{tabular}}
\vspace{-0.300000cm}
\hspace{1cm} 
\subfloat[DEC=-50, robust-2 weighting with a 1 arcsec Gaussian taper]{\begin{tabular}{|lccccc||ccccc||ccccc|} 
 \\ \cline{2-16} \multicolumn{1}{c}{ } & \multicolumn{5}{|c}{650MHz}  & \multicolumn{5}{c}{800MHz}  & \multicolumn{5}{c|}{1100MHz} \\ \cline{1-16} 
 resbin  &1 & 2 & 3 & 4 & 5 & 1 & 2 & 3 & 4 & 5 & 1 & 2 & 3 & 4 & 5 \\ \hline
SKA1REF2 & 1.02 \cellcolor{blue!32.44} & 1.31 \cellcolor{red!60.00} & 2.26 \cellcolor{green!60.00} & 3.27 \cellcolor{orange!60.00} & 743.31 \cellcolor{purple!60.00} & 0.97 \cellcolor{blue!45.76} & 1.29 \cellcolor{red!60.00} & 2.26 \cellcolor{green!60.00} & 3.26 \cellcolor{orange!37.09} & 788.86 \cellcolor{purple!60.00} & 0.92 \cellcolor{blue!60.00} & 1.27 \cellcolor{red!60.00} & 2.26 \cellcolor{green!60.00} & 3.26 \cellcolor{orange!32.39} & 762.04 \cellcolor{purple!59.29}\\ \hline 
SKA1W9-12A54B90 & 1.14 \cellcolor{blue!60.00} & 1.28 \cellcolor{red!44.26} & 2.25 \cellcolor{green!18.00} & 3.26 \cellcolor{orange!51.37} & 735.05 \cellcolor{purple!18.00} & 1.01 \cellcolor{blue!60.00} & 1.25 \cellcolor{red!23.29} & 2.26 \cellcolor{green!40.29} & 3.26 \cellcolor{orange!55.70} & 786.91 \cellcolor{purple!30.51} & 0.90 \cellcolor{blue!43.92} & 1.24 \cellcolor{red!18.00} & 2.25 \cellcolor{green!37.16} & 3.26 \cellcolor{orange!30.99} & 760.90 \cellcolor{purple!51.16}\\ \hline 
SKA1W9-12A60B100 & 1.08 \cellcolor{blue!45.13} & 1.26 \cellcolor{red!27.99} & 2.25 \cellcolor{green!24.68} & 3.26 \cellcolor{orange!44.60} & 738.92 \cellcolor{purple!37.66} & 0.97 \cellcolor{blue!45.35} & 1.25 \cellcolor{red!20.11} & 2.26 \cellcolor{green!50.57} & 3.26 \cellcolor{orange!60.00} & 787.29 \cellcolor{purple!36.28} & 0.89 \cellcolor{blue!32.66} & 1.25 \cellcolor{red!28.80} & 2.26 \cellcolor{green!53.74} & 3.26 \cellcolor{orange!26.35} & 760.87 \cellcolor{purple!50.99}\\ \hline 
SKA1W9-12A72B120 & 1.00 \cellcolor{blue!27.05} & 1.25 \cellcolor{red!19.24} & 2.26 \cellcolor{green!59.52} & 3.26 \cellcolor{orange!49.50} & 741.45 \cellcolor{purple!50.58} & 0.92 \cellcolor{blue!26.50} & 1.24 \cellcolor{red!18.00} & 2.25 \cellcolor{green!26.57} & 3.25 \cellcolor{orange!28.02} & 786.27 \cellcolor{purple!20.84} & 0.87 \cellcolor{blue!20.99} & 1.26 \cellcolor{red!40.01} & 2.25 \cellcolor{green!18.00} & 3.25 \cellcolor{orange!18.00} & 756.20 \cellcolor{purple!18.00}\\ \hline 
SKA1W9-12A80B133 & 0.97 \cellcolor{blue!18.00} & 1.24 \cellcolor{red!18.00} & 2.26 \cellcolor{green!39.95} & 3.25 \cellcolor{orange!18.00} & 737.83 \cellcolor{purple!32.15} & 0.90 \cellcolor{blue!18.00} & 1.25 \cellcolor{red!26.75} & 2.25 \cellcolor{green!18.00} & 3.25 \cellcolor{orange!18.00} & 786.08 \cellcolor{purple!18.00} & 0.87 \cellcolor{blue!18.00} & 1.26 \cellcolor{red!42.98} & 2.25 \cellcolor{green!34.95} & 3.27 \cellcolor{orange!60.00} & 762.15 \cellcolor{purple!60.00}\\ \hline 
\end{tabular}}
\vspace{-0.300000cm}
\hspace{1cm} 

\vspace{.50cm}
\caption{FWHM PSF sizes (in arcseconds) for the different layouts at different scales. These values are generated for angular scales \{0.4-1, 1-2, 2-3, 3-4, 600-3600\} arcsec labeled as {\it resbin} \{1, 2, 3, 4, 5\} respectively, for natural, robust-2 weighting and robust-2 weighting with a 1 arcsec Gaussian taper. For each column, the intensity of the color increases with the value.}\label{tab:psf_mean}}
 \end{table}

% Auto generated table
 \vspace{-4cm}\begin{table}[H]
 \tiny{\subfloat[DEC=-10, natural weighting]{\begin{tabular}{|lccccc||ccccc||ccccc|} 
 \\ \cline{2-16} \multicolumn{1}{c}{ } & \multicolumn{5}{|c}{650MHz}  & \multicolumn{5}{c}{800MHz}  & \multicolumn{5}{c|}{1100MHz} \\ \cline{1-16} 
 resbin  &1 & 2 & 3 & 4 & 5 & 1 & 2 & 3 & 4 & 5 & 1 & 2 & 3 & 4 & 5 \\ \hline
SKA1REF2 & 0.05 \cellcolor{blue!39.00} & 0.02 \cellcolor{red!34.80} & 0.04 \cellcolor{green!39.00} & 0.07 \cellcolor{orange!60.00} & 0.00 \cellcolor{purple!18.00} & 0.06 \cellcolor{blue!60.00} & 0.06 \cellcolor{red!39.00} & 0.02 \cellcolor{green!60.00} & 0.06 \cellcolor{orange!18.00} & 0.03 \cellcolor{purple!60.00} & 0.06 \cellcolor{blue!60.00} & 0.02 \cellcolor{red!46.00} & 0.07 \cellcolor{green!46.00} & 0.03 \cellcolor{orange!60.00} & 0.05 \cellcolor{purple!60.00}\\ \hline 
SKA1W9-12A54B90 & 0.04 \cellcolor{blue!18.00} & 0.05 \cellcolor{red!60.00} & 0.05 \cellcolor{green!60.00} & 0.05 \cellcolor{orange!32.00} & 0.01 \cellcolor{purple!60.00} & 0.06 \cellcolor{blue!60.00} & 0.05 \cellcolor{red!18.00} & 0.01 \cellcolor{green!39.00} & 0.06 \cellcolor{orange!18.00} & 0.02 \cellcolor{purple!18.00} & 0.06 \cellcolor{blue!60.00} & 0.03 \cellcolor{red!60.00} & 0.05 \cellcolor{green!18.00} & 0.02 \cellcolor{orange!39.00} & 0.02 \cellcolor{purple!28.50}\\ \hline 
SKA1W9-12A60B100 & 0.05 \cellcolor{blue!39.00} & 0.05 \cellcolor{red!60.00} & 0.03 \cellcolor{green!18.00} & 0.04 \cellcolor{orange!18.00} & 0.01 \cellcolor{purple!60.00} & 0.05 \cellcolor{blue!18.00} & 0.07 \cellcolor{red!60.00} & 0.00 \cellcolor{green!18.00} & 0.07 \cellcolor{orange!60.00} & 0.03 \cellcolor{purple!60.00} & 0.06 \cellcolor{blue!60.00} & 0.03 \cellcolor{red!60.00} & 0.08 \cellcolor{green!60.00} & 0.01 \cellcolor{orange!18.00} & 0.02 \cellcolor{purple!28.50}\\ \hline 
SKA1W9-12A72B120 & 0.05 \cellcolor{blue!39.00} & 0.03 \cellcolor{red!43.20} & 0.05 \cellcolor{green!60.00} & 0.06 \cellcolor{orange!46.00} & 0.00 \cellcolor{purple!18.00} & 0.06 \cellcolor{blue!60.00} & 0.06 \cellcolor{red!39.00} & 0.01 \cellcolor{green!39.00} & 0.06 \cellcolor{orange!18.00} & 0.03 \cellcolor{purple!60.00} & 0.05 \cellcolor{blue!18.00} & 0.00 \cellcolor{red!18.00} & 0.06 \cellcolor{green!32.00} & 0.01 \cellcolor{orange!18.00} & 0.01 \cellcolor{purple!18.00}\\ \hline 
SKA1W9-12A80B133 & 0.06 \cellcolor{blue!60.00} & 0.00 \cellcolor{red!18.00} & 0.04 \cellcolor{green!39.00} & 0.04 \cellcolor{orange!18.00} & 0.01 \cellcolor{purple!60.00} & 0.05 \cellcolor{blue!18.00} & 0.06 \cellcolor{red!39.00} & 0.01 \cellcolor{green!39.00} & 0.06 \cellcolor{orange!18.00} & 0.02 \cellcolor{purple!18.00} & 0.06 \cellcolor{blue!60.00} & 0.01 \cellcolor{red!32.00} & 0.07 \cellcolor{green!46.00} & 0.02 \cellcolor{orange!39.00} & 0.03 \cellcolor{purple!39.00}\\ \hline 
\end{tabular}}
\vspace{-0.300000cm}
\hspace{1cm} 
\subfloat[DEC=-10, robust-2 weighting ]{\begin{tabular}{|lccccc||ccccc||ccccc|} 
 \\ \cline{2-16} \multicolumn{1}{c}{ } & \multicolumn{5}{|c}{650MHz}  & \multicolumn{5}{c}{800MHz}  & \multicolumn{5}{c|}{1100MHz} \\ \cline{1-16} 
 resbin  &1 & 2 & 3 & 4 & 5 & 1 & 2 & 3 & 4 & 5 & 1 & 2 & 3 & 4 & 5 \\ \hline
SKA1REF2 & 0.07 \cellcolor{blue!60.00} & 0.01 \cellcolor{red!18.00} & 0.00 \cellcolor{green!18.00} & 0.01 \cellcolor{orange!60.00} & 0.01 \cellcolor{purple!60.00} & 0.07 \cellcolor{blue!60.00} & 0.02 \cellcolor{red!60.00} & 0.01 \cellcolor{green!18.00} & 0.00 \cellcolor{orange!18.00} & 0.03 \cellcolor{purple!60.00} & 0.06 \cellcolor{blue!46.00} & 0.00 \cellcolor{red!18.00} & 0.00 \cellcolor{green!18.00} & 0.00 \cellcolor{orange!18.00} & 0.05 \cellcolor{purple!60.00}\\ \hline 
SKA1W9-12A54B90 & 0.07 \cellcolor{blue!60.00} & 0.04 \cellcolor{red!60.00} & 0.01 \cellcolor{green!60.00} & 0.00 \cellcolor{orange!18.00} & 0.00 \cellcolor{purple!18.00} & 0.07 \cellcolor{blue!60.00} & 0.02 \cellcolor{red!60.00} & 0.01 \cellcolor{green!18.00} & 0.00 \cellcolor{orange!18.00} & 0.02 \cellcolor{purple!18.00} & 0.07 \cellcolor{blue!60.00} & 0.00 \cellcolor{red!18.00} & 0.00 \cellcolor{green!18.00} & 0.00 \cellcolor{orange!18.00} & 0.02 \cellcolor{purple!28.50}\\ \hline 
SKA1W9-12A60B100 & 0.07 \cellcolor{blue!60.00} & 0.04 \cellcolor{red!60.00} & 0.01 \cellcolor{green!60.00} & 0.00 \cellcolor{orange!18.00} & 0.01 \cellcolor{purple!60.00} & 0.07 \cellcolor{blue!60.00} & 0.01 \cellcolor{red!39.00} & 0.01 \cellcolor{green!18.00} & 0.00 \cellcolor{orange!18.00} & 0.03 \cellcolor{purple!60.00} & 0.06 \cellcolor{blue!46.00} & 0.00 \cellcolor{red!18.00} & 0.00 \cellcolor{green!18.00} & 0.00 \cellcolor{orange!18.00} & 0.02 \cellcolor{purple!28.50}\\ \hline 
SKA1W9-12A72B120 & 0.06 \cellcolor{blue!18.00} & 0.02 \cellcolor{red!32.00} & 0.00 \cellcolor{green!18.00} & 0.00 \cellcolor{orange!18.00} & 0.01 \cellcolor{purple!60.00} & 0.07 \cellcolor{blue!60.00} & 0.00 \cellcolor{red!18.00} & 0.01 \cellcolor{green!18.00} & 0.00 \cellcolor{orange!18.00} & 0.03 \cellcolor{purple!60.00} & 0.04 \cellcolor{blue!18.00} & 0.01 \cellcolor{red!60.00} & 0.00 \cellcolor{green!18.00} & 0.00 \cellcolor{orange!18.00} & 0.01 \cellcolor{purple!18.00}\\ \hline 
SKA1W9-12A80B133 & 0.06 \cellcolor{blue!18.00} & 0.01 \cellcolor{red!18.00} & 0.00 \cellcolor{green!18.00} & 0.00 \cellcolor{orange!18.00} & 0.00 \cellcolor{purple!18.00} & 0.06 \cellcolor{blue!18.00} & 0.00 \cellcolor{red!18.00} & 0.01 \cellcolor{green!18.00} & 0.01 \cellcolor{orange!60.00} & 0.02 \cellcolor{purple!18.00} & 0.04 \cellcolor{blue!18.00} & 0.00 \cellcolor{red!18.00} & 0.01 \cellcolor{green!60.00} & 0.00 \cellcolor{orange!18.00} & 0.03 \cellcolor{purple!39.00}\\ \hline 
\end{tabular}}
\vspace{-0.300000cm}
\hspace{1cm} 
\subfloat[DEC=-10, robust-2 weighting with a 1 arcsec Gaussian taper]{\begin{tabular}{|lccccc||ccccc||ccccc|} 
 \\ \cline{2-16} \multicolumn{1}{c}{ } & \multicolumn{5}{|c}{650MHz}  & \multicolumn{5}{c}{800MHz}  & \multicolumn{5}{c|}{1100MHz} \\ \cline{1-16} 
 resbin  &1 & 2 & 3 & 4 & 5 & 1 & 2 & 3 & 4 & 5 & 1 & 2 & 3 & 4 & 5 \\ \hline
SKA1REF2 & 0.05 \cellcolor{blue!60.00} & 0.02 \cellcolor{red!46.00} & 0.00 \cellcolor{green!18.00} & 0.01 \cellcolor{orange!60.00} & 0.01 \cellcolor{purple!60.00} & 0.04 \cellcolor{blue!60.00} & 0.02 \cellcolor{red!60.00} & 0.01 \cellcolor{green!18.00} & 0.00 \cellcolor{orange!18.00} & 0.03 \cellcolor{purple!60.00} & 0.03 \cellcolor{blue!60.00} & 0.00 \cellcolor{red!18.00} & 0.00 \cellcolor{green!18.00} & 0.00 \cellcolor{orange!18.00} & 0.05 \cellcolor{purple!60.00}\\ \hline 
SKA1W9-12A54B90 & 0.05 \cellcolor{blue!60.00} & 0.03 \cellcolor{red!60.00} & 0.01 \cellcolor{green!60.00} & 0.00 \cellcolor{orange!18.00} & 0.00 \cellcolor{purple!18.00} & 0.04 \cellcolor{blue!60.00} & 0.01 \cellcolor{red!39.00} & 0.01 \cellcolor{green!18.00} & 0.00 \cellcolor{orange!18.00} & 0.02 \cellcolor{purple!18.00} & 0.02 \cellcolor{blue!39.00} & 0.00 \cellcolor{red!18.00} & 0.00 \cellcolor{green!18.00} & 0.00 \cellcolor{orange!18.00} & 0.02 \cellcolor{purple!28.50}\\ \hline 
SKA1W9-12A60B100 & 0.05 \cellcolor{blue!60.00} & 0.03 \cellcolor{red!60.00} & 0.01 \cellcolor{green!60.00} & 0.00 \cellcolor{orange!18.00} & 0.01 \cellcolor{purple!60.00} & 0.04 \cellcolor{blue!60.00} & 0.01 \cellcolor{red!39.00} & 0.01 \cellcolor{green!18.00} & 0.00 \cellcolor{orange!18.00} & 0.03 \cellcolor{purple!60.00} & 0.02 \cellcolor{blue!39.00} & 0.00 \cellcolor{red!18.00} & 0.00 \cellcolor{green!18.00} & 0.00 \cellcolor{orange!18.00} & 0.02 \cellcolor{purple!28.50}\\ \hline 
SKA1W9-12A72B120 & 0.04 \cellcolor{blue!39.00} & 0.01 \cellcolor{red!32.00} & 0.00 \cellcolor{green!18.00} & 0.00 \cellcolor{orange!18.00} & 0.01 \cellcolor{purple!60.00} & 0.03 \cellcolor{blue!39.00} & 0.00 \cellcolor{red!18.00} & 0.01 \cellcolor{green!18.00} & 0.00 \cellcolor{orange!18.00} & 0.03 \cellcolor{purple!60.00} & 0.01 \cellcolor{blue!18.00} & 0.01 \cellcolor{red!60.00} & 0.00 \cellcolor{green!18.00} & 0.00 \cellcolor{orange!18.00} & 0.01 \cellcolor{purple!18.00}\\ \hline 
SKA1W9-12A80B133 & 0.03 \cellcolor{blue!18.00} & 0.00 \cellcolor{red!18.00} & 0.00 \cellcolor{green!18.00} & 0.00 \cellcolor{orange!18.00} & 0.00 \cellcolor{purple!18.00} & 0.02 \cellcolor{blue!18.00} & 0.00 \cellcolor{red!18.00} & 0.01 \cellcolor{green!18.00} & 0.01 \cellcolor{orange!60.00} & 0.02 \cellcolor{purple!18.00} & 0.01 \cellcolor{blue!18.00} & 0.00 \cellcolor{red!18.00} & 0.00 \cellcolor{green!18.00} & 0.00 \cellcolor{orange!18.00} & 0.03 \cellcolor{purple!39.00}\\ \hline 
\end{tabular}}
\vspace{-0.300000cm}
\hspace{1cm} 
\subfloat[DEC=-30, natural weighting]{\begin{tabular}{|lccccc||ccccc||ccccc|} 
 \\ \cline{2-16} \multicolumn{1}{c}{ } & \multicolumn{5}{|c}{650MHz}  & \multicolumn{5}{c}{800MHz}  & \multicolumn{5}{c|}{1100MHz} \\ \cline{1-16} 
 resbin  &1 & 2 & 3 & 4 & 5 & 1 & 2 & 3 & 4 & 5 & 1 & 2 & 3 & 4 & 5 \\ \hline
SKA1REF2 & 0.13 \cellcolor{blue!18.00} & 0.07 \cellcolor{red!49.50} & 0.08 \cellcolor{green!60.00} & 0.06 \cellcolor{orange!18.00} & 0.02 \cellcolor{purple!18.00} & 0.09 \cellcolor{blue!18.00} & 0.08 \cellcolor{red!60.00} & 0.06 \cellcolor{green!18.00} & 0.08 \cellcolor{orange!60.00} & 0.03 \cellcolor{purple!60.00} & 0.06 \cellcolor{blue!18.00} & 0.08 \cellcolor{red!60.00} & 0.06 \cellcolor{green!39.00} & 0.05 \cellcolor{orange!39.00} & 0.07 \cellcolor{purple!60.00}\\ \hline 
SKA1W9-12A54B90 & 0.20 \cellcolor{blue!60.00} & 0.09 \cellcolor{red!60.00} & 0.03 \cellcolor{green!18.00} & 0.07 \cellcolor{orange!39.00} & 0.03 \cellcolor{purple!60.00} & 0.17 \cellcolor{blue!60.00} & 0.03 \cellcolor{red!33.75} & 0.08 \cellcolor{green!60.00} & 0.06 \cellcolor{orange!48.00} & 0.01 \cellcolor{purple!18.00} & 0.12 \cellcolor{blue!60.00} & 0.02 \cellcolor{red!18.00} & 0.07 \cellcolor{green!60.00} & 0.06 \cellcolor{orange!60.00} & 0.03 \cellcolor{purple!18.00}\\ \hline 
SKA1W9-12A60B100 & 0.19 \cellcolor{blue!54.00} & 0.06 \cellcolor{red!44.25} & 0.06 \cellcolor{green!43.20} & 0.08 \cellcolor{orange!60.00} & 0.03 \cellcolor{purple!60.00} & 0.15 \cellcolor{blue!49.50} & 0.00 \cellcolor{red!18.00} & 0.07 \cellcolor{green!39.00} & 0.07 \cellcolor{orange!54.00} & 0.03 \cellcolor{purple!60.00} & 0.12 \cellcolor{blue!60.00} & 0.05 \cellcolor{red!39.00} & 0.07 \cellcolor{green!60.00} & 0.04 \cellcolor{orange!18.00} & 0.04 \cellcolor{purple!28.50}\\ \hline 
SKA1W9-12A72B120 & 0.16 \cellcolor{blue!36.00} & 0.01 \cellcolor{red!18.00} & 0.05 \cellcolor{green!34.80} & 0.06 \cellcolor{orange!18.00} & 0.02 \cellcolor{purple!18.00} & 0.13 \cellcolor{blue!39.00} & 0.00 \cellcolor{red!18.00} & 0.07 \cellcolor{green!39.00} & 0.06 \cellcolor{orange!48.00} & 0.03 \cellcolor{purple!60.00} & 0.11 \cellcolor{blue!53.00} & 0.06 \cellcolor{red!46.00} & 0.05 \cellcolor{green!18.00} & 0.04 \cellcolor{orange!18.00} & 0.04 \cellcolor{purple!28.50}\\ \hline 
SKA1W9-12A80B133 & 0.14 \cellcolor{blue!24.00} & 0.02 \cellcolor{red!23.25} & 0.06 \cellcolor{green!43.20} & 0.08 \cellcolor{orange!60.00} & 0.03 \cellcolor{purple!60.00} & 0.12 \cellcolor{blue!33.75} & 0.03 \cellcolor{red!33.75} & 0.07 \cellcolor{green!39.00} & 0.01 \cellcolor{orange!18.00} & 0.02 \cellcolor{purple!39.00} & 0.08 \cellcolor{blue!32.00} & 0.07 \cellcolor{red!53.00} & 0.06 \cellcolor{green!39.00} & 0.04 \cellcolor{orange!18.00} & 0.04 \cellcolor{purple!28.50}\\ \hline 
\end{tabular}}
\vspace{-0.300000cm}
\hspace{1cm} 
\subfloat[DEC=-30, robust-2 weighting ]{\begin{tabular}{|lccccc||ccccc||ccccc|} 
 \\ \cline{2-16} \multicolumn{1}{c}{ } & \multicolumn{5}{|c}{650MHz}  & \multicolumn{5}{c}{800MHz}  & \multicolumn{5}{c|}{1100MHz} \\ \cline{1-16} 
 resbin  &1 & 2 & 3 & 4 & 5 & 1 & 2 & 3 & 4 & 5 & 1 & 2 & 3 & 4 & 5 \\ \hline
SKA1REF2 & 0.12 \cellcolor{blue!60.00} & 0.04 \cellcolor{red!60.00} & 0.02 \cellcolor{green!60.00} & 0.01 \cellcolor{orange!60.00} & 0.05 \cellcolor{purple!60.00} & 0.11 \cellcolor{blue!60.00} & 0.03 \cellcolor{red!60.00} & 0.00 \cellcolor{green!18.00} & 0.01 \cellcolor{orange!60.00} & 0.03 \cellcolor{purple!60.00} & 0.07 \cellcolor{blue!39.00} & 0.02 \cellcolor{red!60.00} & 0.00 \cellcolor{green!18.00} & 0.01 \cellcolor{orange!60.00} & 0.07 \cellcolor{purple!60.00}\\ \hline 
SKA1W9-12A54B90 & 0.10 \cellcolor{blue!18.00} & 0.03 \cellcolor{red!46.00} & 0.00 \cellcolor{green!18.00} & 0.00 \cellcolor{orange!18.00} & 0.02 \cellcolor{purple!18.00} & 0.10 \cellcolor{blue!39.00} & 0.02 \cellcolor{red!46.00} & 0.00 \cellcolor{green!18.00} & 0.01 \cellcolor{orange!60.00} & 0.01 \cellcolor{purple!18.00} & 0.08 \cellcolor{blue!60.00} & 0.01 \cellcolor{red!39.00} & 0.01 \cellcolor{green!60.00} & 0.00 \cellcolor{orange!18.00} & 0.04 \cellcolor{purple!18.00}\\ \hline 
SKA1W9-12A60B100 & 0.10 \cellcolor{blue!18.00} & 0.03 \cellcolor{red!46.00} & 0.01 \cellcolor{green!39.00} & 0.01 \cellcolor{orange!60.00} & 0.04 \cellcolor{purple!46.00} & 0.10 \cellcolor{blue!39.00} & 0.02 \cellcolor{red!46.00} & 0.00 \cellcolor{green!18.00} & 0.00 \cellcolor{orange!18.00} & 0.03 \cellcolor{purple!60.00} & 0.07 \cellcolor{blue!39.00} & 0.01 \cellcolor{red!39.00} & 0.01 \cellcolor{green!60.00} & 0.01 \cellcolor{orange!60.00} & 0.04 \cellcolor{purple!18.00}\\ \hline 
SKA1W9-12A72B120 & 0.10 \cellcolor{blue!18.00} & 0.01 \cellcolor{red!18.00} & 0.01 \cellcolor{green!39.00} & 0.00 \cellcolor{orange!18.00} & 0.03 \cellcolor{purple!32.00} & 0.10 \cellcolor{blue!39.00} & 0.01 \cellcolor{red!32.00} & 0.01 \cellcolor{green!60.00} & 0.01 \cellcolor{orange!60.00} & 0.03 \cellcolor{purple!60.00} & 0.06 \cellcolor{blue!18.00} & 0.00 \cellcolor{red!18.00} & 0.01 \cellcolor{green!60.00} & 0.00 \cellcolor{orange!18.00} & 0.04 \cellcolor{purple!18.00}\\ \hline 
SKA1W9-12A80B133 & 0.11 \cellcolor{blue!39.00} & 0.01 \cellcolor{red!18.00} & 0.01 \cellcolor{green!39.00} & 0.00 \cellcolor{orange!18.00} & 0.04 \cellcolor{purple!46.00} & 0.09 \cellcolor{blue!18.00} & 0.00 \cellcolor{red!18.00} & 0.00 \cellcolor{green!18.00} & 0.00 \cellcolor{orange!18.00} & 0.02 \cellcolor{purple!39.00} & 0.06 \cellcolor{blue!18.00} & 0.01 \cellcolor{red!39.00} & 0.01 \cellcolor{green!60.00} & 0.00 \cellcolor{orange!18.00} & 0.05 \cellcolor{purple!32.00}\\ \hline 
\end{tabular}}
\vspace{-0.300000cm}
\hspace{1cm} 
\subfloat[DEC=-30, robust-2 weighting with a 1 arcsec Gaussian taper]{\begin{tabular}{|lccccc||ccccc||ccccc|} 
 \\ \cline{2-16} \multicolumn{1}{c}{ } & \multicolumn{5}{|c}{650MHz}  & \multicolumn{5}{c}{800MHz}  & \multicolumn{5}{c|}{1100MHz} \\ \cline{1-16} 
 resbin  &1 & 2 & 3 & 4 & 5 & 1 & 2 & 3 & 4 & 5 & 1 & 2 & 3 & 4 & 5 \\ \hline
SKA1REF2 & 0.07 \cellcolor{blue!60.00} & 0.04 \cellcolor{red!60.00} & 0.02 \cellcolor{green!60.00} & 0.01 \cellcolor{orange!60.00} & 0.05 \cellcolor{purple!60.00} & 0.05 \cellcolor{blue!60.00} & 0.03 \cellcolor{red!60.00} & 0.00 \cellcolor{green!18.00} & 0.01 \cellcolor{orange!60.00} & 0.03 \cellcolor{purple!60.00} & 0.03 \cellcolor{blue!60.00} & 0.02 \cellcolor{red!60.00} & 0.00 \cellcolor{green!18.00} & 0.01 \cellcolor{orange!60.00} & 0.07 \cellcolor{purple!60.00}\\ \hline 
SKA1W9-12A54B90 & 0.06 \cellcolor{blue!39.00} & 0.02 \cellcolor{red!32.00} & 0.00 \cellcolor{green!18.00} & 0.00 \cellcolor{orange!18.00} & 0.02 \cellcolor{purple!18.00} & 0.04 \cellcolor{blue!39.00} & 0.01 \cellcolor{red!32.00} & 0.00 \cellcolor{green!18.00} & 0.01 \cellcolor{orange!60.00} & 0.01 \cellcolor{purple!18.00} & 0.02 \cellcolor{blue!39.00} & 0.01 \cellcolor{red!39.00} & 0.01 \cellcolor{green!60.00} & 0.00 \cellcolor{orange!18.00} & 0.04 \cellcolor{purple!18.00}\\ \hline 
SKA1W9-12A60B100 & 0.05 \cellcolor{blue!18.00} & 0.02 \cellcolor{red!32.00} & 0.01 \cellcolor{green!39.00} & 0.01 \cellcolor{orange!60.00} & 0.04 \cellcolor{purple!46.00} & 0.04 \cellcolor{blue!39.00} & 0.01 \cellcolor{red!32.00} & 0.00 \cellcolor{green!18.00} & 0.00 \cellcolor{orange!18.00} & 0.03 \cellcolor{purple!60.00} & 0.02 \cellcolor{blue!39.00} & 0.00 \cellcolor{red!18.00} & 0.01 \cellcolor{green!60.00} & 0.01 \cellcolor{orange!60.00} & 0.04 \cellcolor{purple!18.00}\\ \hline 
SKA1W9-12A72B120 & 0.05 \cellcolor{blue!18.00} & 0.01 \cellcolor{red!18.00} & 0.01 \cellcolor{green!39.00} & 0.00 \cellcolor{orange!18.00} & 0.03 \cellcolor{purple!32.00} & 0.03 \cellcolor{blue!18.00} & 0.01 \cellcolor{red!32.00} & 0.01 \cellcolor{green!60.00} & 0.01 \cellcolor{orange!60.00} & 0.03 \cellcolor{purple!60.00} & 0.01 \cellcolor{blue!18.00} & 0.00 \cellcolor{red!18.00} & 0.01 \cellcolor{green!60.00} & 0.00 \cellcolor{orange!18.00} & 0.04 \cellcolor{purple!18.00}\\ \hline 
SKA1W9-12A80B133 & 0.05 \cellcolor{blue!18.00} & 0.01 \cellcolor{red!18.00} & 0.01 \cellcolor{green!39.00} & 0.00 \cellcolor{orange!18.00} & 0.04 \cellcolor{purple!46.00} & 0.03 \cellcolor{blue!18.00} & 0.00 \cellcolor{red!18.00} & 0.00 \cellcolor{green!18.00} & 0.00 \cellcolor{orange!18.00} & 0.02 \cellcolor{purple!39.00} & 0.01 \cellcolor{blue!18.00} & 0.01 \cellcolor{red!39.00} & 0.01 \cellcolor{green!60.00} & 0.00 \cellcolor{orange!18.00} & 0.05 \cellcolor{purple!32.00}\\ \hline 
\end{tabular}}
\vspace{-0.300000cm}
\hspace{1cm} 
\subfloat[DEC=-50, natural weighting]{\begin{tabular}{|lccccc||ccccc||ccccc|} 
 \\ \cline{2-16} \multicolumn{1}{c}{ } & \multicolumn{5}{|c}{650MHz}  & \multicolumn{5}{c}{800MHz}  & \multicolumn{5}{c|}{1100MHz} \\ \cline{1-16} 
 resbin  &1 & 2 & 3 & 4 & 5 & 1 & 2 & 3 & 4 & 5 & 1 & 2 & 3 & 4 & 5 \\ \hline
SKA1REF2 & 0.11 \cellcolor{blue!18.00} & 0.04 \cellcolor{red!36.00} & 0.06 \cellcolor{green!60.00} & 0.02 \cellcolor{orange!18.00} & 0.02 \cellcolor{purple!60.00} & 0.10 \cellcolor{blue!18.00} & 0.07 \cellcolor{red!60.00} & 0.07 \cellcolor{green!60.00} & 0.08 \cellcolor{orange!60.00} & 0.03 \cellcolor{purple!18.00} & 0.06 \cellcolor{blue!18.00} & 0.05 \cellcolor{red!60.00} & 0.06 \cellcolor{green!39.00} & 0.03 \cellcolor{orange!60.00} & 0.08 \cellcolor{purple!60.00}\\ \hline 
SKA1W9-12A54B90 & 0.16 \cellcolor{blue!60.00} & 0.08 \cellcolor{red!60.00} & 0.04 \cellcolor{green!18.00} & 0.04 \cellcolor{orange!60.00} & 0.01 \cellcolor{purple!18.00} & 0.13 \cellcolor{blue!60.00} & 0.04 \cellcolor{red!42.00} & 0.05 \cellcolor{green!18.00} & 0.07 \cellcolor{orange!51.60} & 0.04 \cellcolor{purple!32.00} & 0.10 \cellcolor{blue!60.00} & 0.03 \cellcolor{red!32.00} & 0.07 \cellcolor{green!60.00} & 0.02 \cellcolor{orange!18.00} & 0.06 \cellcolor{purple!43.20}\\ \hline 
SKA1W9-12A60B100 & 0.16 \cellcolor{blue!60.00} & 0.04 \cellcolor{red!36.00} & 0.05 \cellcolor{green!39.00} & 0.04 \cellcolor{orange!60.00} & 0.01 \cellcolor{purple!18.00} & 0.12 \cellcolor{blue!46.00} & 0.02 \cellcolor{red!30.00} & 0.07 \cellcolor{green!60.00} & 0.08 \cellcolor{orange!60.00} & 0.06 \cellcolor{purple!60.00} & 0.10 \cellcolor{blue!60.00} & 0.02 \cellcolor{red!18.00} & 0.07 \cellcolor{green!60.00} & 0.02 \cellcolor{orange!18.00} & 0.06 \cellcolor{purple!43.20}\\ \hline 
SKA1W9-12A72B120 & 0.14 \cellcolor{blue!43.20} & 0.01 \cellcolor{red!18.00} & 0.04 \cellcolor{green!18.00} & 0.03 \cellcolor{orange!39.00} & 0.01 \cellcolor{purple!18.00} & 0.10 \cellcolor{blue!18.00} & 0.00 \cellcolor{red!18.00} & 0.05 \cellcolor{green!18.00} & 0.07 \cellcolor{orange!51.60} & 0.06 \cellcolor{purple!60.00} & 0.09 \cellcolor{blue!49.50} & 0.04 \cellcolor{red!46.00} & 0.06 \cellcolor{green!39.00} & 0.02 \cellcolor{orange!18.00} & 0.03 \cellcolor{purple!18.00}\\ \hline 
SKA1W9-12A80B133 & 0.12 \cellcolor{blue!26.40} & 0.04 \cellcolor{red!36.00} & 0.05 \cellcolor{green!39.00} & 0.04 \cellcolor{orange!60.00} & 0.01 \cellcolor{purple!18.00} & 0.10 \cellcolor{blue!18.00} & 0.03 \cellcolor{red!36.00} & 0.06 \cellcolor{green!39.00} & 0.03 \cellcolor{orange!18.00} & 0.05 \cellcolor{purple!46.00} & 0.07 \cellcolor{blue!28.50} & 0.04 \cellcolor{red!46.00} & 0.05 \cellcolor{green!18.00} & 0.03 \cellcolor{orange!60.00} & 0.07 \cellcolor{purple!51.60}\\ \hline 
\end{tabular}}
\vspace{-0.300000cm}
\hspace{1cm} 
\subfloat[DEC=-50, robust-2 weighting ]{\begin{tabular}{|lccccc||ccccc||ccccc|} 
 \\ \cline{2-16} \multicolumn{1}{c}{ } & \multicolumn{5}{|c}{650MHz}  & \multicolumn{5}{c}{800MHz}  & \multicolumn{5}{c|}{1100MHz} \\ \cline{1-16} 
 resbin  &1 & 2 & 3 & 4 & 5 & 1 & 2 & 3 & 4 & 5 & 1 & 2 & 3 & 4 & 5 \\ \hline
SKA1REF2 & 0.08 \cellcolor{blue!60.00} & 0.03 \cellcolor{red!60.00} & 0.01 \cellcolor{green!60.00} & 0.00 \cellcolor{orange!18.00} & 0.02 \cellcolor{purple!60.00} & 0.08 \cellcolor{blue!60.00} & 0.01 \cellcolor{red!60.00} & 0.00 \cellcolor{green!18.00} & 0.01 \cellcolor{orange!60.00} & 0.03 \cellcolor{purple!18.00} & 0.04 \cellcolor{blue!18.00} & 0.01 \cellcolor{red!60.00} & 0.00 \cellcolor{green!18.00} & 0.00 \cellcolor{orange!18.00} & 0.08 \cellcolor{purple!60.00}\\ \hline 
SKA1W9-12A54B90 & 0.07 \cellcolor{blue!18.00} & 0.01 \cellcolor{red!18.00} & 0.01 \cellcolor{green!60.00} & 0.01 \cellcolor{orange!60.00} & 0.00 \cellcolor{purple!18.00} & 0.07 \cellcolor{blue!39.00} & 0.01 \cellcolor{red!60.00} & 0.00 \cellcolor{green!18.00} & 0.00 \cellcolor{orange!18.00} & 0.04 \cellcolor{purple!32.00} & 0.06 \cellcolor{blue!60.00} & 0.01 \cellcolor{red!60.00} & 0.01 \cellcolor{green!60.00} & 0.01 \cellcolor{orange!60.00} & 0.06 \cellcolor{purple!43.20}\\ \hline 
SKA1W9-12A60B100 & 0.08 \cellcolor{blue!60.00} & 0.02 \cellcolor{red!39.00} & 0.00 \cellcolor{green!18.00} & 0.00 \cellcolor{orange!18.00} & 0.01 \cellcolor{purple!39.00} & 0.08 \cellcolor{blue!60.00} & 0.01 \cellcolor{red!60.00} & 0.00 \cellcolor{green!18.00} & 0.00 \cellcolor{orange!18.00} & 0.06 \cellcolor{purple!60.00} & 0.05 \cellcolor{blue!39.00} & 0.01 \cellcolor{red!60.00} & 0.00 \cellcolor{green!18.00} & 0.00 \cellcolor{orange!18.00} & 0.06 \cellcolor{purple!43.20}\\ \hline 
SKA1W9-12A72B120 & 0.08 \cellcolor{blue!60.00} & 0.02 \cellcolor{red!39.00} & 0.00 \cellcolor{green!18.00} & 0.01 \cellcolor{orange!60.00} & 0.02 \cellcolor{purple!60.00} & 0.07 \cellcolor{blue!39.00} & 0.01 \cellcolor{red!60.00} & 0.00 \cellcolor{green!18.00} & 0.00 \cellcolor{orange!18.00} & 0.06 \cellcolor{purple!60.00} & 0.04 \cellcolor{blue!18.00} & 0.00 \cellcolor{red!18.00} & 0.01 \cellcolor{green!60.00} & 0.01 \cellcolor{orange!60.00} & 0.03 \cellcolor{purple!18.00}\\ \hline 
SKA1W9-12A80B133 & 0.08 \cellcolor{blue!60.00} & 0.01 \cellcolor{red!18.00} & 0.00 \cellcolor{green!18.00} & 0.00 \cellcolor{orange!18.00} & 0.00 \cellcolor{purple!18.00} & 0.06 \cellcolor{blue!18.00} & 0.00 \cellcolor{red!18.00} & 0.00 \cellcolor{green!18.00} & 0.00 \cellcolor{orange!18.00} & 0.05 \cellcolor{purple!46.00} & 0.04 \cellcolor{blue!18.00} & 0.01 \cellcolor{red!60.00} & 0.00 \cellcolor{green!18.00} & 0.00 \cellcolor{orange!18.00} & 0.07 \cellcolor{purple!51.60}\\ \hline 
\end{tabular}}
\vspace{-0.300000cm}
\hspace{1cm} 
\subfloat[DEC=-50, robust-2 weighting with a 1 arcsec Gaussian taper]{\begin{tabular}{|lccccc||ccccc||ccccc|} 
 \\ \cline{2-16} \multicolumn{1}{c}{ } & \multicolumn{5}{|c}{650MHz}  & \multicolumn{5}{c}{800MHz}  & \multicolumn{5}{c|}{1100MHz} \\ \cline{1-16} 
 resbin  &1 & 2 & 3 & 4 & 5 & 1 & 2 & 3 & 4 & 5 & 1 & 2 & 3 & 4 & 5 \\ \hline
SKA1REF2 & 0.04 \cellcolor{blue!60.00} & 0.02 \cellcolor{red!60.00} & 0.01 \cellcolor{green!60.00} & 0.00 \cellcolor{orange!18.00} & 0.02 \cellcolor{purple!60.00} & 0.03 \cellcolor{blue!60.00} & 0.01 \cellcolor{red!60.00} & 0.00 \cellcolor{green!18.00} & 0.01 \cellcolor{orange!60.00} & 0.03 \cellcolor{purple!18.00} & 0.02 \cellcolor{blue!60.00} & 0.01 \cellcolor{red!60.00} & 0.00 \cellcolor{green!18.00} & 0.00 \cellcolor{orange!18.00} & 0.08 \cellcolor{purple!60.00}\\ \hline 
SKA1W9-12A54B90 & 0.04 \cellcolor{blue!60.00} & 0.01 \cellcolor{red!18.00} & 0.00 \cellcolor{green!18.00} & 0.01 \cellcolor{orange!60.00} & 0.00 \cellcolor{purple!18.00} & 0.03 \cellcolor{blue!60.00} & 0.01 \cellcolor{red!60.00} & 0.00 \cellcolor{green!18.00} & 0.00 \cellcolor{orange!18.00} & 0.04 \cellcolor{purple!32.00} & 0.01 \cellcolor{blue!39.00} & 0.01 \cellcolor{red!60.00} & 0.00 \cellcolor{green!18.00} & 0.01 \cellcolor{orange!60.00} & 0.06 \cellcolor{purple!43.20}\\ \hline 
SKA1W9-12A60B100 & 0.04 \cellcolor{blue!60.00} & 0.01 \cellcolor{red!18.00} & 0.00 \cellcolor{green!18.00} & 0.00 \cellcolor{orange!18.00} & 0.01 \cellcolor{purple!39.00} & 0.03 \cellcolor{blue!60.00} & 0.01 \cellcolor{red!60.00} & 0.00 \cellcolor{green!18.00} & 0.00 \cellcolor{orange!18.00} & 0.06 \cellcolor{purple!60.00} & 0.01 \cellcolor{blue!39.00} & 0.00 \cellcolor{red!18.00} & 0.00 \cellcolor{green!18.00} & 0.00 \cellcolor{orange!18.00} & 0.06 \cellcolor{purple!43.20}\\ \hline 
SKA1W9-12A72B120 & 0.03 \cellcolor{blue!18.00} & 0.01 \cellcolor{red!18.00} & 0.00 \cellcolor{green!18.00} & 0.01 \cellcolor{orange!60.00} & 0.02 \cellcolor{purple!60.00} & 0.02 \cellcolor{blue!18.00} & 0.00 \cellcolor{red!18.00} & 0.00 \cellcolor{green!18.00} & 0.00 \cellcolor{orange!18.00} & 0.06 \cellcolor{purple!60.00} & 0.01 \cellcolor{blue!39.00} & 0.00 \cellcolor{red!18.00} & 0.01 \cellcolor{green!60.00} & 0.01 \cellcolor{orange!60.00} & 0.03 \cellcolor{purple!18.00}\\ \hline 
SKA1W9-12A80B133 & 0.03 \cellcolor{blue!18.00} & 0.01 \cellcolor{red!18.00} & 0.00 \cellcolor{green!18.00} & 0.00 \cellcolor{orange!18.00} & 0.00 \cellcolor{purple!18.00} & 0.02 \cellcolor{blue!18.00} & 0.00 \cellcolor{red!18.00} & 0.00 \cellcolor{green!18.00} & 0.00 \cellcolor{orange!18.00} & 0.05 \cellcolor{purple!46.00} & 0.00 \cellcolor{blue!18.00} & 0.01 \cellcolor{red!60.00} & 0.00 \cellcolor{green!18.00} & 0.00 \cellcolor{orange!18.00} & 0.07 \cellcolor{purple!51.60}\\ \hline 
\end{tabular}}
\vspace{-0.300000cm}
\hspace{1cm} 

\vspace{.25cm}
\caption{PSF symmetry (see \autoref{sec:exp})  for the different layouts at different scales. These values are generate at 650, 800 and 1100MHz, for angular scales \{0.4-1, 1-2, 2-3, 3-4, 600-3600\} arcsec labeled as {\it resbin} \{1, 2, 3, 4, 5\} respectively. This is done for natural, robust-2 weighting and robust-2 weighting with a 1 arcsec Gaussian taper, at declinations -10, -30, and -50 degrees. For each column, the intensity of the color increases with the value.}\label{tab:psf_sym}}
 \end{table}

% Auto generated table
 \nopagebreak 
 \begin{table}[!htp]
 \tiny{\subfloat[DEC=-10, natural weighting]{\begin{tabular}{|lccccc||ccccc||ccccc|} 
 \\ \cline{2-16} \multicolumn{1}{c}{ } & \multicolumn{5}{|c}{650MHz}  & \multicolumn{5}{c}{800MHz}  & \multicolumn{5}{c|}{1100MHz} \\ \cline{1-16} 
 resbin  &1 & 2 & 3 & 4 & 5 & 1 & 2 & 3 & 4 & 5 & 1 & 2 & 3 & 4 & 5 \\ \hline
SKA1REF2 & 2.80 \cellcolor{blue!20.97} & 3.00 \cellcolor{red!25.41} & 3.90 \cellcolor{green!19.01} & 4.70 \cellcolor{orange!19.47} & 9.70 \cellcolor{purple!18.00} & 2.60 \cellcolor{blue!22.87} & 2.90 \cellcolor{red!21.29} & 4.00 \cellcolor{green!19.02} & 4.80 \cellcolor{orange!18.00} & 12.00 \cellcolor{purple!18.00} & 2.50 \cellcolor{blue!24.46} & 3.00 \cellcolor{red!19.40} & 4.00 \cellcolor{green!18.00} & 4.70 \cellcolor{orange!18.00} & 17.00 \cellcolor{purple!18.00}\\ \hline 
SKA1W9-12A72B120 & 2.10 \cellcolor{blue!18.00} & 2.10 \cellcolor{red!18.00} & 3.70 \cellcolor{green!18.00} & 4.40 \cellcolor{orange!18.00} & 10.00 \cellcolor{purple!18.28} & 1.80 \cellcolor{blue!18.00} & 2.50 \cellcolor{red!18.00} & 3.80 \cellcolor{green!18.00} & 4.80 \cellcolor{orange!18.00} & 12.00 \cellcolor{purple!18.00} & 1.70 \cellcolor{blue!18.00} & 2.80 \cellcolor{red!18.00} & 4.00 \cellcolor{green!18.00} & 5.30 \cellcolor{orange!25.00} & 18.00 \cellcolor{purple!18.72}\\ \hline 
SKA1W9-0A72B120 & 2.20 \cellcolor{blue!18.42} & 2.30 \cellcolor{red!19.65} & 4.20 \cellcolor{green!20.53} & 5.00 \cellcolor{orange!20.93} & 10.00 \cellcolor{purple!18.28} & 1.90 \cellcolor{blue!18.61} & 2.80 \cellcolor{red!20.47} & 4.30 \cellcolor{green!20.56} & 5.20 \cellcolor{orange!21.23} & 13.00 \cellcolor{purple!18.82} & 1.80 \cellcolor{blue!18.81} & 3.20 \cellcolor{red!20.80} & 4.50 \cellcolor{green!22.37} & 6.10 \cellcolor{orange!34.33} & 18.00 \cellcolor{purple!18.72}\\ \hline 
SKASUR & 12.00 \cellcolor{blue!60.00} & 7.20 \cellcolor{red!60.00} & 12.00 \cellcolor{green!60.00} & 13.00 \cellcolor{orange!60.00} & 55.00 \cellcolor{purple!60.00} & 8.70 \cellcolor{blue!60.00} & 7.60 \cellcolor{red!60.00} & 12.00 \cellcolor{green!60.00} & 10.00 \cellcolor{orange!60.00} & 63.00 \cellcolor{purple!60.00} & 6.90 \cellcolor{blue!60.00} & 8.80 \cellcolor{red!60.00} & 8.80 \cellcolor{green!60.00} & 8.30 \cellcolor{orange!60.00} & 75.00 \cellcolor{purple!60.00}\\ \hline 
\end{tabular}}
\hspace{1cm} 
\subfloat[DEC=-10, robust-2 weighting ]{\begin{tabular}{|lccccc||ccccc||ccccc|} 
 \\ \cline{2-16} \multicolumn{1}{c}{ } & \multicolumn{5}{|c}{650MHz}  & \multicolumn{5}{c}{800MHz}  & \multicolumn{5}{c|}{1100MHz} \\ \cline{1-16} 
 resbin  &1 & 2 & 3 & 4 & 5 & 1 & 2 & 3 & 4 & 5 & 1 & 2 & 3 & 4 & 5 \\ \hline
SKA1REF2 & 3.20 \cellcolor{blue!21.17} & 3.90 \cellcolor{red!27.10} & 5.10 \cellcolor{green!20.23} & 5.60 \cellcolor{orange!19.56} & 9.40 \cellcolor{purple!18.00} & 3.20 \cellcolor{blue!22.04} & 4.10 \cellcolor{red!26.54} & 5.00 \cellcolor{green!19.79} & 5.30 \cellcolor{orange!18.43} & 12.00 \cellcolor{purple!18.00} & 3.20 \cellcolor{blue!22.94} & 4.30 \cellcolor{red!22.20} & 4.60 \cellcolor{green!18.98} & 4.80 \cellcolor{orange!18.00} & 17.00 \cellcolor{purple!18.00}\\ \hline 
SKA1W9-12A72B120 & 2.80 \cellcolor{blue!18.00} & 2.60 \cellcolor{red!18.00} & 4.60 \cellcolor{green!18.00} & 5.20 \cellcolor{orange!18.00} & 9.80 \cellcolor{purple!18.36} & 2.70 \cellcolor{blue!18.00} & 2.90 \cellcolor{red!18.00} & 4.60 \cellcolor{green!18.00} & 5.20 \cellcolor{orange!18.00} & 12.00 \cellcolor{purple!18.00} & 2.60 \cellcolor{blue!18.00} & 3.70 \cellcolor{red!18.00} & 4.40 \cellcolor{green!18.00} & 5.40 \cellcolor{orange!20.47} & 18.00 \cellcolor{purple!18.70}\\ \hline 
SKA1W9-0A72B120 & 2.90 \cellcolor{blue!18.79} & 2.90 \cellcolor{red!20.10} & 5.50 \cellcolor{green!22.02} & 6.30 \cellcolor{orange!22.28} & 9.80 \cellcolor{purple!18.36} & 2.90 \cellcolor{blue!19.62} & 3.40 \cellcolor{red!21.56} & 5.60 \cellcolor{green!22.47} & 6.50 \cellcolor{orange!23.57} & 13.00 \cellcolor{purple!18.79} & 2.80 \cellcolor{blue!19.65} & 4.30 \cellcolor{red!22.20} & 5.60 \cellcolor{green!23.86} & 6.80 \cellcolor{orange!26.24} & 18.00 \cellcolor{purple!18.70}\\ \hline 
SKASUR & 8.10 \cellcolor{blue!60.00} & 8.60 \cellcolor{red!60.00} & 14.00 \cellcolor{green!60.00} & 16.00 \cellcolor{orange!60.00} & 56.00 \cellcolor{purple!60.00} & 7.90 \cellcolor{blue!60.00} & 8.80 \cellcolor{red!60.00} & 14.00 \cellcolor{green!60.00} & 15.00 \cellcolor{orange!60.00} & 65.00 \cellcolor{purple!60.00} & 7.70 \cellcolor{blue!60.00} & 9.70 \cellcolor{red!60.00} & 13.00 \cellcolor{green!60.00} & 15.00 \cellcolor{orange!60.00} & 77.00 \cellcolor{purple!60.00}\\ \hline 
\end{tabular}}
\hspace{1cm} 
\subfloat[DEC=-10, robust-2 weighting with a 1 arcsec Gaussian taper]{\begin{tabular}{|lccccc||ccccc||ccccc|} 
 \\ \cline{2-16} \multicolumn{1}{c}{ } & \multicolumn{5}{|c}{650MHz}  & \multicolumn{5}{c}{800MHz}  & \multicolumn{5}{c|}{1100MHz} \\ \cline{1-16} 
 resbin  &1 & 2 & 3 & 4 & 5 & 1 & 2 & 3 & 4 & 5 & 1 & 2 & 3 & 4 & 5 \\ \hline
SKA1REF2 & 3.00 \cellcolor{blue!23.76} & 3.70 \cellcolor{red!25.70} & 5.00 \cellcolor{green!20.21} & 5.60 \cellcolor{orange!19.56} & 9.40 \cellcolor{purple!18.00} & 3.10 \cellcolor{blue!24.72} & 3.80 \cellcolor{red!24.52} & 5.00 \cellcolor{green!20.21} & 5.30 \cellcolor{orange!18.43} & 12.00 \cellcolor{purple!18.00} & 3.10 \cellcolor{blue!23.14} & 3.90 \cellcolor{red!21.50} & 4.50 \cellcolor{green!18.97} & 4.70 \cellcolor{orange!18.00} & 17.00 \cellcolor{purple!18.00}\\ \hline 
SKA1W9-12A72B120 & 2.30 \cellcolor{blue!18.00} & 2.60 \cellcolor{red!18.00} & 4.50 \cellcolor{green!18.00} & 5.20 \cellcolor{orange!18.00} & 9.80 \cellcolor{purple!18.36} & 2.30 \cellcolor{blue!18.00} & 2.90 \cellcolor{red!18.00} & 4.50 \cellcolor{green!18.00} & 5.20 \cellcolor{orange!18.00} & 12.00 \cellcolor{purple!18.00} & 2.50 \cellcolor{blue!18.00} & 3.40 \cellcolor{red!18.00} & 4.30 \cellcolor{green!18.00} & 5.40 \cellcolor{orange!20.85} & 18.00 \cellcolor{purple!18.70}\\ \hline 
SKA1W9-0A72B120 & 2.50 \cellcolor{blue!19.65} & 2.90 \cellcolor{red!20.10} & 5.50 \cellcolor{green!22.42} & 6.30 \cellcolor{orange!22.28} & 9.80 \cellcolor{purple!18.36} & 2.60 \cellcolor{blue!20.52} & 3.40 \cellcolor{red!21.62} & 5.50 \cellcolor{green!22.42} & 6.50 \cellcolor{orange!23.57} & 13.00 \cellcolor{purple!18.79} & 2.90 \cellcolor{blue!21.43} & 4.10 \cellcolor{red!22.90} & 5.50 \cellcolor{green!23.79} & 6.80 \cellcolor{orange!26.56} & 18.00 \cellcolor{purple!18.70}\\ \hline 
SKASUR & 7.40 \cellcolor{blue!60.00} & 8.60 \cellcolor{red!60.00} & 14.00 \cellcolor{green!60.00} & 16.00 \cellcolor{orange!60.00} & 56.00 \cellcolor{purple!60.00} & 7.30 \cellcolor{blue!60.00} & 8.70 \cellcolor{red!60.00} & 14.00 \cellcolor{green!60.00} & 15.00 \cellcolor{orange!60.00} & 65.00 \cellcolor{purple!60.00} & 7.40 \cellcolor{blue!60.00} & 9.40 \cellcolor{red!60.00} & 13.00 \cellcolor{green!60.00} & 15.00 \cellcolor{orange!60.00} & 77.00 \cellcolor{purple!60.00}\\ \hline 
\end{tabular}}
\hspace{1cm} 
\subfloat[DEC=-30, natural weighting]{\begin{tabular}{|lccccc||ccccc||ccccc|} 
 \\ \cline{2-16} \multicolumn{1}{c}{ } & \multicolumn{5}{|c}{650MHz}  & \multicolumn{5}{c}{800MHz}  & \multicolumn{5}{c|}{1100MHz} \\ \cline{1-16} 
 resbin  &1 & 2 & 3 & 4 & 5 & 1 & 2 & 3 & 4 & 5 & 1 & 2 & 3 & 4 & 5 \\ \hline
SKA1REF2 & 2.80 \cellcolor{blue!21.05} & 3.00 \cellcolor{red!24.11} & 3.90 \cellcolor{green!19.01} & 4.80 \cellcolor{orange!19.68} & 10.00 \cellcolor{purple!18.00} & 2.60 \cellcolor{blue!22.48} & 3.00 \cellcolor{red!20.29} & 4.10 \cellcolor{green!19.54} & 4.80 \cellcolor{orange!18.00} & 13.00 \cellcolor{purple!18.00} & 2.50 \cellcolor{blue!24.52} & 3.00 \cellcolor{red!19.35} & 4.00 \cellcolor{green!18.00} & 4.80 \cellcolor{orange!18.00} & 19.00 \cellcolor{purple!18.00}\\ \hline 
SKA1W9-12A72B120 & 2.00 \cellcolor{blue!18.00} & 2.20 \cellcolor{red!18.00} & 3.70 \cellcolor{green!18.00} & 4.50 \cellcolor{orange!18.00} & 11.00 \cellcolor{purple!19.05} & 1.80 \cellcolor{blue!18.00} & 2.70 \cellcolor{red!18.00} & 3.80 \cellcolor{green!18.00} & 4.80 \cellcolor{orange!18.00} & 13.00 \cellcolor{purple!18.00} & 1.60 \cellcolor{blue!18.00} & 2.80 \cellcolor{red!18.00} & 4.00 \cellcolor{green!18.00} & 5.30 \cellcolor{orange!23.38} & 19.00 \cellcolor{purple!18.00}\\ \hline 
SKA1W9-0A72B120 & 2.10 \cellcolor{blue!18.38} & 2.40 \cellcolor{red!19.53} & 4.30 \cellcolor{green!21.04} & 5.10 \cellcolor{orange!21.36} & 11.00 \cellcolor{purple!19.05} & 1.90 \cellcolor{blue!18.56} & 3.00 \cellcolor{red!20.29} & 4.20 \cellcolor{green!20.05} & 5.20 \cellcolor{orange!20.71} & 14.00 \cellcolor{purple!18.86} & 1.80 \cellcolor{blue!19.45} & 3.20 \cellcolor{red!20.71} & 4.60 \cellcolor{green!22.85} & 6.20 \cellcolor{orange!33.08} & 19.00 \cellcolor{purple!18.00}\\ \hline 
SKASUR & 13.00 \cellcolor{blue!60.00} & 7.70 \cellcolor{red!60.00} & 12.00 \cellcolor{green!60.00} & 12.00 \cellcolor{orange!60.00} & 50.00 \cellcolor{purple!60.00} & 9.30 \cellcolor{blue!60.00} & 8.20 \cellcolor{red!60.00} & 12.00 \cellcolor{green!60.00} & 11.00 \cellcolor{orange!60.00} & 62.00 \cellcolor{purple!60.00} & 7.40 \cellcolor{blue!60.00} & 9.00 \cellcolor{red!60.00} & 9.20 \cellcolor{green!60.00} & 8.70 \cellcolor{orange!60.00} & 75.00 \cellcolor{purple!60.00}\\ \hline 
\end{tabular}}
\hspace{1cm} 
\subfloat[DEC=-30, robust-2 weighting ]{\begin{tabular}{|lccccc||ccccc||ccccc|} 
 \\ \cline{2-16} \multicolumn{1}{c}{ } & \multicolumn{5}{|c}{650MHz}  & \multicolumn{5}{c}{800MHz}  & \multicolumn{5}{c|}{1100MHz} \\ \cline{1-16} 
 resbin  &1 & 2 & 3 & 4 & 5 & 1 & 2 & 3 & 4 & 5 & 1 & 2 & 3 & 4 & 5 \\ \hline
SKA1REF2 & 3.00 \cellcolor{blue!20.95} & 3.70 \cellcolor{red!26.67} & 4.70 \cellcolor{green!19.57} & 5.00 \cellcolor{orange!19.47} & 10.00 \cellcolor{purple!18.00} & 3.00 \cellcolor{blue!21.17} & 3.90 \cellcolor{red!25.24} & 4.50 \cellcolor{green!19.70} & 4.80 \cellcolor{orange!18.00} & 13.00 \cellcolor{purple!18.00} & 3.10 \cellcolor{blue!23.44} & 3.90 \cellcolor{red!21.33} & 4.20 \cellcolor{green!18.47} & 4.50 \cellcolor{orange!18.00} & 19.00 \cellcolor{purple!18.00}\\ \hline 
SKA1W9-12A72B120 & 2.60 \cellcolor{blue!18.00} & 2.40 \cellcolor{red!18.00} & 4.30 \cellcolor{green!18.00} & 4.60 \cellcolor{orange!18.00} & 11.00 \cellcolor{purple!19.00} & 2.60 \cellcolor{blue!18.00} & 2.90 \cellcolor{red!18.00} & 4.10 \cellcolor{green!18.00} & 4.90 \cellcolor{orange!18.38} & 13.00 \cellcolor{purple!18.00} & 2.40 \cellcolor{blue!18.00} & 3.40 \cellcolor{red!18.00} & 4.10 \cellcolor{green!18.00} & 5.10 \cellcolor{orange!20.65} & 19.00 \cellcolor{purple!18.00}\\ \hline 
SKA1W9-0A72B120 & 2.80 \cellcolor{blue!19.47} & 2.70 \cellcolor{red!20.00} & 5.10 \cellcolor{green!21.14} & 5.70 \cellcolor{orange!22.05} & 11.00 \cellcolor{purple!19.00} & 2.70 \cellcolor{blue!18.79} & 3.30 \cellcolor{red!20.90} & 5.10 \cellcolor{green!22.24} & 5.90 \cellcolor{orange!22.12} & 14.00 \cellcolor{purple!18.81} & 2.60 \cellcolor{blue!19.56} & 4.10 \cellcolor{red!22.67} & 4.90 \cellcolor{green!21.78} & 6.30 \cellcolor{orange!25.96} & 19.00 \cellcolor{purple!18.00}\\ \hline 
SKASUR & 8.30 \cellcolor{blue!60.00} & 8.70 \cellcolor{red!60.00} & 15.00 \cellcolor{green!60.00} & 16.00 \cellcolor{orange!60.00} & 52.00 \cellcolor{purple!60.00} & 7.90 \cellcolor{blue!60.00} & 8.70 \cellcolor{red!60.00} & 14.00 \cellcolor{green!60.00} & 16.00 \cellcolor{orange!60.00} & 65.00 \cellcolor{purple!60.00} & 7.80 \cellcolor{blue!60.00} & 9.70 \cellcolor{red!60.00} & 13.00 \cellcolor{green!60.00} & 14.00 \cellcolor{orange!60.00} & 79.00 \cellcolor{purple!60.00}\\ \hline 
\end{tabular}}
\hspace{1cm} 
\subfloat[DEC=-30, robust-2 weighting with a 1 arcsec Gaussian taper]{\begin{tabular}{|lccccc||ccccc||ccccc|} 
 \\ \cline{2-16} \multicolumn{1}{c}{ } & \multicolumn{5}{|c}{650MHz}  & \multicolumn{5}{c}{800MHz}  & \multicolumn{5}{c|}{1100MHz} \\ \cline{1-16} 
 resbin  &1 & 2 & 3 & 4 & 5 & 1 & 2 & 3 & 4 & 5 & 1 & 2 & 3 & 4 & 5 \\ \hline
SKA1REF2 & 2.90 \cellcolor{blue!24.22} & 3.50 \cellcolor{red!25.57} & 4.70 \cellcolor{green!20.14} & 4.90 \cellcolor{orange!19.11} & 10.00 \cellcolor{purple!18.00} & 2.80 \cellcolor{blue!23.88} & 3.60 \cellcolor{red!23.79} & 4.40 \cellcolor{green!19.27} & 4.80 \cellcolor{orange!18.00} & 13.00 \cellcolor{purple!18.00} & 2.90 \cellcolor{blue!23.04} & 3.50 \cellcolor{red!19.97} & 4.20 \cellcolor{green!18.47} & 4.50 \cellcolor{orange!18.00} & 19.00 \cellcolor{purple!18.00}\\ \hline 
SKA1W9-12A72B120 & 2.10 \cellcolor{blue!18.00} & 2.40 \cellcolor{red!18.00} & 4.20 \cellcolor{green!18.00} & 4.60 \cellcolor{orange!18.00} & 11.00 \cellcolor{purple!19.00} & 2.10 \cellcolor{blue!18.00} & 2.80 \cellcolor{red!18.00} & 4.10 \cellcolor{green!18.00} & 4.80 \cellcolor{orange!18.00} & 13.00 \cellcolor{purple!18.00} & 2.30 \cellcolor{blue!18.00} & 3.20 \cellcolor{red!18.00} & 4.10 \cellcolor{green!18.00} & 5.10 \cellcolor{orange!20.65} & 19.00 \cellcolor{purple!18.00}\\ \hline 
SKA1W9-0A72B120 & 2.40 \cellcolor{blue!20.33} & 2.80 \cellcolor{red!20.75} & 5.00 \cellcolor{green!21.43} & 5.70 \cellcolor{orange!22.05} & 11.00 \cellcolor{purple!19.00} & 2.40 \cellcolor{blue!20.52} & 3.30 \cellcolor{red!21.62} & 5.00 \cellcolor{green!21.82} & 5.90 \cellcolor{orange!22.12} & 14.00 \cellcolor{purple!18.81} & 2.70 \cellcolor{blue!21.36} & 3.80 \cellcolor{red!21.94} & 4.90 \cellcolor{green!21.78} & 6.30 \cellcolor{orange!25.96} & 19.00 \cellcolor{purple!18.00}\\ \hline 
SKASUR & 7.50 \cellcolor{blue!60.00} & 8.50 \cellcolor{red!60.00} & 14.00 \cellcolor{green!60.00} & 16.00 \cellcolor{orange!60.00} & 52.00 \cellcolor{purple!60.00} & 7.10 \cellcolor{blue!60.00} & 8.60 \cellcolor{red!60.00} & 14.00 \cellcolor{green!60.00} & 16.00 \cellcolor{orange!60.00} & 65.00 \cellcolor{purple!60.00} & 7.30 \cellcolor{blue!60.00} & 9.60 \cellcolor{red!60.00} & 13.00 \cellcolor{green!60.00} & 14.00 \cellcolor{orange!60.00} & 79.00 \cellcolor{purple!60.00}\\ \hline 
\end{tabular}}
\hspace{1cm} 
\subfloat[DEC=-50, natural weighting]{\begin{tabular}{|lccccc||ccccc||ccccc|} 
 \\ \cline{2-16} \multicolumn{1}{c}{ } & \multicolumn{5}{|c}{650MHz}  & \multicolumn{5}{c}{800MHz}  & \multicolumn{5}{c|}{1100MHz} \\ \cline{1-16} 
 resbin  &1 & 2 & 3 & 4 & 5 & 1 & 2 & 3 & 4 & 5 & 1 & 2 & 3 & 4 & 5 \\ \hline
SKA1REF2 & 2.80 \cellcolor{blue!20.24} & 3.00 \cellcolor{red!23.51} & 3.90 \cellcolor{green!19.50} & 4.80 \cellcolor{orange!19.00} & 10.00 \cellcolor{purple!18.00} & 2.50 \cellcolor{blue!21.61} & 3.00 \cellcolor{red!20.85} & 4.00 \cellcolor{green!19.02} & 4.80 \cellcolor{orange!18.67} & 13.00 \cellcolor{purple!18.00} & 2.50 \cellcolor{blue!23.48} & 3.00 \cellcolor{red!19.31} & 4.00 \cellcolor{green!18.00} & 4.80 \cellcolor{orange!18.00} & 18.00 \cellcolor{purple!18.00}\\ \hline 
SKA1W9-12A72B120 & 2.00 \cellcolor{blue!18.00} & 2.20 \cellcolor{red!18.00} & 3.60 \cellcolor{green!18.00} & 4.60 \cellcolor{orange!18.00} & 11.00 \cellcolor{purple!19.40} & 1.70 \cellcolor{blue!18.00} & 2.60 \cellcolor{red!18.00} & 3.80 \cellcolor{green!18.00} & 4.70 \cellcolor{orange!18.00} & 13.00 \cellcolor{purple!18.00} & 1.60 \cellcolor{blue!18.00} & 2.80 \cellcolor{red!18.00} & 4.00 \cellcolor{green!18.00} & 5.40 \cellcolor{orange!24.00} & 19.00 \cellcolor{purple!19.05}\\ \hline 
SKA1W9-0A72B120 & 2.10 \cellcolor{blue!18.28} & 2.40 \cellcolor{red!19.38} & 4.30 \cellcolor{green!21.50} & 5.10 \cellcolor{orange!20.50} & 11.00 \cellcolor{purple!19.40} & 1.90 \cellcolor{blue!18.90} & 3.00 \cellcolor{red!20.85} & 4.30 \cellcolor{green!20.56} & 5.20 \cellcolor{orange!21.33} & 13.00 \cellcolor{purple!18.00} & 1.80 \cellcolor{blue!19.22} & 3.20 \cellcolor{red!20.62} & 4.50 \cellcolor{green!21.89} & 6.30 \cellcolor{orange!33.00} & 19.00 \cellcolor{purple!19.05}\\ \hline 
SKASUR & 17.00 \cellcolor{blue!60.00} & 8.30 \cellcolor{red!60.00} & 12.00 \cellcolor{green!60.00} & 13.00 \cellcolor{orange!60.00} & 40.00 \cellcolor{purple!60.00} & 11.00 \cellcolor{blue!60.00} & 8.50 \cellcolor{red!60.00} & 12.00 \cellcolor{green!60.00} & 11.00 \cellcolor{orange!60.00} & 46.00 \cellcolor{purple!60.00} & 8.50 \cellcolor{blue!60.00} & 9.20 \cellcolor{red!60.00} & 9.40 \cellcolor{green!60.00} & 9.00 \cellcolor{orange!60.00} & 58.00 \cellcolor{purple!60.00}\\ \hline 
\end{tabular}}
\hspace{1cm} 
\subfloat[DEC=-50, robust-2 weighting ]{\begin{tabular}{|lccccc||ccccc||ccccc|} 
 \\ \cline{2-16} \multicolumn{1}{c}{ } & \multicolumn{5}{|c}{650MHz}  & \multicolumn{5}{c}{800MHz}  & \multicolumn{5}{c|}{1100MHz} \\ \cline{1-16} 
 resbin  &1 & 2 & 3 & 4 & 5 & 1 & 2 & 3 & 4 & 5 & 1 & 2 & 3 & 4 & 5 \\ \hline
SKA1REF2 & 3.00 \cellcolor{blue!21.00} & 3.60 \cellcolor{red!26.27} & 4.40 \cellcolor{green!19.68} & 4.80 \cellcolor{orange!19.10} & 9.80 \cellcolor{purple!18.00} & 2.90 \cellcolor{blue!20.95} & 3.80 \cellcolor{red!24.89} & 4.30 \cellcolor{green!19.66} & 4.60 \cellcolor{orange!18.00} & 13.00 \cellcolor{purple!18.00} & 3.00 \cellcolor{blue!22.58} & 3.80 \cellcolor{red!20.85} & 4.10 \cellcolor{green!18.47} & 4.40 \cellcolor{orange!18.00} & 18.00 \cellcolor{purple!18.00}\\ \hline 
SKA1W9-12A72B120 & 2.60 \cellcolor{blue!18.00} & 2.30 \cellcolor{red!18.00} & 4.00 \cellcolor{green!18.00} & 4.50 \cellcolor{orange!18.00} & 10.00 \cellcolor{purple!18.26} & 2.50 \cellcolor{blue!18.00} & 2.80 \cellcolor{red!18.00} & 3.90 \cellcolor{green!18.00} & 4.60 \cellcolor{orange!18.00} & 13.00 \cellcolor{purple!18.00} & 2.40 \cellcolor{blue!18.00} & 3.40 \cellcolor{red!18.00} & 4.00 \cellcolor{green!18.00} & 5.00 \cellcolor{orange!20.93} & 19.00 \cellcolor{purple!18.98}\\ \hline 
SKA1W9-0A72B120 & 2.70 \cellcolor{blue!18.75} & 2.60 \cellcolor{red!19.91} & 4.90 \cellcolor{green!21.78} & 5.50 \cellcolor{orange!21.65} & 10.00 \cellcolor{purple!18.26} & 2.70 \cellcolor{blue!19.47} & 3.20 \cellcolor{red!20.75} & 4.70 \cellcolor{green!21.33} & 5.50 \cellcolor{orange!21.63} & 13.00 \cellcolor{purple!18.00} & 2.60 \cellcolor{blue!19.53} & 4.00 \cellcolor{red!22.27} & 4.70 \cellcolor{green!21.27} & 5.90 \cellcolor{orange!25.33} & 19.00 \cellcolor{purple!18.98}\\ \hline 
SKASUR & 8.20 \cellcolor{blue!60.00} & 8.90 \cellcolor{red!60.00} & 14.00 \cellcolor{green!60.00} & 16.00 \cellcolor{orange!60.00} & 42.00 \cellcolor{purple!60.00} & 8.20 \cellcolor{blue!60.00} & 8.90 \cellcolor{red!60.00} & 14.00 \cellcolor{green!60.00} & 15.00 \cellcolor{orange!60.00} & 49.00 \cellcolor{purple!60.00} & 7.90 \cellcolor{blue!60.00} & 9.30 \cellcolor{red!60.00} & 13.00 \cellcolor{green!60.00} & 13.00 \cellcolor{orange!60.00} & 61.00 \cellcolor{purple!60.00}\\ \hline 
\end{tabular}}
\hspace{1cm} 
\subfloat[DEC=-50, robust-2 weighting with a 1 arcsec Gaussian taper]{\begin{tabular}{|lccccc||ccccc||ccccc|} 
 \\ \cline{2-16} \multicolumn{1}{c}{ } & \multicolumn{5}{|c}{650MHz}  & \multicolumn{5}{c}{800MHz}  & \multicolumn{5}{c|}{1100MHz} \\ \cline{1-16} 
 resbin  &1 & 2 & 3 & 4 & 5 & 1 & 2 & 3 & 4 & 5 & 1 & 2 & 3 & 4 & 5 \\ \hline
SKA1REF2 & 2.80 \cellcolor{blue!24.22} & 3.40 \cellcolor{red!24.89} & 4.40 \cellcolor{green!20.08} & 4.80 \cellcolor{orange!19.10} & 9.80 \cellcolor{purple!18.00} & 2.80 \cellcolor{blue!24.59} & 3.50 \cellcolor{red!23.79} & 4.30 \cellcolor{green!19.66} & 4.60 \cellcolor{orange!18.00} & 13.00 \cellcolor{purple!18.00} & 2.80 \cellcolor{blue!23.48} & 3.40 \cellcolor{red!20.10} & 4.00 \cellcolor{green!18.52} & 4.40 \cellcolor{orange!18.00} & 18.00 \cellcolor{purple!18.00}\\ \hline 
SKA1W9-12A72B120 & 2.00 \cellcolor{blue!18.00} & 2.40 \cellcolor{red!18.00} & 3.90 \cellcolor{green!18.00} & 4.50 \cellcolor{orange!18.00} & 10.00 \cellcolor{purple!18.26} & 2.00 \cellcolor{blue!18.00} & 2.70 \cellcolor{red!18.00} & 3.90 \cellcolor{green!18.00} & 4.60 \cellcolor{orange!18.00} & 13.00 \cellcolor{purple!18.00} & 2.20 \cellcolor{blue!18.00} & 3.10 \cellcolor{red!18.00} & 3.90 \cellcolor{green!18.00} & 5.00 \cellcolor{orange!20.93} & 19.00 \cellcolor{purple!18.98}\\ \hline 
SKA1W9-0A72B120 & 2.30 \cellcolor{blue!20.33} & 2.70 \cellcolor{red!20.07} & 4.80 \cellcolor{green!21.74} & 5.40 \cellcolor{orange!21.29} & 10.00 \cellcolor{purple!18.26} & 2.40 \cellcolor{blue!21.29} & 3.20 \cellcolor{red!21.62} & 4.70 \cellcolor{green!21.33} & 5.50 \cellcolor{orange!21.63} & 13.00 \cellcolor{purple!18.00} & 2.70 \cellcolor{blue!22.57} & 3.70 \cellcolor{red!22.20} & 4.60 \cellcolor{green!21.63} & 5.90 \cellcolor{orange!25.33} & 19.00 \cellcolor{purple!18.98}\\ \hline 
SKASUR & 7.40 \cellcolor{blue!60.00} & 8.50 \cellcolor{red!60.00} & 14.00 \cellcolor{green!60.00} & 16.00 \cellcolor{orange!60.00} & 42.00 \cellcolor{purple!60.00} & 7.10 \cellcolor{blue!60.00} & 8.50 \cellcolor{red!60.00} & 14.00 \cellcolor{green!60.00} & 15.00 \cellcolor{orange!60.00} & 49.00 \cellcolor{purple!60.00} & 6.80 \cellcolor{blue!60.00} & 9.10 \cellcolor{red!60.00} & 12.00 \cellcolor{green!60.00} & 13.00 \cellcolor{orange!60.00} & 61.00 \cellcolor{purple!60.00}\\ \hline 
\end{tabular}}
\hspace{1cm} 

\vspace{.0cm}
\caption{Noise (in $\mu$Jy) for a 50MHz band after an 8hr synthesis with a 60s integration for the differenr layouts at different angular scales. These values are generated at 650, 800 and 1100 MHz, at angular scales \{0.4-1, 1-2, 2-3, 3-4, 600-3600\} arcsec and are labeled {\it resbin} \{1, 2, 3, 4, 5\} respectively. This is done for natural and robust-2 weighting at declination -30 degrees. For each column, the intensity of the color increases with the value.}\label{tab:noise50}}
 \end{table}

% Auto generated table
 \vspace{-4cm}\begin{table}[H]
 \tiny{\subfloat[DEC=-10, natural weighting]{\begin{tabular}{|lccccc||ccccc||ccccc|} 
 \\ \cline{2-16} \multicolumn{1}{c}{ } & \multicolumn{5}{|c}{650MHz}  & \multicolumn{5}{c}{800MHz}  & \multicolumn{5}{c|}{1100MHz} \\ \cline{1-16} 
 resbin  &1 & 2 & 3 & 4 & 5 & 1 & 2 & 3 & 4 & 5 & 1 & 2 & 3 & 4 & 5 \\ \hline
SKA1REF2 & 1.60 \cellcolor{blue!54.32} & 1.60 \cellcolor{red!60.00} & 2.20 \cellcolor{green!39.00} & 2.60 \cellcolor{orange!18.00} & 5.30 \cellcolor{purple!18.00} & 1.40 \cellcolor{blue!60.00} & 1.60 \cellcolor{red!60.00} & 2.20 \cellcolor{green!18.00} & 2.60 \cellcolor{orange!18.00} & 6.60 \cellcolor{purple!18.00} & 1.40 \cellcolor{blue!60.00} & 1.60 \cellcolor{red!18.00} & 2.20 \cellcolor{green!18.00} & 2.60 \cellcolor{orange!18.00} & 9.40 \cellcolor{purple!18.00}\\ \hline 
SKA1X9-12A54B90 & 1.70 \cellcolor{blue!60.00} & 0.97 \cellcolor{red!18.00} & 2.10 \cellcolor{green!18.00} & 2.70 \cellcolor{orange!32.00} & 5.60 \cellcolor{purple!60.00} & 1.10 \cellcolor{blue!34.29} & 1.20 \cellcolor{red!18.00} & 2.20 \cellcolor{green!18.00} & 2.90 \cellcolor{orange!60.00} & 7.00 \cellcolor{purple!51.60} & 0.93 \cellcolor{blue!21.29} & 1.60 \cellcolor{red!18.00} & 2.40 \cellcolor{green!60.00} & 2.60 \cellcolor{orange!18.00} & 10.00 \cellcolor{purple!60.00}\\ \hline 
SKA1X9-12A60B100 & 1.30 \cellcolor{blue!37.30} & 1.00 \cellcolor{red!20.00} & 2.20 \cellcolor{green!39.00} & 2.70 \cellcolor{orange!32.00} & 5.60 \cellcolor{purple!60.00} & 0.97 \cellcolor{blue!23.14} & 1.40 \cellcolor{red!39.00} & 2.30 \cellcolor{green!39.00} & 2.70 \cellcolor{orange!32.00} & 7.10 \cellcolor{purple!60.00} & 0.91 \cellcolor{blue!19.65} & 1.70 \cellcolor{red!39.00} & 2.30 \cellcolor{green!39.00} & 2.70 \cellcolor{orange!32.00} & 10.00 \cellcolor{purple!60.00}\\ \hline 
SKA1X9-12A72B120 & 1.00 \cellcolor{blue!20.27} & 1.30 \cellcolor{red!40.00} & 2.30 \cellcolor{green!60.00} & 2.70 \cellcolor{orange!32.00} & 5.50 \cellcolor{purple!46.00} & 0.91 \cellcolor{blue!18.00} & 1.60 \cellcolor{red!60.00} & 2.40 \cellcolor{green!60.00} & 2.80 \cellcolor{orange!46.00} & 6.90 \cellcolor{purple!43.20} & 0.89 \cellcolor{blue!18.00} & 1.70 \cellcolor{red!39.00} & 2.40 \cellcolor{green!60.00} & 2.80 \cellcolor{orange!46.00} & 10.00 \cellcolor{purple!60.00}\\ \hline 
SKA1X9-12A80B133 & 0.96 \cellcolor{blue!18.00} & 1.50 \cellcolor{red!53.33} & 2.30 \cellcolor{green!60.00} & 2.90 \cellcolor{orange!60.00} & 5.60 \cellcolor{purple!60.00} & 0.92 \cellcolor{blue!18.86} & 1.60 \cellcolor{red!60.00} & 2.30 \cellcolor{green!39.00} & 2.90 \cellcolor{orange!60.00} & 7.00 \cellcolor{purple!51.60} & 0.90 \cellcolor{blue!18.82} & 1.80 \cellcolor{red!60.00} & 2.40 \cellcolor{green!60.00} & 2.90 \cellcolor{orange!60.00} & 10.00 \cellcolor{purple!60.00}\\ \hline 
\end{tabular}}
\vspace{-0.300000cm}
\hspace{1cm} 
\subfloat[DEC=-10, robust-2 weighting ]{\begin{tabular}{|lccccc||ccccc||ccccc|} 
 \\ \cline{2-16} \multicolumn{1}{c}{ } & \multicolumn{5}{|c}{650MHz}  & \multicolumn{5}{c}{800MHz}  & \multicolumn{5}{c|}{1100MHz} \\ \cline{1-16} 
 resbin  &1 & 2 & 3 & 4 & 5 & 1 & 2 & 3 & 4 & 5 & 1 & 2 & 3 & 4 & 5 \\ \hline
SKA1REF2 & 1.80 \cellcolor{blue!60.00} & 2.20 \cellcolor{red!60.00} & 2.80 \cellcolor{green!43.20} & 3.00 \cellcolor{orange!18.00} & 5.20 \cellcolor{purple!18.00} & 1.70 \cellcolor{blue!49.50} & 2.20 \cellcolor{red!60.00} & 2.70 \cellcolor{green!18.00} & 2.90 \cellcolor{orange!18.00} & 6.60 \cellcolor{purple!18.00} & 1.80 \cellcolor{blue!60.00} & 2.30 \cellcolor{red!51.60} & 2.50 \cellcolor{green!18.00} & 2.60 \cellcolor{orange!18.00} & 9.50 \cellcolor{purple!18.00}\\ \hline 
SKA1X9-12A54B90 & 1.80 \cellcolor{blue!60.00} & 1.50 \cellcolor{red!23.25} & 2.50 \cellcolor{green!18.00} & 3.20 \cellcolor{orange!39.00} & 5.50 \cellcolor{purple!60.00} & 1.80 \cellcolor{blue!60.00} & 1.40 \cellcolor{red!18.00} & 2.70 \cellcolor{green!18.00} & 3.10 \cellcolor{orange!39.00} & 7.00 \cellcolor{purple!51.60} & 1.70 \cellcolor{blue!51.60} & 1.90 \cellcolor{red!18.00} & 2.60 \cellcolor{green!28.50} & 2.80 \cellcolor{orange!32.00} & 10.00 \cellcolor{purple!60.00}\\ \hline 
SKA1X9-12A60B100 & 1.70 \cellcolor{blue!49.50} & 1.40 \cellcolor{red!18.00} & 2.70 \cellcolor{green!34.80} & 3.20 \cellcolor{orange!39.00} & 5.40 \cellcolor{purple!46.00} & 1.60 \cellcolor{blue!39.00} & 1.60 \cellcolor{red!28.50} & 2.80 \cellcolor{green!32.00} & 3.10 \cellcolor{orange!39.00} & 7.10 \cellcolor{purple!60.00} & 1.60 \cellcolor{blue!43.20} & 2.10 \cellcolor{red!34.80} & 2.60 \cellcolor{green!28.50} & 2.80 \cellcolor{orange!32.00} & 10.00 \cellcolor{purple!60.00}\\ \hline 
SKA1X9-12A72B120 & 1.50 \cellcolor{blue!28.50} & 1.50 \cellcolor{red!23.25} & 3.00 \cellcolor{green!60.00} & 3.30 \cellcolor{orange!49.50} & 5.40 \cellcolor{purple!46.00} & 1.50 \cellcolor{blue!28.50} & 1.90 \cellcolor{red!44.25} & 2.90 \cellcolor{green!46.00} & 3.20 \cellcolor{orange!49.50} & 6.90 \cellcolor{purple!43.20} & 1.40 \cellcolor{blue!26.40} & 2.40 \cellcolor{red!60.00} & 2.80 \cellcolor{green!49.50} & 3.00 \cellcolor{orange!46.00} & 10.00 \cellcolor{purple!60.00}\\ \hline 
SKA1X9-12A80B133 & 1.40 \cellcolor{blue!18.00} & 1.70 \cellcolor{red!33.75} & 3.00 \cellcolor{green!60.00} & 3.40 \cellcolor{orange!60.00} & 5.40 \cellcolor{purple!46.00} & 1.40 \cellcolor{blue!18.00} & 2.00 \cellcolor{red!49.50} & 3.00 \cellcolor{green!60.00} & 3.30 \cellcolor{orange!60.00} & 7.00 \cellcolor{purple!51.60} & 1.30 \cellcolor{blue!18.00} & 2.40 \cellcolor{red!60.00} & 2.90 \cellcolor{green!60.00} & 3.20 \cellcolor{orange!60.00} & 10.00 \cellcolor{purple!60.00}\\ \hline 
\end{tabular}}
\vspace{-0.300000cm}
\hspace{1cm} 
\subfloat[DEC=-10, robust-2 weighting with a 1 arcsec Gaussian taper]{\begin{tabular}{|lccccc||ccccc||ccccc|} 
 \\ \cline{2-16} \multicolumn{1}{c}{ } & \multicolumn{5}{|c}{650MHz}  & \multicolumn{5}{c}{800MHz}  & \multicolumn{5}{c|}{1100MHz} \\ \cline{1-16} 
 resbin  &1 & 2 & 3 & 4 & 5 & 1 & 2 & 3 & 4 & 5 & 1 & 2 & 3 & 4 & 5 \\ \hline
SKA1REF2 & 1.70 \cellcolor{blue!60.00} & 2.00 \cellcolor{red!60.00} & 2.80 \cellcolor{green!43.20} & 3.00 \cellcolor{orange!18.00} & 5.20 \cellcolor{purple!18.00} & 1.70 \cellcolor{blue!60.00} & 2.10 \cellcolor{red!60.00} & 2.70 \cellcolor{green!18.00} & 2.90 \cellcolor{orange!18.00} & 6.60 \cellcolor{purple!18.00} & 1.70 \cellcolor{blue!60.00} & 2.10 \cellcolor{red!49.50} & 2.50 \cellcolor{green!18.00} & 2.60 \cellcolor{orange!18.00} & 9.50 \cellcolor{purple!18.00}\\ \hline 
SKA1X9-12A54B90 & 1.40 \cellcolor{blue!28.50} & 1.40 \cellcolor{red!18.00} & 2.50 \cellcolor{green!18.00} & 3.20 \cellcolor{orange!39.00} & 5.50 \cellcolor{purple!60.00} & 1.30 \cellcolor{blue!18.00} & 1.50 \cellcolor{red!18.00} & 2.70 \cellcolor{green!18.00} & 3.10 \cellcolor{orange!39.00} & 7.00 \cellcolor{purple!51.60} & 1.30 \cellcolor{blue!18.00} & 1.80 \cellcolor{red!18.00} & 2.60 \cellcolor{green!32.00} & 2.80 \cellcolor{orange!32.00} & 10.00 \cellcolor{purple!60.00}\\ \hline 
SKA1X9-12A60B100 & 1.30 \cellcolor{blue!18.00} & 1.40 \cellcolor{red!18.00} & 2.70 \cellcolor{green!34.80} & 3.20 \cellcolor{orange!39.00} & 5.40 \cellcolor{purple!46.00} & 1.30 \cellcolor{blue!18.00} & 1.60 \cellcolor{red!25.00} & 2.80 \cellcolor{green!39.00} & 3.00 \cellcolor{orange!28.50} & 7.10 \cellcolor{purple!60.00} & 1.40 \cellcolor{blue!28.50} & 2.00 \cellcolor{red!39.00} & 2.60 \cellcolor{green!32.00} & 2.80 \cellcolor{orange!32.00} & 10.00 \cellcolor{purple!60.00}\\ \hline 
SKA1X9-12A72B120 & 1.30 \cellcolor{blue!18.00} & 1.60 \cellcolor{red!32.00} & 2.90 \cellcolor{green!51.60} & 3.30 \cellcolor{orange!49.50} & 5.40 \cellcolor{purple!46.00} & 1.40 \cellcolor{blue!28.50} & 1.90 \cellcolor{red!46.00} & 2.90 \cellcolor{green!60.00} & 3.20 \cellcolor{orange!49.50} & 6.90 \cellcolor{purple!43.20} & 1.50 \cellcolor{blue!39.00} & 2.20 \cellcolor{red!60.00} & 2.70 \cellcolor{green!46.00} & 3.00 \cellcolor{orange!46.00} & 10.00 \cellcolor{purple!60.00}\\ \hline 
SKA1X9-12A80B133 & 1.30 \cellcolor{blue!18.00} & 1.70 \cellcolor{red!39.00} & 3.00 \cellcolor{green!60.00} & 3.40 \cellcolor{orange!60.00} & 5.40 \cellcolor{purple!46.00} & 1.40 \cellcolor{blue!28.50} & 2.00 \cellcolor{red!53.00} & 2.90 \cellcolor{green!60.00} & 3.30 \cellcolor{orange!60.00} & 7.00 \cellcolor{purple!51.60} & 1.60 \cellcolor{blue!49.50} & 2.20 \cellcolor{red!60.00} & 2.80 \cellcolor{green!60.00} & 3.20 \cellcolor{orange!60.00} & 10.00 \cellcolor{purple!60.00}\\ \hline 
\end{tabular}}
\vspace{-0.300000cm}
\hspace{1cm} 
\subfloat[DEC=-30, natural weighting]{\begin{tabular}{|lccccc||ccccc||ccccc|} 
 \\ \cline{2-16} \multicolumn{1}{c}{ } & \multicolumn{5}{|c}{650MHz}  & \multicolumn{5}{c}{800MHz}  & \multicolumn{5}{c|}{1100MHz} \\ \cline{1-16} 
 resbin  &1 & 2 & 3 & 4 & 5 & 1 & 2 & 3 & 4 & 5 & 1 & 2 & 3 & 4 & 5 \\ \hline
SKA1REF2 & 1.50 \cellcolor{blue!60.00} & 1.60 \cellcolor{red!60.00} & 2.10 \cellcolor{green!18.00} & 2.60 \cellcolor{orange!18.00} & 5.60 \cellcolor{purple!18.00} & 1.40 \cellcolor{blue!60.00} & 1.60 \cellcolor{red!49.50} & 2.20 \cellcolor{green!18.00} & 2.60 \cellcolor{orange!18.00} & 7.10 \cellcolor{purple!18.00} & 1.40 \cellcolor{blue!60.00} & 1.60 \cellcolor{red!18.00} & 2.20 \cellcolor{green!18.00} & 2.60 \cellcolor{orange!18.00} & 10.00 \cellcolor{purple!18.00}\\ \hline 
SKA1X9-12A54B90 & 1.40 \cellcolor{blue!52.88} & 0.98 \cellcolor{red!18.00} & 2.20 \cellcolor{green!32.00} & 2.80 \cellcolor{orange!39.00} & 5.90 \cellcolor{purple!60.00} & 1.00 \cellcolor{blue!27.06} & 1.30 \cellcolor{red!18.00} & 2.20 \cellcolor{green!18.00} & 2.90 \cellcolor{orange!49.50} & 7.50 \cellcolor{purple!60.00} & 0.91 \cellcolor{blue!20.42} & 1.70 \cellcolor{red!39.00} & 2.30 \cellcolor{green!39.00} & 2.70 \cellcolor{orange!28.50} & 11.00 \cellcolor{purple!60.00}\\ \hline 
SKA1X9-12A60B100 & 1.20 \cellcolor{blue!38.64} & 1.10 \cellcolor{red!26.13} & 2.30 \cellcolor{green!46.00} & 2.70 \cellcolor{orange!28.50} & 5.90 \cellcolor{purple!60.00} & 0.96 \cellcolor{blue!23.76} & 1.50 \cellcolor{red!39.00} & 2.30 \cellcolor{green!39.00} & 2.70 \cellcolor{orange!28.50} & 7.50 \cellcolor{purple!60.00} & 0.90 \cellcolor{blue!19.62} & 1.70 \cellcolor{red!39.00} & 2.40 \cellcolor{green!60.00} & 2.80 \cellcolor{orange!39.00} & 11.00 \cellcolor{purple!60.00}\\ \hline 
SKA1X9-12A72B120 & 0.95 \cellcolor{blue!20.85} & 1.50 \cellcolor{red!53.23} & 2.40 \cellcolor{green!60.00} & 2.80 \cellcolor{orange!39.00} & 5.90 \cellcolor{purple!60.00} & 0.93 \cellcolor{blue!21.29} & 1.70 \cellcolor{red!60.00} & 2.40 \cellcolor{green!60.00} & 2.70 \cellcolor{orange!28.50} & 7.50 \cellcolor{purple!60.00} & 0.88 \cellcolor{blue!18.00} & 1.80 \cellcolor{red!60.00} & 2.40 \cellcolor{green!60.00} & 2.90 \cellcolor{orange!49.50} & 11.00 \cellcolor{purple!60.00}\\ \hline 
SKA1X9-12A80B133 & 0.91 \cellcolor{blue!18.00} & 1.50 \cellcolor{red!53.23} & 2.20 \cellcolor{green!32.00} & 3.00 \cellcolor{orange!60.00} & 5.90 \cellcolor{purple!60.00} & 0.89 \cellcolor{blue!18.00} & 1.70 \cellcolor{red!60.00} & 2.40 \cellcolor{green!60.00} & 3.00 \cellcolor{orange!60.00} & 7.50 \cellcolor{purple!60.00} & 0.92 \cellcolor{blue!21.23} & 1.70 \cellcolor{red!39.00} & 2.30 \cellcolor{green!39.00} & 3.00 \cellcolor{orange!60.00} & 11.00 \cellcolor{purple!60.00}\\ \hline 
\end{tabular}}
\vspace{-0.300000cm}
\hspace{1cm} 
\subfloat[DEC=-30, robust-2 weighting ]{\begin{tabular}{|lccccc||ccccc||ccccc|} 
 \\ \cline{2-16} \multicolumn{1}{c}{ } & \multicolumn{5}{|c}{650MHz}  & \multicolumn{5}{c}{800MHz}  & \multicolumn{5}{c|}{1100MHz} \\ \cline{1-16} 
 resbin  &1 & 2 & 3 & 4 & 5 & 1 & 2 & 3 & 4 & 5 & 1 & 2 & 3 & 4 & 5 \\ \hline
SKA1REF2 & 1.70 \cellcolor{blue!60.00} & 2.00 \cellcolor{red!60.00} & 2.60 \cellcolor{green!32.00} & 2.70 \cellcolor{orange!18.00} & 5.40 \cellcolor{purple!18.00} & 1.60 \cellcolor{blue!60.00} & 2.10 \cellcolor{red!60.00} & 2.50 \cellcolor{green!18.00} & 2.60 \cellcolor{orange!18.00} & 7.10 \cellcolor{purple!18.00} & 1.70 \cellcolor{blue!60.00} & 2.10 \cellcolor{red!39.00} & 2.30 \cellcolor{green!18.00} & 2.50 \cellcolor{orange!18.00} & 10.00 \cellcolor{purple!18.00}\\ \hline 
SKA1X9-12A54B90 & 1.70 \cellcolor{blue!60.00} & 1.30 \cellcolor{red!18.00} & 2.50 \cellcolor{green!18.00} & 2.90 \cellcolor{orange!39.00} & 5.70 \cellcolor{purple!49.50} & 1.60 \cellcolor{blue!60.00} & 1.30 \cellcolor{red!18.00} & 2.50 \cellcolor{green!18.00} & 2.80 \cellcolor{orange!39.00} & 7.50 \cellcolor{purple!60.00} & 1.60 \cellcolor{blue!51.60} & 1.90 \cellcolor{red!18.00} & 2.40 \cellcolor{green!28.50} & 2.60 \cellcolor{orange!28.50} & 11.00 \cellcolor{purple!60.00}\\ \hline 
SKA1X9-12A60B100 & 1.60 \cellcolor{blue!49.50} & 1.30 \cellcolor{red!18.00} & 2.60 \cellcolor{green!32.00} & 2.90 \cellcolor{orange!39.00} & 5.80 \cellcolor{purple!60.00} & 1.50 \cellcolor{blue!46.00} & 1.50 \cellcolor{red!28.50} & 2.60 \cellcolor{green!39.00} & 2.90 \cellcolor{orange!49.50} & 7.50 \cellcolor{purple!60.00} & 1.50 \cellcolor{blue!43.20} & 2.00 \cellcolor{red!28.50} & 2.40 \cellcolor{green!28.50} & 2.60 \cellcolor{orange!28.50} & 11.00 \cellcolor{purple!60.00}\\ \hline 
SKA1X9-12A72B120 & 1.40 \cellcolor{blue!28.50} & 1.50 \cellcolor{red!30.00} & 2.80 \cellcolor{green!60.00} & 3.10 \cellcolor{orange!60.00} & 5.70 \cellcolor{purple!49.50} & 1.40 \cellcolor{blue!32.00} & 1.90 \cellcolor{red!49.50} & 2.70 \cellcolor{green!60.00} & 3.00 \cellcolor{orange!60.00} & 7.50 \cellcolor{purple!60.00} & 1.30 \cellcolor{blue!26.40} & 2.20 \cellcolor{red!49.50} & 2.60 \cellcolor{green!49.50} & 2.80 \cellcolor{orange!49.50} & 11.00 \cellcolor{purple!60.00}\\ \hline 
SKA1X9-12A80B133 & 1.30 \cellcolor{blue!18.00} & 1.70 \cellcolor{red!42.00} & 2.80 \cellcolor{green!60.00} & 3.10 \cellcolor{orange!60.00} & 5.70 \cellcolor{purple!49.50} & 1.30 \cellcolor{blue!18.00} & 2.10 \cellcolor{red!60.00} & 2.70 \cellcolor{green!60.00} & 3.00 \cellcolor{orange!60.00} & 7.50 \cellcolor{purple!60.00} & 1.20 \cellcolor{blue!18.00} & 2.30 \cellcolor{red!60.00} & 2.70 \cellcolor{green!60.00} & 2.90 \cellcolor{orange!60.00} & 11.00 \cellcolor{purple!60.00}\\ \hline 
\end{tabular}}
\vspace{-0.300000cm}
\hspace{1cm} 
\subfloat[DEC=-30, robust-2 weighting with a 1 arcsec Gaussian taper]{\begin{tabular}{|lccccc||ccccc||ccccc|} 
 \\ \cline{2-16} \multicolumn{1}{c}{ } & \multicolumn{5}{|c}{650MHz}  & \multicolumn{5}{c}{800MHz}  & \multicolumn{5}{c|}{1100MHz} \\ \cline{1-16} 
 resbin  &1 & 2 & 3 & 4 & 5 & 1 & 2 & 3 & 4 & 5 & 1 & 2 & 3 & 4 & 5 \\ \hline
SKA1REF2 & 1.50 \cellcolor{blue!60.00} & 1.90 \cellcolor{red!60.00} & 2.50 \cellcolor{green!18.00} & 2.70 \cellcolor{orange!18.00} & 5.40 \cellcolor{purple!18.00} & 1.60 \cellcolor{blue!60.00} & 2.00 \cellcolor{red!60.00} & 2.40 \cellcolor{green!18.00} & 2.60 \cellcolor{orange!18.00} & 7.10 \cellcolor{purple!18.00} & 1.60 \cellcolor{blue!60.00} & 1.90 \cellcolor{red!32.00} & 2.30 \cellcolor{green!18.00} & 2.50 \cellcolor{orange!18.00} & 10.00 \cellcolor{purple!18.00}\\ \hline 
SKA1X9-12A54B90 & 1.20 \cellcolor{blue!18.00} & 1.30 \cellcolor{red!18.00} & 2.50 \cellcolor{green!18.00} & 2.90 \cellcolor{orange!39.00} & 5.70 \cellcolor{purple!49.50} & 1.20 \cellcolor{blue!18.00} & 1.40 \cellcolor{red!18.00} & 2.50 \cellcolor{green!32.00} & 2.80 \cellcolor{orange!39.00} & 7.50 \cellcolor{purple!60.00} & 1.30 \cellcolor{blue!18.00} & 1.80 \cellcolor{red!18.00} & 2.40 \cellcolor{green!32.00} & 2.60 \cellcolor{orange!28.50} & 11.00 \cellcolor{purple!60.00}\\ \hline 
SKA1X9-12A60B100 & 1.20 \cellcolor{blue!18.00} & 1.30 \cellcolor{red!18.00} & 2.60 \cellcolor{green!39.00} & 2.90 \cellcolor{orange!39.00} & 5.80 \cellcolor{purple!60.00} & 1.20 \cellcolor{blue!18.00} & 1.50 \cellcolor{red!25.00} & 2.50 \cellcolor{green!32.00} & 2.90 \cellcolor{orange!49.50} & 7.50 \cellcolor{purple!60.00} & 1.40 \cellcolor{blue!32.00} & 1.90 \cellcolor{red!32.00} & 2.40 \cellcolor{green!32.00} & 2.60 \cellcolor{orange!28.50} & 11.00 \cellcolor{purple!60.00}\\ \hline 
SKA1X9-12A72B120 & 1.20 \cellcolor{blue!18.00} & 1.60 \cellcolor{red!39.00} & 2.70 \cellcolor{green!60.00} & 3.10 \cellcolor{orange!60.00} & 5.70 \cellcolor{purple!49.50} & 1.30 \cellcolor{blue!28.50} & 1.90 \cellcolor{red!53.00} & 2.70 \cellcolor{green!60.00} & 3.00 \cellcolor{orange!60.00} & 7.50 \cellcolor{purple!60.00} & 1.50 \cellcolor{blue!46.00} & 2.10 \cellcolor{red!60.00} & 2.50 \cellcolor{green!46.00} & 2.80 \cellcolor{orange!49.50} & 11.00 \cellcolor{purple!60.00}\\ \hline 
SKA1X9-12A80B133 & 1.30 \cellcolor{blue!32.00} & 1.70 \cellcolor{red!46.00} & 2.70 \cellcolor{green!60.00} & 3.10 \cellcolor{orange!60.00} & 5.70 \cellcolor{purple!49.50} & 1.40 \cellcolor{blue!39.00} & 2.00 \cellcolor{red!60.00} & 2.70 \cellcolor{green!60.00} & 3.00 \cellcolor{orange!60.00} & 7.50 \cellcolor{purple!60.00} & 1.50 \cellcolor{blue!46.00} & 2.10 \cellcolor{red!60.00} & 2.60 \cellcolor{green!60.00} & 2.90 \cellcolor{orange!60.00} & 11.00 \cellcolor{purple!60.00}\\ \hline 
\end{tabular}}
\vspace{-0.300000cm}
\hspace{1cm} 
\subfloat[DEC=-50, natural weighting]{\begin{tabular}{|lccccc||ccccc||ccccc|} 
 \\ \cline{2-16} \multicolumn{1}{c}{ } & \multicolumn{5}{|c}{650MHz}  & \multicolumn{5}{c}{800MHz}  & \multicolumn{5}{c|}{1100MHz} \\ \cline{1-16} 
 resbin  &1 & 2 & 3 & 4 & 5 & 1 & 2 & 3 & 4 & 5 & 1 & 2 & 3 & 4 & 5 \\ \hline
SKA1REF2 & 1.50 \cellcolor{blue!60.00} & 1.60 \cellcolor{red!60.00} & 2.10 \cellcolor{green!18.00} & 2.60 \cellcolor{orange!18.00} & 5.50 \cellcolor{purple!18.00} & 1.40 \cellcolor{blue!60.00} & 1.60 \cellcolor{red!51.60} & 2.20 \cellcolor{green!18.00} & 2.60 \cellcolor{orange!18.00} & 7.00 \cellcolor{purple!18.00} & 1.40 \cellcolor{blue!60.00} & 1.60 \cellcolor{red!18.00} & 2.20 \cellcolor{green!18.00} & 2.60 \cellcolor{orange!18.00} & 10.00 \cellcolor{purple!18.00}\\ \hline 
SKA1X9-12A54B90 & 1.50 \cellcolor{blue!60.00} & 0.96 \cellcolor{red!18.00} & 2.20 \cellcolor{green!39.00} & 2.70 \cellcolor{orange!32.00} & 5.90 \cellcolor{purple!60.00} & 1.00 \cellcolor{blue!26.40} & 1.20 \cellcolor{red!18.00} & 2.20 \cellcolor{green!18.00} & 2.80 \cellcolor{orange!39.00} & 7.40 \cellcolor{purple!60.00} & 0.90 \cellcolor{blue!20.38} & 1.70 \cellcolor{red!39.00} & 2.40 \cellcolor{green!60.00} & 2.60 \cellcolor{orange!18.00} & 11.00 \cellcolor{purple!60.00}\\ \hline 
SKA1X9-12A60B100 & 1.20 \cellcolor{blue!37.89} & 1.10 \cellcolor{red!27.19} & 2.30 \cellcolor{green!60.00} & 2.70 \cellcolor{orange!32.00} & 5.90 \cellcolor{purple!60.00} & 0.94 \cellcolor{blue!21.36} & 1.50 \cellcolor{red!43.20} & 2.30 \cellcolor{green!39.00} & 2.70 \cellcolor{orange!28.50} & 7.40 \cellcolor{purple!60.00} & 0.89 \cellcolor{blue!19.58} & 1.70 \cellcolor{red!39.00} & 2.40 \cellcolor{green!60.00} & 2.80 \cellcolor{orange!46.00} & 11.00 \cellcolor{purple!60.00}\\ \hline 
SKA1X9-12A72B120 & 0.97 \cellcolor{blue!20.95} & 1.40 \cellcolor{red!46.87} & 2.30 \cellcolor{green!60.00} & 2.80 \cellcolor{orange!46.00} & 5.80 \cellcolor{purple!49.50} & 0.91 \cellcolor{blue!18.84} & 1.70 \cellcolor{red!60.00} & 2.40 \cellcolor{green!60.00} & 2.80 \cellcolor{orange!39.00} & 7.30 \cellcolor{purple!49.50} & 0.87 \cellcolor{blue!18.00} & 1.70 \cellcolor{red!39.00} & 2.40 \cellcolor{green!60.00} & 2.90 \cellcolor{orange!60.00} & 11.00 \cellcolor{purple!60.00}\\ \hline 
SKA1X9-12A80B133 & 0.93 \cellcolor{blue!18.00} & 1.60 \cellcolor{red!60.00} & 2.30 \cellcolor{green!60.00} & 2.90 \cellcolor{orange!60.00} & 5.90 \cellcolor{purple!60.00} & 0.90 \cellcolor{blue!18.00} & 1.70 \cellcolor{red!60.00} & 2.30 \cellcolor{green!39.00} & 3.00 \cellcolor{orange!60.00} & 7.40 \cellcolor{purple!60.00} & 0.91 \cellcolor{blue!21.17} & 1.80 \cellcolor{red!60.00} & 2.40 \cellcolor{green!60.00} & 2.90 \cellcolor{orange!60.00} & 11.00 \cellcolor{purple!60.00}\\ \hline 
\end{tabular}}
\vspace{-0.300000cm}
\hspace{1cm} 
\subfloat[DEC=-50, robust-2 weighting ]{\begin{tabular}{|lccccc||ccccc||ccccc|} 
 \\ \cline{2-16} \multicolumn{1}{c}{ } & \multicolumn{5}{|c}{650MHz}  & \multicolumn{5}{c}{800MHz}  & \multicolumn{5}{c|}{1100MHz} \\ \cline{1-16} 
 resbin  &1 & 2 & 3 & 4 & 5 & 1 & 2 & 3 & 4 & 5 & 1 & 2 & 3 & 4 & 5 \\ \hline
SKA1REF2 & 1.60 \cellcolor{blue!60.00} & 2.00 \cellcolor{red!60.00} & 2.50 \cellcolor{green!32.00} & 2.60 \cellcolor{orange!18.00} & 5.40 \cellcolor{purple!18.00} & 1.60 \cellcolor{blue!60.00} & 2.10 \cellcolor{red!60.00} & 2.40 \cellcolor{green!18.00} & 2.60 \cellcolor{orange!18.00} & 7.00 \cellcolor{purple!18.00} & 1.70 \cellcolor{blue!60.00} & 2.10 \cellcolor{red!49.50} & 2.20 \cellcolor{green!18.00} & 2.40 \cellcolor{orange!18.00} & 10.00 \cellcolor{purple!18.00}\\ \hline 
SKA1X9-12A54B90 & 1.60 \cellcolor{blue!60.00} & 1.30 \cellcolor{red!23.25} & 2.40 \cellcolor{green!18.00} & 2.70 \cellcolor{orange!32.00} & 5.60 \cellcolor{purple!46.00} & 1.60 \cellcolor{blue!60.00} & 1.30 \cellcolor{red!18.00} & 2.40 \cellcolor{green!18.00} & 2.70 \cellcolor{orange!39.00} & 7.40 \cellcolor{purple!60.00} & 1.60 \cellcolor{blue!51.60} & 1.80 \cellcolor{red!18.00} & 2.30 \cellcolor{green!32.00} & 2.60 \cellcolor{orange!39.00} & 11.00 \cellcolor{purple!60.00}\\ \hline 
SKA1X9-12A60B100 & 1.60 \cellcolor{blue!60.00} & 1.20 \cellcolor{red!18.00} & 2.50 \cellcolor{green!32.00} & 2.80 \cellcolor{orange!46.00} & 5.70 \cellcolor{purple!60.00} & 1.50 \cellcolor{blue!46.00} & 1.50 \cellcolor{red!28.50} & 2.40 \cellcolor{green!18.00} & 2.80 \cellcolor{orange!60.00} & 7.40 \cellcolor{purple!60.00} & 1.50 \cellcolor{blue!43.20} & 2.00 \cellcolor{red!39.00} & 2.30 \cellcolor{green!32.00} & 2.50 \cellcolor{orange!28.50} & 11.00 \cellcolor{purple!60.00}\\ \hline 
SKA1X9-12A72B120 & 1.40 \cellcolor{blue!32.00} & 1.40 \cellcolor{red!28.50} & 2.60 \cellcolor{green!46.00} & 2.90 \cellcolor{orange!60.00} & 5.70 \cellcolor{purple!60.00} & 1.40 \cellcolor{blue!32.00} & 1.90 \cellcolor{red!49.50} & 2.50 \cellcolor{green!39.00} & 2.80 \cellcolor{orange!60.00} & 7.30 \cellcolor{purple!49.50} & 1.30 \cellcolor{blue!26.40} & 2.20 \cellcolor{red!60.00} & 2.40 \cellcolor{green!46.00} & 2.70 \cellcolor{orange!49.50} & 11.00 \cellcolor{purple!60.00}\\ \hline 
SKA1X9-12A80B133 & 1.30 \cellcolor{blue!18.00} & 1.70 \cellcolor{red!44.25} & 2.70 \cellcolor{green!60.00} & 2.90 \cellcolor{orange!60.00} & 5.70 \cellcolor{purple!60.00} & 1.30 \cellcolor{blue!18.00} & 2.00 \cellcolor{red!54.75} & 2.60 \cellcolor{green!60.00} & 2.80 \cellcolor{orange!60.00} & 7.40 \cellcolor{purple!60.00} & 1.20 \cellcolor{blue!18.00} & 2.20 \cellcolor{red!60.00} & 2.50 \cellcolor{green!60.00} & 2.80 \cellcolor{orange!60.00} & 11.00 \cellcolor{purple!60.00}\\ \hline 
\end{tabular}}
\vspace{-0.300000cm}
\hspace{1cm} 
\subfloat[DEC=-50, robust-2 weighting with a 1 arcsec Gaussian taper]{\begin{tabular}{|lccccc||ccccc||ccccc|} 
 \\ \cline{2-16} \multicolumn{1}{c}{ } & \multicolumn{5}{|c}{650MHz}  & \multicolumn{5}{c}{800MHz}  & \multicolumn{5}{c|}{1100MHz} \\ \cline{1-16} 
 resbin  &1 & 2 & 3 & 4 & 5 & 1 & 2 & 3 & 4 & 5 & 1 & 2 & 3 & 4 & 5 \\ \hline
SKA1REF2 & 1.50 \cellcolor{blue!60.00} & 1.90 \cellcolor{red!60.00} & 2.50 \cellcolor{green!39.00} & 2.60 \cellcolor{orange!18.00} & 5.40 \cellcolor{purple!18.00} & 1.60 \cellcolor{blue!60.00} & 1.90 \cellcolor{red!60.00} & 2.30 \cellcolor{green!18.00} & 2.60 \cellcolor{orange!18.00} & 7.00 \cellcolor{purple!18.00} & 1.50 \cellcolor{blue!60.00} & 1.90 \cellcolor{red!46.00} & 2.20 \cellcolor{green!18.00} & 2.40 \cellcolor{orange!18.00} & 10.00 \cellcolor{purple!18.00}\\ \hline 
SKA1X9-12A54B90 & 1.20 \cellcolor{blue!18.00} & 1.20 \cellcolor{red!18.00} & 2.40 \cellcolor{green!18.00} & 2.70 \cellcolor{orange!32.00} & 5.60 \cellcolor{purple!46.00} & 1.20 \cellcolor{blue!18.00} & 1.40 \cellcolor{red!18.00} & 2.40 \cellcolor{green!39.00} & 2.70 \cellcolor{orange!39.00} & 7.40 \cellcolor{purple!60.00} & 1.20 \cellcolor{blue!18.00} & 1.70 \cellcolor{red!18.00} & 2.30 \cellcolor{green!32.00} & 2.50 \cellcolor{orange!28.50} & 11.00 \cellcolor{purple!60.00}\\ \hline 
SKA1X9-12A60B100 & 1.20 \cellcolor{blue!18.00} & 1.30 \cellcolor{red!24.00} & 2.40 \cellcolor{green!18.00} & 2.80 \cellcolor{orange!46.00} & 5.70 \cellcolor{purple!60.00} & 1.20 \cellcolor{blue!18.00} & 1.50 \cellcolor{red!26.40} & 2.40 \cellcolor{green!39.00} & 2.80 \cellcolor{orange!60.00} & 7.40 \cellcolor{purple!60.00} & 1.30 \cellcolor{blue!32.00} & 1.80 \cellcolor{red!32.00} & 2.30 \cellcolor{green!32.00} & 2.50 \cellcolor{orange!28.50} & 11.00 \cellcolor{purple!60.00}\\ \hline 
SKA1X9-12A72B120 & 1.20 \cellcolor{blue!18.00} & 1.50 \cellcolor{red!36.00} & 2.60 \cellcolor{green!60.00} & 2.90 \cellcolor{orange!60.00} & 5.70 \cellcolor{purple!60.00} & 1.30 \cellcolor{blue!28.50} & 1.80 \cellcolor{red!51.60} & 2.50 \cellcolor{green!60.00} & 2.80 \cellcolor{orange!60.00} & 7.30 \cellcolor{purple!49.50} & 1.50 \cellcolor{blue!60.00} & 2.00 \cellcolor{red!60.00} & 2.40 \cellcolor{green!46.00} & 2.60 \cellcolor{orange!39.00} & 11.00 \cellcolor{purple!60.00}\\ \hline 
SKA1X9-12A80B133 & 1.30 \cellcolor{blue!32.00} & 1.70 \cellcolor{red!48.00} & 2.60 \cellcolor{green!60.00} & 2.90 \cellcolor{orange!60.00} & 5.70 \cellcolor{purple!60.00} & 1.30 \cellcolor{blue!28.50} & 1.90 \cellcolor{red!60.00} & 2.50 \cellcolor{green!60.00} & 2.80 \cellcolor{orange!60.00} & 7.40 \cellcolor{purple!60.00} & 1.50 \cellcolor{blue!60.00} & 2.00 \cellcolor{red!60.00} & 2.50 \cellcolor{green!60.00} & 2.80 \cellcolor{orange!60.00} & 11.00 \cellcolor{purple!60.00}\\ \hline 
\end{tabular}}
\vspace{-0.300000cm}
\hspace{1cm} 

\vspace{.25cm}
\caption{Noise (in $\mu$Jy) for a 166MHz band after an 8hr synthesis with a 60s integration for the different layouts at different scales. These values are generated at 650, 800 and 1100 MHz, at angular scales \{0.4-1, 1-2, 2-3, 3-4, 600-3600\} arcsec labeled as {\it resbin} \{1, 2, 3, 4, 5\} respectively. This is done for natural, robust-2 weighting and robust-2 weighting with a 1 arcsec Gaussian taper, at declinations -10, -30 and -50 degrees. For each column, the intensity of the color increases with the value.}\label{tab:noise166}}
 \end{table}

% Auto generated table
 \vspace{-4cm}\begin{table}[H]
 \tiny{\subfloat[DEC=-10, natural weighting]{\begin{tabular}{|lccccc||ccccc||ccccc|} 
 \\ \cline{2-16} \multicolumn{1}{c}{ } & \multicolumn{5}{|c}{650MHz}  & \multicolumn{5}{c}{800MHz}  & \multicolumn{5}{c|}{1100MHz} \\ \cline{1-16} 
 resbin  &1 & 2 & 3 & 4 & 5 & 1 & 2 & 3 & 4 & 5 & 1 & 2 & 3 & 4 & 5 \\ \hline
SKA1REF2 & 9.23 \cellcolor{blue!21.76} & 8.77 \cellcolor{red!18.00} & 6.70 \cellcolor{green!51.28} & 5.66 \cellcolor{orange!60.00} & 2.74 \cellcolor{purple!60.00} & 8.83 \cellcolor{blue!18.00} & 7.68 \cellcolor{red!18.43} & 5.72 \cellcolor{green!60.00} & 4.82 \cellcolor{orange!60.00} & 1.88 \cellcolor{purple!60.00} & 7.13 \cellcolor{blue!18.00} & 6.11 \cellcolor{red!52.73} & 4.53 \cellcolor{green!60.00} & 3.85 \cellcolor{orange!60.00} & 1.06 \cellcolor{purple!60.00}\\ \hline 
SKA1X9-12A54B90 & 8.66 \cellcolor{blue!18.00} & 14.85 \cellcolor{red!60.00} & 6.81 \cellcolor{green!60.00} & 5.40 \cellcolor{orange!41.49} & 2.56 \cellcolor{purple!18.00} & 11.46 \cellcolor{blue!40.68} & 10.59 \cellcolor{red!60.00} & 5.63 \cellcolor{green!52.12} & 4.36 \cellcolor{orange!20.57} & 1.78 \cellcolor{purple!21.82} & 10.80 \cellcolor{blue!55.23} & 6.20 \cellcolor{red!60.00} & 4.25 \cellcolor{green!26.40} & 3.78 \cellcolor{orange!53.32} & 1.00 \cellcolor{purple!24.00}\\ \hline 
SKA1X9-12A60B100 & 11.45 \cellcolor{blue!36.42} & 13.78 \cellcolor{red!52.61} & 6.50 \cellcolor{green!35.43} & 5.32 \cellcolor{orange!35.80} & 2.59 \cellcolor{purple!25.00} & 12.82 \cellcolor{blue!52.41} & 9.18 \cellcolor{red!39.86} & 5.39 \cellcolor{green!31.12} & 4.56 \cellcolor{orange!37.71} & 1.77 \cellcolor{purple!18.00} & 10.95 \cellcolor{blue!56.75} & 6.05 \cellcolor{red!47.88} & 4.27 \cellcolor{green!28.80} & 3.65 \cellcolor{orange!40.91} & 1.00 \cellcolor{purple!24.00}\\ \hline 
SKA1X9-12A72B120 & 14.39 \cellcolor{blue!55.84} & 10.87 \cellcolor{red!32.51} & 6.28 \cellcolor{green!18.00} & 5.26 \cellcolor{orange!31.53} & 2.61 \cellcolor{purple!29.67} & 13.70 \cellcolor{blue!60.00} & 8.05 \cellcolor{red!23.71} & 5.24 \cellcolor{green!18.00} & 4.53 \cellcolor{orange!35.14} & 1.80 \cellcolor{purple!29.45} & 11.27 \cellcolor{blue!60.00} & 5.76 \cellcolor{red!24.46} & 4.18 \cellcolor{green!18.00} & 3.56 \cellcolor{orange!32.32} & 0.99 \cellcolor{purple!18.00}\\ \hline 
SKA1X9-12A80B133 & 15.02 \cellcolor{blue!60.00} & 9.91 \cellcolor{red!25.88} & 6.35 \cellcolor{green!23.55} & 5.07 \cellcolor{orange!18.00} & 2.60 \cellcolor{purple!27.33} & 13.63 \cellcolor{blue!59.40} & 7.65 \cellcolor{red!18.00} & 5.33 \cellcolor{green!25.87} & 4.33 \cellcolor{orange!18.00} & 1.79 \cellcolor{purple!25.64} & 11.12 \cellcolor{blue!58.48} & 5.68 \cellcolor{red!18.00} & 4.20 \cellcolor{green!20.40} & 3.41 \cellcolor{orange!18.00} & 1.00 \cellcolor{purple!24.00}\\ \hline 
\end{tabular}}
\vspace{-0.300000cm}
\hspace{1cm} 
\subfloat[DEC=-10, robust-2 weighting ]{\begin{tabular}{|lccccc||ccccc||ccccc|} 
 \\ \cline{2-16} \multicolumn{1}{c}{ } & \multicolumn{5}{|c}{650MHz}  & \multicolumn{5}{c}{800MHz}  & \multicolumn{5}{c|}{1100MHz} \\ \cline{1-16} 
 resbin  &1 & 2 & 3 & 4 & 5 & 1 & 2 & 3 & 4 & 5 & 1 & 2 & 3 & 4 & 5 \\ \hline
SKA1REF2 & 8.11 \cellcolor{blue!18.00} & 6.69 \cellcolor{red!18.00} & 5.16 \cellcolor{green!33.70} & 4.76 \cellcolor{orange!60.00} & 2.80 \cellcolor{purple!60.00} & 7.15 \cellcolor{blue!18.68} & 5.63 \cellcolor{red!18.00} & 4.56 \cellcolor{green!53.85} & 4.30 \cellcolor{orange!60.00} & 1.88 \cellcolor{purple!60.00} & 5.62 \cellcolor{blue!18.00} & 4.31 \cellcolor{red!24.89} & 3.96 \cellcolor{green!60.00} & 3.79 \cellcolor{orange!60.00} & 1.06 \cellcolor{purple!60.00}\\ \hline 
SKA1X9-12A54B90 & 8.14 \cellcolor{blue!18.61} & 9.66 \cellcolor{red!52.84} & 5.78 \cellcolor{green!60.00} & 4.50 \cellcolor{orange!39.00} & 2.65 \cellcolor{purple!18.00} & 7.12 \cellcolor{blue!18.00} & 8.68 \cellcolor{red!60.00} & 4.62 \cellcolor{green!60.00} & 4.06 \cellcolor{orange!40.62} & 1.78 \cellcolor{purple!21.82} & 5.97 \cellcolor{blue!25.99} & 5.38 \cellcolor{red!60.00} & 3.80 \cellcolor{green!45.39} & 3.51 \cellcolor{orange!41.91} & 1.00 \cellcolor{purple!24.00}\\ \hline 
SKA1X9-12A60B100 & 8.70 \cellcolor{blue!30.03} & 10.27 \cellcolor{red!60.00} & 5.33 \cellcolor{green!40.91} & 4.55 \cellcolor{orange!43.04} & 2.67 \cellcolor{purple!23.60} & 7.64 \cellcolor{blue!29.81} & 8.03 \cellcolor{red!51.05} & 4.47 \cellcolor{green!44.63} & 4.09 \cellcolor{orange!43.04} & 1.77 \cellcolor{purple!18.00} & 6.36 \cellcolor{blue!34.89} & 4.84 \cellcolor{red!42.28} & 3.83 \cellcolor{green!48.13} & 3.61 \cellcolor{orange!48.37} & 1.00 \cellcolor{purple!24.00}\\ \hline 
SKA1X9-12A72B120 & 9.53 \cellcolor{blue!46.95} & 9.52 \cellcolor{red!51.20} & 4.88 \cellcolor{green!21.82} & 4.33 \cellcolor{orange!25.27} & 2.70 \cellcolor{purple!32.00} & 8.48 \cellcolor{blue!48.88} & 6.69 \cellcolor{red!32.60} & 4.25 \cellcolor{green!22.10} & 3.90 \cellcolor{orange!27.69} & 1.80 \cellcolor{purple!29.45} & 7.16 \cellcolor{blue!53.15} & 4.25 \cellcolor{red!22.92} & 3.62 \cellcolor{green!28.96} & 3.37 \cellcolor{orange!32.86} & 0.99 \cellcolor{purple!18.00}\\ \hline 
SKA1X9-12A80B133 & 10.17 \cellcolor{blue!60.00} & 8.57 \cellcolor{red!40.06} & 4.79 \cellcolor{green!18.00} & 4.24 \cellcolor{orange!18.00} & 2.67 \cellcolor{purple!23.60} & 8.97 \cellcolor{blue!60.00} & 6.15 \cellcolor{red!25.16} & 4.21 \cellcolor{green!18.00} & 3.78 \cellcolor{orange!18.00} & 1.79 \cellcolor{purple!25.64} & 7.46 \cellcolor{blue!60.00} & 4.10 \cellcolor{red!18.00} & 3.50 \cellcolor{green!18.00} & 3.14 \cellcolor{orange!18.00} & 1.00 \cellcolor{purple!24.00}\\ \hline 
\end{tabular}}
\vspace{-0.300000cm}
\hspace{1cm} 
\subfloat[DEC=-10, robust-2 weighting with a 1 arcsec Gaussian taper]{\begin{tabular}{|lccccc||ccccc||ccccc|} 
 \\ \cline{2-16} \multicolumn{1}{c}{ } & \multicolumn{5}{|c}{650MHz}  & \multicolumn{5}{c}{800MHz}  & \multicolumn{5}{c|}{1100MHz} \\ \cline{1-16} 
 resbin  &1 & 2 & 3 & 4 & 5 & 1 & 2 & 3 & 4 & 5 & 1 & 2 & 3 & 4 & 5 \\ \hline
SKA1REF2 & 8.62 \cellcolor{blue!18.00} & 7.10 \cellcolor{red!18.00} & 5.22 \cellcolor{green!34.19} & 4.78 \cellcolor{orange!60.00} & 2.80 \cellcolor{purple!60.00} & 7.35 \cellcolor{blue!18.00} & 6.02 \cellcolor{red!18.00} & 4.63 \cellcolor{green!55.90} & 4.32 \cellcolor{orange!60.00} & 1.88 \cellcolor{purple!60.00} & 5.92 \cellcolor{blue!18.00} & 4.72 \cellcolor{red!28.40} & 4.02 \cellcolor{green!60.00} & 3.81 \cellcolor{orange!60.00} & 1.06 \cellcolor{purple!60.00}\\ \hline 
SKA1X9-12A54B90 & 10.61 \cellcolor{blue!51.04} & 10.16 \cellcolor{red!60.00} & 5.81 \cellcolor{green!60.00} & 4.52 \cellcolor{orange!39.40} & 2.65 \cellcolor{purple!18.00} & 9.62 \cellcolor{blue!58.92} & 8.49 \cellcolor{red!60.00} & 4.67 \cellcolor{green!60.00} & 4.08 \cellcolor{orange!40.98} & 1.78 \cellcolor{purple!21.82} & 7.64 \cellcolor{blue!60.00} & 5.54 \cellcolor{red!60.00} & 3.86 \cellcolor{green!45.70} & 3.54 \cellcolor{orange!42.82} & 1.00 \cellcolor{purple!24.00}\\ \hline 
SKA1X9-12A60B100 & 11.15 \cellcolor{blue!60.00} & 10.09 \cellcolor{red!59.04} & 5.38 \cellcolor{green!41.19} & 4.57 \cellcolor{orange!43.36} & 2.67 \cellcolor{purple!23.60} & 9.68 \cellcolor{blue!60.00} & 7.81 \cellcolor{red!48.44} & 4.54 \cellcolor{green!46.68} & 4.11 \cellcolor{orange!43.36} & 1.77 \cellcolor{purple!18.00} & 7.06 \cellcolor{blue!45.84} & 5.10 \cellcolor{red!43.05} & 3.89 \cellcolor{green!48.38} & 3.63 \cellcolor{orange!48.55} & 1.00 \cellcolor{purple!24.00}\\ \hline 
SKA1X9-12A72B120 & 11.15 \cellcolor{blue!60.00} & 9.13 \cellcolor{red!45.86} & 4.94 \cellcolor{green!21.94} & 4.35 \cellcolor{orange!25.92} & 2.70 \cellcolor{purple!32.00} & 9.20 \cellcolor{blue!51.35} & 6.74 \cellcolor{red!30.24} & 4.31 \cellcolor{green!23.12} & 3.92 \cellcolor{orange!28.30} & 1.80 \cellcolor{purple!29.45} & 6.51 \cellcolor{blue!32.41} & 4.57 \cellcolor{red!22.62} & 3.67 \cellcolor{green!28.72} & 3.39 \cellcolor{orange!33.27} & 0.99 \cellcolor{purple!18.00}\\ \hline 
SKA1X9-12A80B133 & 11.00 \cellcolor{blue!57.51} & 8.40 \cellcolor{red!35.84} & 4.85 \cellcolor{green!18.00} & 4.25 \cellcolor{orange!18.00} & 2.67 \cellcolor{purple!23.60} & 8.86 \cellcolor{blue!45.22} & 6.27 \cellcolor{red!22.25} & 4.26 \cellcolor{green!18.00} & 3.79 \cellcolor{orange!18.00} & 1.79 \cellcolor{purple!25.64} & 6.15 \cellcolor{blue!23.62} & 4.45 \cellcolor{red!18.00} & 3.55 \cellcolor{green!18.00} & 3.15 \cellcolor{orange!18.00} & 1.00 \cellcolor{purple!24.00}\\ \hline 
\end{tabular}}
\vspace{-0.300000cm}
\hspace{1cm} 
\subfloat[DEC=-30, natural weighting]{\begin{tabular}{|lccccc||ccccc||ccccc|} 
 \\ \cline{2-16} \multicolumn{1}{c}{ } & \multicolumn{5}{|c}{650MHz}  & \multicolumn{5}{c}{800MHz}  & \multicolumn{5}{c|}{1100MHz} \\ \cline{1-16} 
 resbin  &1 & 2 & 3 & 4 & 5 & 1 & 2 & 3 & 4 & 5 & 1 & 2 & 3 & 4 & 5 \\ \hline
SKA1REF2 & 9.52 \cellcolor{blue!18.00} & 8.85 \cellcolor{red!18.00} & 6.74 \cellcolor{green!60.00} & 5.51 \cellcolor{orange!60.00} & 2.57 \cellcolor{purple!60.00} & 8.86 \cellcolor{blue!18.00} & 7.65 \cellcolor{red!23.29} & 5.60 \cellcolor{green!50.57} & 4.79 \cellcolor{orange!60.00} & 1.76 \cellcolor{purple!60.00} & 7.14 \cellcolor{blue!18.00} & 6.11 \cellcolor{red!60.00} & 4.53 \cellcolor{green!60.00} & 3.86 \cellcolor{orange!60.00} & 1.00 \cellcolor{purple!60.00}\\ \hline 
SKA1X9-12A54B90 & 10.04 \cellcolor{blue!21.46} & 14.82 \cellcolor{red!60.00} & 6.59 \cellcolor{green!50.16} & 5.15 \cellcolor{orange!37.09} & 2.47 \cellcolor{purple!27.69} & 12.30 \cellcolor{blue!46.00} & 9.94 \cellcolor{red!60.00} & 5.71 \cellcolor{green!60.00} & 4.30 \cellcolor{orange!23.89} & 1.66 \cellcolor{purple!18.00} & 11.02 \cellcolor{blue!56.25} & 6.00 \cellcolor{red!48.73} & 4.32 \cellcolor{green!37.38} & 3.73 \cellcolor{orange!50.59} & 0.92 \cellcolor{purple!18.00}\\ \hline 
SKA1X9-12A60B100 & 12.35 \cellcolor{blue!36.84} & 13.26 \cellcolor{red!49.03} & 6.28 \cellcolor{green!29.81} & 5.29 \cellcolor{orange!46.00} & 2.44 \cellcolor{purple!18.00} & 12.99 \cellcolor{blue!51.62} & 8.46 \cellcolor{red!36.27} & 5.37 \cellcolor{green!30.86} & 4.56 \cellcolor{orange!43.05} & 1.67 \cellcolor{purple!22.20} & 11.13 \cellcolor{blue!57.34} & 5.87 \cellcolor{red!35.41} & 4.14 \cellcolor{green!18.00} & 3.62 \cellcolor{orange!42.62} & 0.94 \cellcolor{purple!28.50}\\ \hline 
SKA1X9-12A72B120 & 15.18 \cellcolor{blue!55.67} & 9.96 \cellcolor{red!25.81} & 6.10 \cellcolor{green!18.00} & 5.18 \cellcolor{orange!39.00} & 2.46 \cellcolor{purple!24.46} & 13.47 \cellcolor{blue!55.52} & 7.42 \cellcolor{red!19.60} & 5.22 \cellcolor{green!18.00} & 4.57 \cellcolor{orange!43.79} & 1.66 \cellcolor{purple!18.00} & 11.40 \cellcolor{blue!60.00} & 5.70 \cellcolor{red!18.00} & 4.14 \cellcolor{green!18.00} & 3.49 \cellcolor{orange!33.21} & 0.94 \cellcolor{purple!28.50}\\ \hline 
SKA1X9-12A80B133 & 15.83 \cellcolor{blue!60.00} & 9.43 \cellcolor{red!22.08} & 6.44 \cellcolor{green!40.31} & 4.85 \cellcolor{orange!18.00} & 2.45 \cellcolor{purple!21.23} & 14.02 \cellcolor{blue!60.00} & 7.32 \cellcolor{red!18.00} & 5.30 \cellcolor{green!24.86} & 4.22 \cellcolor{orange!18.00} & 1.67 \cellcolor{purple!22.20} & 10.82 \cellcolor{blue!54.28} & 5.72 \cellcolor{red!20.05} & 4.28 \cellcolor{green!33.08} & 3.28 \cellcolor{orange!18.00} & 0.92 \cellcolor{purple!18.00}\\ \hline 
\end{tabular}}
\vspace{-0.300000cm}
\hspace{1cm} 
\subfloat[DEC=-30, robust-2 weighting ]{\begin{tabular}{|lccccc||ccccc||ccccc|} 
 \\ \cline{2-16} \multicolumn{1}{c}{ } & \multicolumn{5}{|c}{650MHz}  & \multicolumn{5}{c}{800MHz}  & \multicolumn{5}{c|}{1100MHz} \\ \cline{1-16} 
 resbin  &1 & 2 & 3 & 4 & 5 & 1 & 2 & 3 & 4 & 5 & 1 & 2 & 3 & 4 & 5 \\ \hline
SKA1REF2 & 8.74 \cellcolor{blue!19.77} & 7.07 \cellcolor{red!18.00} & 5.62 \cellcolor{green!49.50} & 5.29 \cellcolor{orange!60.00} & 2.66 \cellcolor{purple!60.00} & 7.60 \cellcolor{blue!18.00} & 5.91 \cellcolor{red!18.00} & 5.03 \cellcolor{green!60.00} & 4.74 \cellcolor{orange!60.00} & 1.75 \cellcolor{purple!60.00} & 5.87 \cellcolor{blue!18.00} & 4.66 \cellcolor{red!30.38} & 4.34 \cellcolor{green!60.00} & 4.05 \cellcolor{orange!60.00} & 0.99 \cellcolor{purple!60.00}\\ \hline 
SKA1X9-12A54B90 & 8.65 \cellcolor{blue!18.00} & 11.24 \cellcolor{red!57.18} & 5.75 \cellcolor{green!60.00} & 5.02 \cellcolor{orange!41.10} & 2.52 \cellcolor{purple!27.33} & 7.60 \cellcolor{blue!18.00} & 9.61 \cellcolor{red!60.00} & 4.96 \cellcolor{green!53.61} & 4.50 \cellcolor{orange!43.20} & 1.66 \cellcolor{purple!18.00} & 6.29 \cellcolor{blue!25.98} & 5.33 \cellcolor{red!60.00} & 4.10 \cellcolor{green!42.32} & 3.81 \cellcolor{orange!43.74} & 0.92 \cellcolor{purple!18.00}\\ \hline 
SKA1X9-12A60B100 & 9.24 \cellcolor{blue!29.63} & 11.54 \cellcolor{red!60.00} & 5.55 \cellcolor{green!43.85} & 4.91 \cellcolor{orange!33.40} & 2.48 \cellcolor{purple!18.00} & 8.07 \cellcolor{blue!28.02} & 8.32 \cellcolor{red!45.36} & 4.83 \cellcolor{green!41.74} & 4.31 \cellcolor{orange!29.90} & 1.67 \cellcolor{purple!22.67} & 6.74 \cellcolor{blue!34.53} & 4.93 \cellcolor{red!42.32} & 4.08 \cellcolor{green!40.84} & 3.79 \cellcolor{orange!42.39} & 0.94 \cellcolor{purple!30.00}\\ \hline 
SKA1X9-12A72B120 & 10.15 \cellcolor{blue!47.58} & 9.66 \cellcolor{red!42.34} & 5.25 \cellcolor{green!19.62} & 4.69 \cellcolor{orange!18.00} & 2.53 \cellcolor{purple!29.67} & 8.95 \cellcolor{blue!46.78} & 6.48 \cellcolor{red!24.47} & 4.57 \cellcolor{green!18.00} & 4.14 \cellcolor{orange!18.00} & 1.66 \cellcolor{purple!18.00} & 7.63 \cellcolor{blue!51.45} & 4.48 \cellcolor{red!22.42} & 3.87 \cellcolor{green!25.37} & 3.53 \cellcolor{orange!24.77} & 0.94 \cellcolor{purple!30.00}\\ \hline 
SKA1X9-12A80B133 & 10.78 \cellcolor{blue!60.00} & 8.44 \cellcolor{red!30.87} & 5.23 \cellcolor{green!18.00} & 4.69 \cellcolor{orange!18.00} & 2.52 \cellcolor{purple!27.33} & 9.57 \cellcolor{blue!60.00} & 5.98 \cellcolor{red!18.79} & 4.59 \cellcolor{green!19.83} & 4.21 \cellcolor{orange!22.90} & 1.67 \cellcolor{purple!22.67} & 8.08 \cellcolor{blue!60.00} & 4.38 \cellcolor{red!18.00} & 3.77 \cellcolor{green!18.00} & 3.43 \cellcolor{orange!18.00} & 0.92 \cellcolor{purple!18.00}\\ \hline 
\end{tabular}}
\vspace{-0.300000cm}
\hspace{1cm} 
\subfloat[DEC=-30, robust-2 weighting with a 1 arcsec Gaussian taper]{\begin{tabular}{|lccccc||ccccc||ccccc|} 
 \\ \cline{2-16} \multicolumn{1}{c}{ } & \multicolumn{5}{|c}{650MHz}  & \multicolumn{5}{c}{800MHz}  & \multicolumn{5}{c|}{1100MHz} \\ \cline{1-16} 
 resbin  &1 & 2 & 3 & 4 & 5 & 1 & 2 & 3 & 4 & 5 & 1 & 2 & 3 & 4 & 5 \\ \hline
SKA1REF2 & 9.34 \cellcolor{blue!18.00} & 7.53 \cellcolor{red!18.00} & 5.70 \cellcolor{green!50.31} & 5.31 \cellcolor{orange!60.00} & 2.66 \cellcolor{purple!60.00} & 7.90 \cellcolor{blue!18.00} & 6.40 \cellcolor{red!19.52} & 5.11 \cellcolor{green!60.00} & 4.77 \cellcolor{orange!60.00} & 1.75 \cellcolor{purple!60.00} & 6.29 \cellcolor{blue!18.00} & 5.14 \cellcolor{red!36.15} & 4.42 \cellcolor{green!60.00} & 4.07 \cellcolor{orange!60.00} & 0.99 \cellcolor{purple!60.00}\\ \hline 
SKA1X9-12A54B90 & 11.66 \cellcolor{blue!53.56} & 11.45 \cellcolor{red!60.00} & 5.82 \cellcolor{green!60.00} & 5.04 \cellcolor{orange!41.10} & 2.52 \cellcolor{purple!27.33} & 10.54 \cellcolor{blue!60.00} & 9.06 \cellcolor{red!60.00} & 5.04 \cellcolor{green!53.74} & 4.52 \cellcolor{orange!42.79} & 1.66 \cellcolor{purple!18.00} & 7.89 \cellcolor{blue!60.00} & 5.60 \cellcolor{red!60.00} & 4.17 \cellcolor{green!41.90} & 3.84 \cellcolor{orange!44.42} & 0.92 \cellcolor{purple!18.00}\\ \hline 
SKA1X9-12A60B100 & 12.08 \cellcolor{blue!60.00} & 10.97 \cellcolor{red!54.86} & 5.62 \cellcolor{green!43.85} & 4.93 \cellcolor{orange!33.40} & 2.48 \cellcolor{purple!18.00} & 10.25 \cellcolor{blue!55.39} & 8.07 \cellcolor{red!44.93} & 4.90 \cellcolor{green!41.23} & 4.33 \cellcolor{orange!29.70} & 1.67 \cellcolor{purple!22.67} & 7.30 \cellcolor{blue!44.51} & 5.28 \cellcolor{red!43.41} & 4.15 \cellcolor{green!40.45} & 3.81 \cellcolor{orange!42.39} & 0.94 \cellcolor{purple!30.00}\\ \hline 
SKA1X9-12A72B120 & 11.69 \cellcolor{blue!54.02} & 9.23 \cellcolor{red!36.21} & 5.32 \cellcolor{green!19.62} & 4.71 \cellcolor{orange!18.00} & 2.53 \cellcolor{purple!29.67} & 9.55 \cellcolor{blue!44.25} & 6.68 \cellcolor{red!23.78} & 4.64 \cellcolor{green!18.00} & 4.16 \cellcolor{orange!18.00} & 1.66 \cellcolor{purple!18.00} & 6.63 \cellcolor{blue!26.93} & 4.87 \cellcolor{red!22.15} & 3.93 \cellcolor{green!24.52} & 3.55 \cellcolor{orange!24.77} & 0.94 \cellcolor{purple!30.00}\\ \hline 
SKA1X9-12A80B133 & 11.38 \cellcolor{blue!49.27} & 8.41 \cellcolor{red!27.43} & 5.30 \cellcolor{green!18.00} & 4.71 \cellcolor{orange!18.00} & 2.52 \cellcolor{purple!27.33} & 9.03 \cellcolor{blue!35.98} & 6.30 \cellcolor{red!18.00} & 4.66 \cellcolor{green!19.79} & 4.22 \cellcolor{orange!22.13} & 1.67 \cellcolor{purple!22.67} & 6.46 \cellcolor{blue!22.46} & 4.79 \cellcolor{red!18.00} & 3.84 \cellcolor{green!18.00} & 3.45 \cellcolor{orange!18.00} & 0.92 \cellcolor{purple!18.00}\\ \hline 
\end{tabular}}
\vspace{-0.300000cm}
\hspace{1cm} 
\subfloat[DEC=-50, natural weighting]{\begin{tabular}{|lccccc||ccccc||ccccc|} 
 \\ \cline{2-16} \multicolumn{1}{c}{ } & \multicolumn{5}{|c}{650MHz}  & \multicolumn{5}{c}{800MHz}  & \multicolumn{5}{c|}{1100MHz} \\ \cline{1-16} 
 resbin  &1 & 2 & 3 & 4 & 5 & 1 & 2 & 3 & 4 & 5 & 1 & 2 & 3 & 4 & 5 \\ \hline
SKA1REF2 & 9.48 \cellcolor{blue!18.00} & 8.81 \cellcolor{red!18.00} & 6.76 \cellcolor{green!60.00} & 5.57 \cellcolor{orange!60.00} & 2.60 \cellcolor{purple!60.00} & 8.87 \cellcolor{blue!18.00} & 7.73 \cellcolor{red!22.88} & 5.68 \cellcolor{green!60.00} & 4.87 \cellcolor{orange!60.00} & 1.79 \cellcolor{purple!60.00} & 7.09 \cellcolor{blue!18.00} & 6.14 \cellcolor{red!60.00} & 4.56 \cellcolor{green!60.00} & 3.87 \cellcolor{orange!60.00} & 1.00 \cellcolor{purple!60.00}\\ \hline 
SKA1X9-12A54B90 & 9.50 \cellcolor{blue!18.14} & 15.13 \cellcolor{red!60.00} & 6.53 \cellcolor{green!43.05} & 5.28 \cellcolor{orange!40.35} & 2.47 \cellcolor{purple!25.88} & 12.05 \cellcolor{blue!44.77} & 10.24 \cellcolor{red!60.00} & 5.59 \cellcolor{green!51.00} & 4.39 \cellcolor{orange!28.50} & 1.70 \cellcolor{purple!22.20} & 11.09 \cellcolor{blue!56.71} & 5.93 \cellcolor{red!40.40} & 4.23 \cellcolor{green!25.35} & 3.78 \cellcolor{orange!51.96} & 0.94 \cellcolor{purple!18.00}\\ \hline 
SKA1X9-12A60B100 & 12.00 \cellcolor{blue!35.49} & 13.63 \cellcolor{red!50.03} & 6.39 \cellcolor{green!32.74} & 5.26 \cellcolor{orange!39.00} & 2.46 \cellcolor{purple!23.25} & 13.30 \cellcolor{blue!55.29} & 8.58 \cellcolor{red!35.45} & 5.35 \cellcolor{green!27.00} & 4.55 \cellcolor{orange!39.00} & 1.69 \cellcolor{purple!18.00} & 11.28 \cellcolor{blue!58.55} & 5.85 \cellcolor{red!32.93} & 4.22 \cellcolor{green!24.30} & 3.62 \cellcolor{orange!37.66} & 0.95 \cellcolor{purple!25.00}\\ \hline 
SKA1X9-12A72B120 & 14.83 \cellcolor{blue!55.14} & 10.12 \cellcolor{red!26.71} & 6.19 \cellcolor{green!18.00} & 5.17 \cellcolor{orange!32.90} & 2.47 \cellcolor{purple!25.88} & 13.66 \cellcolor{blue!58.32} & 7.50 \cellcolor{red!19.48} & 5.26 \cellcolor{green!18.00} & 4.51 \cellcolor{orange!36.37} & 1.71 \cellcolor{purple!26.40} & 11.43 \cellcolor{blue!60.00} & 5.71 \cellcolor{red!19.87} & 4.16 \cellcolor{green!18.00} & 3.46 \cellcolor{orange!23.36} & 0.95 \cellcolor{purple!25.00}\\ \hline 
SKA1X9-12A80B133 & 15.53 \cellcolor{blue!60.00} & 9.24 \cellcolor{red!20.86} & 6.36 \cellcolor{green!30.53} & 4.95 \cellcolor{orange!18.00} & 2.44 \cellcolor{purple!18.00} & 13.86 \cellcolor{blue!60.00} & 7.40 \cellcolor{red!18.00} & 5.36 \cellcolor{green!28.00} & 4.23 \cellcolor{orange!18.00} & 1.69 \cellcolor{purple!18.00} & 10.96 \cellcolor{blue!55.45} & 5.69 \cellcolor{red!18.00} & 4.23 \cellcolor{green!25.35} & 3.40 \cellcolor{orange!18.00} & 0.94 \cellcolor{purple!18.00}\\ \hline 
\end{tabular}}
\vspace{-0.300000cm}
\hspace{1cm} 
\subfloat[DEC=-50, robust-2 weighting ]{\begin{tabular}{|lccccc||ccccc||ccccc|} 
 \\ \cline{2-16} \multicolumn{1}{c}{ } & \multicolumn{5}{|c}{650MHz}  & \multicolumn{5}{c}{800MHz}  & \multicolumn{5}{c|}{1100MHz} \\ \cline{1-16} 
 resbin  &1 & 2 & 3 & 4 & 5 & 1 & 2 & 3 & 4 & 5 & 1 & 2 & 3 & 4 & 5 \\ \hline
SKA1REF2 & 8.95 \cellcolor{blue!19.30} & 7.26 \cellcolor{red!18.00} & 5.79 \cellcolor{green!44.25} & 5.50 \cellcolor{orange!60.00} & 2.68 \cellcolor{purple!60.00} & 7.68 \cellcolor{blue!19.40} & 6.04 \cellcolor{red!18.00} & 5.25 \cellcolor{green!60.00} & 4.83 \cellcolor{orange!60.00} & 1.79 \cellcolor{purple!60.00} & 6.02 \cellcolor{blue!18.00} & 4.80 \cellcolor{red!30.00} & 4.53 \cellcolor{green!60.00} & 4.20 \cellcolor{orange!60.00} & 1.00 \cellcolor{purple!60.00}\\ \hline 
SKA1X9-12A54B90 & 8.89 \cellcolor{blue!18.00} & 11.35 \cellcolor{red!56.09} & 6.03 \cellcolor{green!60.00} & 5.26 \cellcolor{orange!42.32} & 2.56 \cellcolor{purple!26.40} & 7.61 \cellcolor{blue!18.00} & 9.83 \cellcolor{red!60.00} & 5.15 \cellcolor{green!50.00} & 4.62 \cellcolor{orange!39.00} & 1.70 \cellcolor{purple!22.20} & 6.38 \cellcolor{blue!25.00} & 5.45 \cellcolor{red!60.00} & 4.30 \cellcolor{green!42.44} & 3.91 \cellcolor{orange!42.35} & 0.94 \cellcolor{purple!18.00}\\ \hline 
SKA1X9-12A60B100 & 9.29 \cellcolor{blue!26.66} & 11.77 \cellcolor{red!60.00} & 5.82 \cellcolor{green!46.22} & 5.15 \cellcolor{orange!34.21} & 2.55 \cellcolor{purple!23.60} & 8.12 \cellcolor{blue!28.20} & 8.44 \cellcolor{red!44.60} & 5.12 \cellcolor{green!47.00} & 4.48 \cellcolor{orange!25.00} & 1.69 \cellcolor{purple!18.00} & 6.84 \cellcolor{blue!33.94} & 5.02 \cellcolor{red!40.15} & 4.32 \cellcolor{green!43.96} & 3.93 \cellcolor{orange!43.57} & 0.95 \cellcolor{purple!25.00}\\ \hline 
SKA1X9-12A72B120 & 10.25 \cellcolor{blue!47.44} & 9.99 \cellcolor{red!43.42} & 5.55 \cellcolor{green!28.50} & 4.93 \cellcolor{orange!18.00} & 2.56 \cellcolor{purple!26.40} & 9.17 \cellcolor{blue!49.20} & 6.65 \cellcolor{red!24.76} & 4.91 \cellcolor{green!26.00} & 4.41 \cellcolor{orange!18.00} & 1.71 \cellcolor{purple!26.40} & 7.80 \cellcolor{blue!52.61} & 4.59 \cellcolor{red!20.31} & 4.14 \cellcolor{green!30.22} & 3.76 \cellcolor{orange!33.22} & 0.95 \cellcolor{purple!25.00}\\ \hline 
SKA1X9-12A80B133 & 10.83 \cellcolor{blue!60.00} & 8.55 \cellcolor{red!30.01} & 5.39 \cellcolor{green!18.00} & 4.97 \cellcolor{orange!20.95} & 2.53 \cellcolor{purple!18.00} & 9.71 \cellcolor{blue!60.00} & 6.19 \cellcolor{red!19.66} & 4.83 \cellcolor{green!18.00} & 4.46 \cellcolor{orange!23.00} & 1.69 \cellcolor{purple!18.00} & 8.18 \cellcolor{blue!60.00} & 4.54 \cellcolor{red!18.00} & 3.98 \cellcolor{green!18.00} & 3.51 \cellcolor{orange!18.00} & 0.94 \cellcolor{purple!18.00}\\ \hline 
\end{tabular}}
\vspace{-0.300000cm}
\hspace{1cm} 
\subfloat[DEC=-50, robust-2 weighting with a 1 arcsec Gaussian taper]{\begin{tabular}{|lccccc||ccccc||ccccc|} 
 \\ \cline{2-16} \multicolumn{1}{c}{ } & \multicolumn{5}{|c}{650MHz}  & \multicolumn{5}{c}{800MHz}  & \multicolumn{5}{c|}{1100MHz} \\ \cline{1-16} 
 resbin  &1 & 2 & 3 & 4 & 5 & 1 & 2 & 3 & 4 & 5 & 1 & 2 & 3 & 4 & 5 \\ \hline
SKA1REF2 & 9.61 \cellcolor{blue!18.00} & 7.76 \cellcolor{red!18.00} & 5.87 \cellcolor{green!44.67} & 5.53 \cellcolor{orange!60.00} & 2.68 \cellcolor{purple!60.00} & 8.04 \cellcolor{blue!18.00} & 6.55 \cellcolor{red!18.00} & 5.34 \cellcolor{green!60.00} & 4.85 \cellcolor{orange!60.00} & 1.79 \cellcolor{purple!60.00} & 6.52 \cellcolor{blue!18.00} & 5.35 \cellcolor{red!36.95} & 4.61 \cellcolor{green!60.00} & 4.23 \cellcolor{orange!60.00} & 1.00 \cellcolor{purple!60.00}\\ \hline 
SKA1X9-12A54B90 & 11.87 \cellcolor{blue!56.27} & 11.59 \cellcolor{red!60.00} & 6.10 \cellcolor{green!60.00} & 5.28 \cellcolor{orange!41.90} & 2.56 \cellcolor{purple!26.40} & 10.67 \cellcolor{blue!60.00} & 9.24 \cellcolor{red!60.00} & 5.23 \cellcolor{green!49.26} & 4.64 \cellcolor{orange!39.00} & 1.70 \cellcolor{purple!22.20} & 8.20 \cellcolor{blue!60.00} & 5.80 \cellcolor{red!60.00} & 4.38 \cellcolor{green!42.75} & 3.94 \cellcolor{orange!42.60} & 0.94 \cellcolor{purple!18.00}\\ \hline 
SKA1X9-12A60B100 & 12.09 \cellcolor{blue!60.00} & 11.15 \cellcolor{red!55.17} & 5.90 \cellcolor{green!46.67} & 5.17 \cellcolor{orange!33.93} & 2.55 \cellcolor{purple!23.60} & 10.35 \cellcolor{blue!54.89} & 8.18 \cellcolor{red!43.45} & 5.19 \cellcolor{green!45.35} & 4.50 \cellcolor{orange!25.00} & 1.69 \cellcolor{purple!18.00} & 7.66 \cellcolor{blue!46.50} & 5.47 \cellcolor{red!43.10} & 4.39 \cellcolor{green!43.50} & 3.95 \cellcolor{orange!43.20} & 0.95 \cellcolor{purple!25.00}\\ \hline 
SKA1X9-12A72B120 & 12.03 \cellcolor{blue!58.98} & 9.55 \cellcolor{red!37.63} & 5.62 \cellcolor{green!28.00} & 4.95 \cellcolor{orange!18.00} & 2.56 \cellcolor{purple!26.40} & 9.78 \cellcolor{blue!45.79} & 6.92 \cellcolor{red!23.78} & 4.98 \cellcolor{green!24.84} & 4.43 \cellcolor{orange!18.00} & 1.71 \cellcolor{purple!26.40} & 6.79 \cellcolor{blue!24.75} & 5.02 \cellcolor{red!20.05} & 4.20 \cellcolor{green!29.25} & 3.78 \cellcolor{orange!33.00} & 0.95 \cellcolor{purple!25.00}\\ \hline 
SKA1X9-12A80B133 & 11.43 \cellcolor{blue!48.82} & 8.51 \cellcolor{red!26.22} & 5.47 \cellcolor{green!18.00} & 4.99 \cellcolor{orange!20.90} & 2.53 \cellcolor{purple!18.00} & 9.31 \cellcolor{blue!38.28} & 6.56 \cellcolor{red!18.16} & 4.91 \cellcolor{green!18.00} & 4.47 \cellcolor{orange!22.00} & 1.69 \cellcolor{purple!18.00} & 6.59 \cellcolor{blue!19.75} & 4.98 \cellcolor{red!18.00} & 4.05 \cellcolor{green!18.00} & 3.53 \cellcolor{orange!18.00} & 0.94 \cellcolor{purple!18.00}\\ \hline 
\end{tabular}}
\vspace{-0.300000cm}
\hspace{1cm} 

\vspace{.25cm}
\caption{SNR after 8 hours relative to a 10$\mu$Jy source at 1100Hz (166 MHz band) with a spectral index of -0.7 for the different layouts. These values are generated at 650, 800 and 1100 MHz, at angular scales \{0.4-1, 1-2, 2-3, 3-4, 600-3600\} arcsec labeled as {\it resbin} \{1, 2, 3, 4, 5\} respectively. This is done for natural, robust-2 weighting and robust-2 weighting with a 1 arcsec Gaussian taper, at declinations -10, -30 and -50 degrees. For each column, the intensity of the color increases with the value.}\label{tab:snr10}}
 \end{table}

% Auto generated table
 \vspace{-4cm}\begin{table}[H]
 \tiny{\subfloat[DEC=-10, natural weighting]{\begin{tabular}{|lccccc|} \hline resbin  &1 & 2 & 3 & 4 & 5  \\ \hline 
SKA1REF2 & 14.63 \cellcolor{blue!18.00} & 13.16 \cellcolor{red!18.00} & 9.90 \cellcolor{green!60.00} & 8.37 \cellcolor{orange!60.00} & 3.49 \cellcolor{purple!60.00}\\ \hline 
SKA1X9-12A54B90 & 17.97 \cellcolor{blue!34.48} & 19.26 \cellcolor{red!60.00} & 9.81 \cellcolor{green!54.68} & 7.90 \cellcolor{orange!37.57} & 3.27 \cellcolor{purple!18.00}\\ \hline 
SKA1X9-12A60B100 & 20.38 \cellcolor{blue!46.38} & 17.63 \cellcolor{red!48.78} & 9.46 \cellcolor{green!33.97} & 7.90 \cellcolor{orange!37.57} & 3.29 \cellcolor{purple!21.82}\\ \hline 
SKA1X9-12A72B120 & 22.85 \cellcolor{blue!58.57} & 14.70 \cellcolor{red!28.60} & 9.19 \cellcolor{green!18.00} & 7.80 \cellcolor{orange!32.80} & 3.32 \cellcolor{purple!27.55}\\ \hline 
SKA1X9-12A80B133 & 23.14 \cellcolor{blue!60.00} & 13.75 \cellcolor{red!22.06} & 9.29 \cellcolor{green!23.92} & 7.49 \cellcolor{orange!18.00} & 3.31 \cellcolor{purple!25.64}\\ \hline 
\end{tabular}}
\vspace{0.000000cm}
\hspace{1cm} 
\subfloat[DEC=-10, robust-2 weighting ]{\begin{tabular}{|lccccc|} \hline resbin  &1 & 2 & 3 & 4 & 5  \\ \hline 
SKA1REF2 & 12.19 \cellcolor{blue!18.00} & 9.75 \cellcolor{red!18.00} & 7.94 \cellcolor{green!44.80} & 7.45 \cellcolor{orange!60.00} & 3.54 \cellcolor{purple!60.00}\\ \hline 
SKA1X9-12A54B90 & 12.35 \cellcolor{blue!20.04} & 14.06 \cellcolor{red!60.00} & 8.32 \cellcolor{green!60.00} & 7.00 \cellcolor{orange!40.52} & 3.34 \cellcolor{purple!18.00}\\ \hline 
SKA1X9-12A60B100 & 13.21 \cellcolor{blue!31.02} & 13.90 \cellcolor{red!58.44} & 7.94 \cellcolor{green!44.80} & 7.11 \cellcolor{orange!45.28} & 3.36 \cellcolor{purple!22.20}\\ \hline 
SKA1X9-12A72B120 & 14.63 \cellcolor{blue!49.15} & 12.39 \cellcolor{red!43.73} & 7.41 \cellcolor{green!23.60} & 6.73 \cellcolor{orange!28.82} & 3.39 \cellcolor{purple!28.50}\\ \hline 
SKA1X9-12A80B133 & 15.48 \cellcolor{blue!60.00} & 11.31 \cellcolor{red!33.20} & 7.27 \cellcolor{green!18.00} & 6.48 \cellcolor{orange!18.00} & 3.37 \cellcolor{purple!24.30}\\ \hline 
\end{tabular}}
\vspace{0.000000cm}
\hspace{1cm} 
\subfloat[DEC=-10, robust-2 weighting with a 1 arcsec Gaussian taper]{\begin{tabular}{|lccccc|} \hline resbin  &1 & 2 & 3 & 4 & 5  \\ \hline 
SKA1REF2 & 12.78 \cellcolor{blue!18.00} & 10.44 \cellcolor{red!18.00} & 8.06 \cellcolor{green!46.41} & 7.49 \cellcolor{orange!60.00} & 3.54 \cellcolor{purple!60.00}\\ \hline 
SKA1X9-12A54B90 & 16.23 \cellcolor{blue!58.36} & 14.36 \cellcolor{red!60.00} & 8.39 \cellcolor{green!60.00} & 7.04 \cellcolor{orange!40.71} & 3.34 \cellcolor{purple!18.00}\\ \hline 
SKA1X9-12A60B100 & 16.37 \cellcolor{blue!60.00} & 13.74 \cellcolor{red!53.36} & 8.04 \cellcolor{green!45.59} & 7.14 \cellcolor{orange!45.00} & 3.36 \cellcolor{purple!22.20}\\ \hline 
SKA1X9-12A72B120 & 15.85 \cellcolor{blue!53.92} & 12.23 \cellcolor{red!37.18} & 7.51 \cellcolor{green!23.76} & 6.76 \cellcolor{orange!28.71} & 3.39 \cellcolor{purple!28.50}\\ \hline 
SKA1X9-12A80B133 & 15.41 \cellcolor{blue!48.77} & 11.39 \cellcolor{red!28.18} & 7.37 \cellcolor{green!18.00} & 6.51 \cellcolor{orange!18.00} & 3.37 \cellcolor{purple!24.30}\\ \hline 
\end{tabular}}
\vspace{0.000000cm}
\hspace{1cm} 
\subfloat[DEC=-30, natural weighting]{\begin{tabular}{|lccccc|} \hline resbin  &1 & 2 & 3 & 4 & 5  \\ \hline 
SKA1REF2 & 14.83 \cellcolor{blue!18.00} & 13.20 \cellcolor{red!18.00} & 9.86 \cellcolor{green!60.00} & 8.26 \cellcolor{orange!60.00} & 3.27 \cellcolor{purple!60.00}\\ \hline 
SKA1X9-12A54B90 & 19.33 \cellcolor{blue!39.16} & 18.82 \cellcolor{red!60.00} & 9.73 \cellcolor{green!53.42} & 7.67 \cellcolor{orange!36.17} & 3.11 \cellcolor{purple!20.47}\\ \hline 
SKA1X9-12A60B100 & 21.10 \cellcolor{blue!47.49} & 16.79 \cellcolor{red!44.83} & 9.24 \cellcolor{green!28.63} & 7.87 \cellcolor{orange!44.25} & 3.10 \cellcolor{purple!18.00}\\ \hline 
SKA1X9-12A72B120 & 23.27 \cellcolor{blue!57.70} & 13.67 \cellcolor{red!21.51} & 9.03 \cellcolor{green!18.00} & 7.74 \cellcolor{orange!39.00} & 3.11 \cellcolor{purple!20.47}\\ \hline 
SKA1X9-12A80B133 & 23.76 \cellcolor{blue!60.00} & 13.24 \cellcolor{red!18.30} & 9.37 \cellcolor{green!35.20} & 7.22 \cellcolor{orange!18.00} & 3.11 \cellcolor{purple!20.47}\\ \hline 
\end{tabular}}
\vspace{0.000000cm}
\hspace{1cm} 
\subfloat[DEC=-30, robust-2 weighting ]{\begin{tabular}{|lccccc|} \hline resbin  &1 & 2 & 3 & 4 & 5  \\ \hline 
SKA1REF2 & 12.99 \cellcolor{blue!18.00} & 10.33 \cellcolor{red!18.00} & 8.70 \cellcolor{green!60.00} & 8.18 \cellcolor{orange!60.00} & 3.34 \cellcolor{purple!60.00}\\ \hline 
SKA1X9-12A54B90 & 13.13 \cellcolor{blue!19.66} & 15.72 \cellcolor{red!60.00} & 8.63 \cellcolor{green!56.28} & 7.75 \cellcolor{orange!41.94} & 3.16 \cellcolor{purple!24.00}\\ \hline 
SKA1X9-12A60B100 & 14.00 \cellcolor{blue!29.98} & 15.06 \cellcolor{red!54.86} & 8.42 \cellcolor{green!45.11} & 7.55 \cellcolor{orange!33.54} & 3.13 \cellcolor{purple!18.00}\\ \hline 
SKA1X9-12A72B120 & 15.54 \cellcolor{blue!48.25} & 12.47 \cellcolor{red!34.68} & 7.97 \cellcolor{green!21.19} & 7.18 \cellcolor{orange!18.00} & 3.17 \cellcolor{purple!26.00}\\ \hline 
SKA1X9-12A80B133 & 16.53 \cellcolor{blue!60.00} & 11.23 \cellcolor{red!25.01} & 7.91 \cellcolor{green!18.00} & 7.18 \cellcolor{orange!18.00} & 3.16 \cellcolor{purple!24.00}\\ \hline 
\end{tabular}}
\vspace{0.000000cm}
\hspace{1cm} 
\subfloat[DEC=-30, robust-2 weighting with a 1 arcsec Gaussian taper]{\begin{tabular}{|lccccc|} \hline resbin  &1 & 2 & 3 & 4 & 5  \\ \hline 
SKA1REF2 & 13.75 \cellcolor{blue!18.00} & 11.14 \cellcolor{red!18.00} & 8.84 \cellcolor{green!60.00} & 8.22 \cellcolor{orange!60.00} & 3.34 \cellcolor{purple!60.00}\\ \hline 
SKA1X9-12A54B90 & 17.59 \cellcolor{blue!60.00} & 15.63 \cellcolor{red!60.00} & 8.75 \cellcolor{green!55.33} & 7.78 \cellcolor{orange!41.88} & 3.16 \cellcolor{purple!24.00}\\ \hline 
SKA1X9-12A60B100 & 17.44 \cellcolor{blue!58.36} & 14.61 \cellcolor{red!50.46} & 8.54 \cellcolor{green!44.44} & 7.59 \cellcolor{orange!34.06} & 3.13 \cellcolor{purple!18.00}\\ \hline 
SKA1X9-12A72B120 & 16.48 \cellcolor{blue!47.86} & 12.40 \cellcolor{red!29.79} & 8.08 \cellcolor{green!20.59} & 7.22 \cellcolor{orange!18.82} & 3.17 \cellcolor{purple!26.00}\\ \hline 
SKA1X9-12A80B133 & 15.90 \cellcolor{blue!41.52} & 11.55 \cellcolor{red!21.84} & 8.03 \cellcolor{green!18.00} & 7.20 \cellcolor{orange!18.00} & 3.16 \cellcolor{purple!24.00}\\ \hline 
\end{tabular}}
\vspace{0.000000cm}
\hspace{1cm} 
\subfloat[DEC=-50, natural weighting]{\begin{tabular}{|lccccc|} \hline resbin  &1 & 2 & 3 & 4 & 5  \\ \hline 
SKA1REF2 & 14.79 \cellcolor{blue!18.00} & 13.23 \cellcolor{red!18.69} & 9.94 \cellcolor{green!60.00} & 8.35 \cellcolor{orange!60.00} & 3.32 \cellcolor{purple!60.00}\\ \hline 
SKA1X9-12A54B90 & 18.94 \cellcolor{blue!37.97} & 19.20 \cellcolor{red!60.00} & 9.58 \cellcolor{green!41.33} & 7.84 \cellcolor{orange!38.58} & 3.14 \cellcolor{purple!24.00}\\ \hline 
SKA1X9-12A60B100 & 21.17 \cellcolor{blue!48.69} & 17.14 \cellcolor{red!45.75} & 9.35 \cellcolor{green!29.41} & 7.84 \cellcolor{orange!38.58} & 3.13 \cellcolor{purple!22.00}\\ \hline 
SKA1X9-12A72B120 & 23.18 \cellcolor{blue!58.36} & 13.83 \cellcolor{red!22.84} & 9.13 \cellcolor{green!18.00} & 7.68 \cellcolor{orange!31.86} & 3.15 \cellcolor{purple!26.00}\\ \hline 
SKA1X9-12A80B133 & 23.52 \cellcolor{blue!60.00} & 13.13 \cellcolor{red!18.00} & 9.33 \cellcolor{green!28.37} & 7.35 \cellcolor{orange!18.00} & 3.11 \cellcolor{purple!18.00}\\ \hline 
\end{tabular}}
\vspace{0.000000cm}
\hspace{1cm} 
\subfloat[DEC=-50, robust-2 weighting ]{\begin{tabular}{|lccccc|} \hline resbin  &1 & 2 & 3 & 4 & 5  \\ \hline 
SKA1REF2 & 13.24 \cellcolor{blue!18.00} & 10.60 \cellcolor{red!18.00} & 9.03 \cellcolor{green!60.00} & 8.44 \cellcolor{orange!60.00} & 3.37 \cellcolor{purple!60.00}\\ \hline 
SKA1X9-12A54B90 & 13.33 \cellcolor{blue!19.10} & 15.97 \cellcolor{red!60.00} & 9.02 \cellcolor{green!59.45} & 8.02 \cellcolor{orange!40.40} & 3.21 \cellcolor{purple!24.63}\\ \hline 
SKA1X9-12A60B100 & 14.11 \cellcolor{blue!28.59} & 15.33 \cellcolor{red!54.99} & 8.87 \cellcolor{green!51.27} & 7.87 \cellcolor{orange!33.40} & 3.20 \cellcolor{purple!22.42}\\ \hline 
SKA1X9-12A72B120 & 15.81 \cellcolor{blue!49.29} & 12.85 \cellcolor{red!35.60} & 8.48 \cellcolor{green!30.00} & 7.60 \cellcolor{orange!20.80} & 3.22 \cellcolor{purple!26.84}\\ \hline 
SKA1X9-12A80B133 & 16.69 \cellcolor{blue!60.00} & 11.49 \cellcolor{red!24.96} & 8.26 \cellcolor{green!18.00} & 7.54 \cellcolor{orange!18.00} & 3.18 \cellcolor{purple!18.00}\\ \hline 
\end{tabular}}
\vspace{0.000000cm}
\hspace{1cm} 
\subfloat[DEC=-50, robust-2 weighting with a 1 arcsec Gaussian taper]{\begin{tabular}{|lccccc|} \hline resbin  &1 & 2 & 3 & 4 & 5  \\ \hline 
SKA1REF2 & 14.12 \cellcolor{blue!18.00} & 11.48 \cellcolor{red!18.00} & 9.18 \cellcolor{green!60.00} & 8.49 \cellcolor{orange!60.00} & 3.37 \cellcolor{purple!60.00}\\ \hline 
SKA1X9-12A54B90 & 17.94 \cellcolor{blue!60.00} & 15.91 \cellcolor{red!60.00} & 9.15 \cellcolor{green!58.41} & 8.06 \cellcolor{orange!40.37} & 3.21 \cellcolor{purple!24.63}\\ \hline 
SKA1X9-12A60B100 & 17.66 \cellcolor{blue!56.92} & 14.87 \cellcolor{red!50.14} & 9.00 \cellcolor{green!50.43} & 7.91 \cellcolor{orange!33.52} & 3.20 \cellcolor{purple!22.42}\\ \hline 
SKA1X9-12A72B120 & 16.92 \cellcolor{blue!48.79} & 12.82 \cellcolor{red!30.70} & 8.61 \cellcolor{green!29.70} & 7.64 \cellcolor{orange!21.20} & 3.22 \cellcolor{purple!26.84}\\ \hline 
SKA1X9-12A80B133 & 16.15 \cellcolor{blue!40.32} & 11.85 \cellcolor{red!21.51} & 8.39 \cellcolor{green!18.00} & 7.57 \cellcolor{orange!18.00} & 3.18 \cellcolor{purple!18.00}\\ \hline 
\end{tabular}}
\vspace{0.000000cm}
\hspace{1cm} 

\vspace{.25cm}
\caption{SNR after 8 hours relative to a 10$\mu$Jy source at 1100Hz (166 MHz band) with a spectral index of -0.7 averaged over 650,800 and 1100MHz, for the different layouts at different scales. These values are generated for angular scales \{0.4-1, 1-2, 2-3, 3-4, 600-3600\} arcsec labeled as {\it resbin} \{1, 2, 3, 4, 5\} respectively. This is done for natural, robust-2 weighting and robust-2 weighting with a 1 arcsec Gaussian taper, at declinations -10, -30 and -50 degrees. For each column, the intensity of the color increases with the value.}\label{tab:snravg}}
 \end{table}

% Auto generated table
 \vspace{-4cm}\begin{table}[H]
 \tiny{\subfloat[DEC=-10, natural weighting]{\begin{tabular}{|lccccc|} \hline resbin  &1 & 2 & 3 & 4 & 5  \\ \hline 
SKA1REF2 & 3.74 \cellcolor{blue!60.00} & 4.62 \cellcolor{red!60.00} & 8.16 \cellcolor{green!18.00} & 11.41 \cellcolor{orange!18.00} & 65.67 \cellcolor{purple!18.00}\\ \hline 
SKA1X9-12A54B90 & 2.48 \cellcolor{blue!36.48} & 2.16 \cellcolor{red!18.00} & 8.31 \cellcolor{green!22.81} & 12.81 \cellcolor{orange!38.70} & 74.62 \cellcolor{purple!60.00}\\ \hline 
SKA1X9-12A60B100 & 1.93 \cellcolor{blue!26.21} & 2.57 \cellcolor{red!25.00} & 8.94 \cellcolor{green!43.01} & 12.83 \cellcolor{orange!39.00} & 73.84 \cellcolor{purple!56.34}\\ \hline 
SKA1X9-12A72B120 & 1.53 \cellcolor{blue!18.75} & 3.70 \cellcolor{red!44.29} & 9.47 \cellcolor{green!60.00} & 13.15 \cellcolor{orange!43.73} & 72.51 \cellcolor{purple!50.10}\\ \hline 
SKA1X9-12A80B133 & 1.49 \cellcolor{blue!18.00} & 4.23 \cellcolor{red!53.34} & 9.26 \cellcolor{green!53.27} & 14.25 \cellcolor{orange!60.00} & 72.99 \cellcolor{purple!52.35}\\ \hline 
\end{tabular}}
\vspace{0.000000cm}
\hspace{1cm} 
\subfloat[DEC=-10, robust-2 weighting ]{\begin{tabular}{|lccccc|} \hline resbin  &1 & 2 & 3 & 4 & 5  \\ \hline 
SKA1REF2 & 5.39 \cellcolor{blue!60.00} & 8.42 \cellcolor{red!60.00} & 12.69 \cellcolor{green!31.21} & 14.40 \cellcolor{orange!18.00} & 63.84 \cellcolor{purple!18.00}\\ \hline 
SKA1X9-12A54B90 & 5.24 \cellcolor{blue!56.93} & 4.05 \cellcolor{red!18.00} & 11.57 \cellcolor{green!18.00} & 16.31 \cellcolor{orange!35.33} & 71.58 \cellcolor{purple!60.00}\\ \hline 
SKA1X9-12A60B100 & 4.58 \cellcolor{blue!43.40} & 4.14 \cellcolor{red!18.86} & 12.68 \cellcolor{green!31.10} & 15.85 \cellcolor{orange!31.15} & 71.00 \cellcolor{purple!56.85}\\ \hline 
SKA1X9-12A72B120 & 3.74 \cellcolor{blue!26.20} & 5.21 \cellcolor{red!29.15} & 14.56 \cellcolor{green!53.28} & 17.64 \cellcolor{orange!47.39} & 69.52 \cellcolor{purple!48.82}\\ \hline 
SKA1X9-12A80B133 & 3.34 \cellcolor{blue!18.00} & 6.25 \cellcolor{red!39.14} & 15.13 \cellcolor{green!60.00} & 19.03 \cellcolor{orange!60.00} & 70.55 \cellcolor{purple!54.41}\\ \hline 
\end{tabular}}
\vspace{0.000000cm}
\hspace{1cm} 
\subfloat[DEC=-10, robust-2 weighting with a 1 arcsec Gaussian taper]{\begin{tabular}{|lccccc|} \hline resbin  &1 & 2 & 3 & 4 & 5  \\ \hline 
SKA1REF2 & 4.90 \cellcolor{blue!60.00} & 7.34 \cellcolor{red!60.00} & 12.33 \cellcolor{green!30.05} & 14.27 \cellcolor{orange!18.00} & 63.84 \cellcolor{purple!18.00}\\ \hline 
SKA1X9-12A54B90 & 3.04 \cellcolor{blue!19.10} & 3.88 \cellcolor{red!18.00} & 11.36 \cellcolor{green!18.00} & 16.15 \cellcolor{orange!35.09} & 71.58 \cellcolor{purple!60.00}\\ \hline 
SKA1X9-12A60B100 & 2.99 \cellcolor{blue!18.00} & 4.24 \cellcolor{red!22.37} & 12.37 \cellcolor{green!30.55} & 15.71 \cellcolor{orange!31.09} & 71.00 \cellcolor{purple!56.85}\\ \hline 
SKA1X9-12A72B120 & 3.18 \cellcolor{blue!22.18} & 5.35 \cellcolor{red!35.84} & 14.17 \cellcolor{green!52.92} & 17.50 \cellcolor{orange!47.36} & 69.52 \cellcolor{purple!48.82}\\ \hline 
SKA1X9-12A80B133 & 3.37 \cellcolor{blue!26.36} & 6.17 \cellcolor{red!45.80} & 14.74 \cellcolor{green!60.00} & 18.89 \cellcolor{orange!60.00} & 70.55 \cellcolor{purple!54.41}\\ \hline 
\end{tabular}}
\vspace{0.000000cm}
\hspace{1cm} 
\subfloat[DEC=-30, natural weighting]{\begin{tabular}{|lccccc|} \hline resbin  &1 & 2 & 3 & 4 & 5  \\ \hline 
SKA1REF2 & 3.64 \cellcolor{blue!60.00} & 4.59 \cellcolor{red!60.00} & 8.22 \cellcolor{green!18.00} & 11.73 \cellcolor{orange!18.00} & 74.76 \cellcolor{purple!18.00}\\ \hline 
SKA1X9-12A54B90 & 2.14 \cellcolor{blue!31.62} & 2.26 \cellcolor{red!18.00} & 8.45 \cellcolor{green!24.11} & 13.59 \cellcolor{orange!39.58} & 82.71 \cellcolor{purple!57.47}\\ \hline 
SKA1X9-12A60B100 & 1.80 \cellcolor{blue!25.19} & 2.84 \cellcolor{red!28.45} & 9.37 \cellcolor{green!48.57} & 12.93 \cellcolor{orange!31.92} & 83.22 \cellcolor{purple!60.00}\\ \hline 
SKA1X9-12A72B120 & 1.48 \cellcolor{blue!19.14} & 4.28 \cellcolor{red!54.41} & 9.80 \cellcolor{green!60.00} & 13.37 \cellcolor{orange!37.03} & 82.66 \cellcolor{purple!57.22}\\ \hline 
SKA1X9-12A80B133 & 1.42 \cellcolor{blue!18.00} & 4.56 \cellcolor{red!59.46} & 9.11 \cellcolor{green!41.66} & 15.35 \cellcolor{orange!60.00} & 82.82 \cellcolor{purple!58.01}\\ \hline 
\end{tabular}}
\vspace{0.000000cm}
\hspace{1cm} 
\subfloat[DEC=-30, robust-2 weighting ]{\begin{tabular}{|lccccc|} \hline resbin  &1 & 2 & 3 & 4 & 5  \\ \hline 
SKA1REF2 & 4.74 \cellcolor{blue!60.00} & 7.50 \cellcolor{red!60.00} & 10.57 \cellcolor{green!18.00} & 11.95 \cellcolor{orange!18.00} & 71.79 \cellcolor{purple!18.00}\\ \hline 
SKA1X9-12A54B90 & 4.64 \cellcolor{blue!57.68} & 3.24 \cellcolor{red!18.00} & 10.74 \cellcolor{green!21.25} & 13.33 \cellcolor{orange!34.19} & 80.26 \cellcolor{purple!55.02}\\ \hline 
SKA1X9-12A60B100 & 4.08 \cellcolor{blue!44.69} & 3.53 \cellcolor{red!20.86} & 11.30 \cellcolor{green!31.94} & 14.03 \cellcolor{orange!42.40} & 81.40 \cellcolor{purple!60.00}\\ \hline 
SKA1X9-12A72B120 & 3.31 \cellcolor{blue!26.82} & 5.15 \cellcolor{red!36.83} & 12.60 \cellcolor{green!56.75} & 15.51 \cellcolor{orange!59.77} & 79.79 \cellcolor{purple!52.96}\\ \hline 
SKA1X9-12A80B133 & 2.93 \cellcolor{blue!18.00} & 6.34 \cellcolor{red!48.56} & 12.77 \cellcolor{green!60.00} & 15.53 \cellcolor{orange!60.00} & 79.97 \cellcolor{purple!53.75}\\ \hline 
\end{tabular}}
\vspace{0.000000cm}
\hspace{1cm} 
\subfloat[DEC=-30, robust-2 weighting with a 1 arcsec Gaussian taper]{\begin{tabular}{|lccccc|} \hline resbin  &1 & 2 & 3 & 4 & 5  \\ \hline 
SKA1REF2 & 4.23 \cellcolor{blue!60.00} & 6.45 \cellcolor{red!60.00} & 10.23 \cellcolor{green!18.00} & 11.85 \cellcolor{orange!18.00} & 71.79 \cellcolor{purple!18.00}\\ \hline 
SKA1X9-12A54B90 & 2.59 \cellcolor{blue!18.00} & 3.27 \cellcolor{red!18.00} & 10.44 \cellcolor{green!22.06} & 13.21 \cellcolor{orange!34.00} & 80.26 \cellcolor{purple!55.02}\\ \hline 
SKA1X9-12A60B100 & 2.63 \cellcolor{blue!19.02} & 3.75 \cellcolor{red!24.34} & 10.98 \cellcolor{green!32.52} & 13.88 \cellcolor{orange!41.88} & 81.40 \cellcolor{purple!60.00}\\ \hline 
SKA1X9-12A72B120 & 2.94 \cellcolor{blue!26.96} & 5.21 \cellcolor{red!43.62} & 12.25 \cellcolor{green!57.10} & 15.36 \cellcolor{orange!59.29} & 79.79 \cellcolor{purple!52.96}\\ \hline 
SKA1X9-12A80B133 & 3.16 \cellcolor{blue!32.60} & 5.99 \cellcolor{red!53.92} & 12.40 \cellcolor{green!60.00} & 15.42 \cellcolor{orange!60.00} & 79.97 \cellcolor{purple!53.75}\\ \hline 
\end{tabular}}
\vspace{0.000000cm}
\hspace{1cm} 
\subfloat[DEC=-50, natural weighting]{\begin{tabular}{|lccccc|} \hline resbin  &1 & 2 & 3 & 4 & 5  \\ \hline 
SKA1REF2 & 3.66 \cellcolor{blue!60.00} & 4.57 \cellcolor{red!58.81} & 8.10 \cellcolor{green!18.00} & 11.48 \cellcolor{orange!18.00} & 72.71 \cellcolor{purple!18.00}\\ \hline 
SKA1X9-12A54B90 & 2.23 \cellcolor{blue!32.82} & 2.17 \cellcolor{red!18.00} & 8.72 \cellcolor{green!35.36} & 13.01 \cellcolor{orange!37.24} & 81.12 \cellcolor{purple!54.08}\\ \hline 
SKA1X9-12A60B100 & 1.78 \cellcolor{blue!24.27} & 2.72 \cellcolor{red!27.35} & 9.16 \cellcolor{green!47.68} & 13.01 \cellcolor{orange!37.24} & 81.59 \cellcolor{purple!56.10}\\ \hline 
SKA1X9-12A72B120 & 1.49 \cellcolor{blue!18.76} & 4.18 \cellcolor{red!52.18} & 9.60 \cellcolor{green!60.00} & 13.55 \cellcolor{orange!44.03} & 80.56 \cellcolor{purple!51.68}\\ \hline 
SKA1X9-12A80B133 & 1.45 \cellcolor{blue!18.00} & 4.64 \cellcolor{red!60.00} & 9.19 \cellcolor{green!48.52} & 14.82 \cellcolor{orange!60.00} & 82.50 \cellcolor{purple!60.00}\\ \hline 
\end{tabular}}
\vspace{0.000000cm}
\hspace{1cm} 
\subfloat[DEC=-50, robust-2 weighting ]{\begin{tabular}{|lccccc|} \hline resbin  &1 & 2 & 3 & 4 & 5  \\ \hline 
SKA1REF2 & 4.56 \cellcolor{blue!60.00} & 7.12 \cellcolor{red!60.00} & 9.81 \cellcolor{green!18.00} & 11.22 \cellcolor{orange!18.00} & 70.36 \cellcolor{purple!18.00}\\ \hline 
SKA1X9-12A54B90 & 4.50 \cellcolor{blue!58.51} & 3.14 \cellcolor{red!18.00} & 9.83 \cellcolor{green!18.44} & 12.44 \cellcolor{orange!36.04} & 77.49 \cellcolor{purple!52.82}\\ \hline 
SKA1X9-12A60B100 & 4.02 \cellcolor{blue!46.58} & 3.40 \cellcolor{red!20.74} & 10.17 \cellcolor{green!25.87} & 12.91 \cellcolor{orange!42.99} & 78.12 \cellcolor{purple!55.90}\\ \hline 
SKA1X9-12A72B120 & 3.20 \cellcolor{blue!26.20} & 4.84 \cellcolor{red!35.94} & 11.12 \cellcolor{green!46.66} & 13.84 \cellcolor{orange!56.75} & 77.30 \cellcolor{purple!51.89}\\ \hline 
SKA1X9-12A80B133 & 2.87 \cellcolor{blue!18.00} & 6.06 \cellcolor{red!48.81} & 11.73 \cellcolor{green!60.00} & 14.06 \cellcolor{orange!60.00} & 78.96 \cellcolor{purple!60.00}\\ \hline 
\end{tabular}}
\vspace{0.000000cm}
\hspace{1cm} 
\subfloat[DEC=-50, robust-2 weighting with a 1 arcsec Gaussian taper]{\begin{tabular}{|lccccc|} \hline resbin  &1 & 2 & 3 & 4 & 5  \\ \hline 
SKA1REF2 & 4.01 \cellcolor{blue!60.00} & 6.07 \cellcolor{red!60.00} & 9.50 \cellcolor{green!18.00} & 11.11 \cellcolor{orange!18.00} & 70.36 \cellcolor{purple!18.00}\\ \hline 
SKA1X9-12A54B90 & 2.49 \cellcolor{blue!18.00} & 3.16 \cellcolor{red!18.00} & 9.56 \cellcolor{green!19.35} & 12.32 \cellcolor{orange!35.89} & 77.49 \cellcolor{purple!52.82}\\ \hline 
SKA1X9-12A60B100 & 2.56 \cellcolor{blue!19.93} & 3.62 \cellcolor{red!24.64} & 9.88 \cellcolor{green!26.53} & 12.78 \cellcolor{orange!42.70} & 78.12 \cellcolor{purple!55.90}\\ \hline 
SKA1X9-12A72B120 & 2.79 \cellcolor{blue!26.29} & 4.87 \cellcolor{red!42.68} & 10.80 \cellcolor{green!47.20} & 13.70 \cellcolor{orange!56.30} & 77.30 \cellcolor{purple!51.89}\\ \hline 
SKA1X9-12A80B133 & 3.07 \cellcolor{blue!34.03} & 5.70 \cellcolor{red!54.66} & 11.37 \cellcolor{green!60.00} & 13.95 \cellcolor{orange!60.00} & 78.96 \cellcolor{purple!60.00}\\ \hline 
\end{tabular}}
\vspace{0.000000cm}
\hspace{1cm} 

\vspace{.25cm}
\caption{The hours required to reach a mean SNR of 10 (average over 650,800 and 1100MHz), assuming a 10$\mu$Jy source at 1100MHz with a spectral index of -0.7 for the different layouts at different scales. These values are generated for angular scales \{0.4-1, 1-2, 2-3, 3-4, 600-3600\} arcsec labeled as {\it resbin} \{1, 2, 3, 4, 5\} respectively. This is done forr natural, robust-2 weighting and robust-2 weighting with a 1 arcsec Gaussian taper, at declinations -10, -30 and -50 degrees. For each column, the intensity of the color increases with the value.}\label{tab:hours}}
 \end{table}


%===============================================================================================================
\section{Conclusions}\label{sec:conclusion}
The metrics we have used suggest that the science goals (at least those listed in the SRD) can be met by a layout which
covers significantly less space compared to the baseline layout. Some of these ``smaller'' layouts perform better than the
baseline layout at smaller scales, up to a 50\% improvement in terms of the noise properties, while not compromising the larger
scales. This obviously presents an opportunity to reduce trenching and data transport costs. Moreover, bringing in the dishes
further out translates to a greater sensitivity on the relevant (to the science goals of SKA1-Mid) smaller scales, as can be seen
in Tables \ref{tab:noise50}-\ref{tab:hours}. Even more encouraging is the fact that this doesn't compromise the size or the
symmetry of the PSF as seen in Tables \ref{tab:psf_mean} and \ref{tab:psf_sym}.
\begin{thebibliography}{99}
 \bibitem{bd} \url{http://www.skatelescope.org/wp-content/uploads/2013/05/SKA-TEL-SKO-DD-001-1_BaselineDesign1.pdf}
 \bibitem{srd} \url{https://www.skatelescope.org/wp-content/uploads/2014/03/SKA-TEL_SCI-SKO-SRQ-001-1_Level_0_Requirements-1.pdf}
%  \bibitem{meqtrees} {Noordam, J. E., \& Smirnov, O. M. 2010, A\&A, 524, A61}
\end{thebibliography}
\appendix
\section{PSF cross-sections}\label{app:psf}
\begin{figure}[H]
 \tiny{%%% autogen
 \begin{tabular}{llllll}
\includegraphics[width=0.150000\textwidth,trim= 0 .05cm 0 0.05cm]{{images/SKA1V8_8h60s_dec-30_650MHz_50MHz1ch-natural-psf}.png} &\includegraphics[width=0.150000\textwidth,trim= 0 .05cm 0 0.05cm]{{images/SKA1W8-0C0B120_8h60s_dec-30_650MHz_50MHz1ch-natural-psf}.png} &\includegraphics[width=0.150000\textwidth,trim= 0 .05cm 0 0.05cm]{{images/SKA1W8-0C9B120_8h60s_dec-30_650MHz_50MHz1ch-natural-psf}.png} &\includegraphics[width=0.150000\textwidth,trim= 0 .05cm 0 0.05cm]{{images/SKA1W8-12C0B120_8h60s_dec-30_650MHz_50MHz1ch-natural-psf}.png} &\includegraphics[width=0.150000\textwidth,trim= 0 .05cm 0 0.05cm]{{images/SKASUR_8h60s_dec-30_650MHz_50MHz1ch-natural-psf}.png} &\includegraphics[width=0.150000\textwidth,trim= 0 .05cm 0 0.05cm]{{images/SKASUR75_8h60s_dec-30_650MHz_50MHz1ch-natural-psf}.png} 
 \\ \hfill\includegraphics[width=0.150000\textwidth,trim= 0 .05cm 0 0.05cm]{{images/SKA1V8_8h60s_dec-30_650MHz_50MHz1ch-briggs,robust=-2,fov=10-psf}.png} &\includegraphics[width=0.150000\textwidth,trim= 0 .05cm 0 0.05cm]{{images/SKA1W8-0C0B120_8h60s_dec-30_650MHz_50MHz1ch-briggs,robust=-2,fov=10-psf}.png} &\includegraphics[width=0.150000\textwidth,trim= 0 .05cm 0 0.05cm]{{images/SKA1W8-0C9B120_8h60s_dec-30_650MHz_50MHz1ch-briggs,robust=-2,fov=10-psf}.png} &\includegraphics[width=0.150000\textwidth,trim= 0 .05cm 0 0.05cm]{{images/SKA1W8-12C0B120_8h60s_dec-30_650MHz_50MHz1ch-briggs,robust=-2,fov=10-psf}.png} &\includegraphics[width=0.150000\textwidth,trim= 0 .05cm 0 0.05cm]{{images/SKASUR_8h60s_dec-30_650MHz_50MHz1ch-briggs,robust=-2,fov=10-psf}.png} &\includegraphics[width=0.150000\textwidth,trim= 0 .05cm 0 0.05cm]{{images/SKASUR75_8h60s_dec-30_650MHz_50MHz1ch-briggs,robust=-2,fov=10-psf}.png} 
 \\ \hfill\includegraphics[width=0.150000\textwidth,trim= 0 .05cm 0 0.05cm]{{images/SKA1V8_8h60s_dec-30_650MHz_50MHz1ch-briggs,robust=-2,fov=10,taper=1-psf}.png} &\includegraphics[width=0.150000\textwidth,trim= 0 .05cm 0 0.05cm]{{images/SKA1W8-0C0B120_8h60s_dec-30_650MHz_50MHz1ch-briggs,robust=-2,fov=10,taper=1-psf}.png} &\includegraphics[width=0.150000\textwidth,trim= 0 .05cm 0 0.05cm]{{images/SKA1W8-0C9B120_8h60s_dec-30_650MHz_50MHz1ch-briggs,robust=-2,fov=10,taper=1-psf}.png} &\includegraphics[width=0.150000\textwidth,trim= 0 .05cm 0 0.05cm]{{images/SKA1W8-12C0B120_8h60s_dec-30_650MHz_50MHz1ch-briggs,robust=-2,fov=10,taper=1-psf}.png} &\includegraphics[width=0.150000\textwidth,trim= 0 .05cm 0 0.05cm]{{images/SKASUR_8h60s_dec-30_650MHz_50MHz1ch-briggs,robust=-2,fov=10,taper=1-psf}.png} &\includegraphics[width=0.150000\textwidth,trim= 0 .05cm 0 0.05cm]{{images/SKASUR75_8h60s_dec-30_650MHz_50MHz1ch-briggs,robust=-2,fov=10,taper=1-psf}.png} 
 \\ \hfill\end{tabular}}
 \caption{PSF cross-sections at Dec=-30 deg, Freq=650MHz. Row 1 and 2 are for natural and uniform weighting respectively, and row
3 is for uniform weighting with a 1 arcsec Gaussian taper. The blue and green curves are cross-sections along $l$ and $m$
respectively, and the horizontal line marks the FWHM. FWHM parameters are included in the plot.}
\end{figure}
\begin{figure}[H]
 \tiny{%%% autogen
 \begin{tabular}{lll||ll}
\includegraphics[width=0.180000\textwidth,trim= 0 .05cm 0 0.05cm]{{images/SKA1REF2_8h60s_dec-30_800MHz_1ch-natural-psf.fits}.png} &\includegraphics[width=0.180000\textwidth,trim= 0 .05cm 0 0.05cm]{{images/SKA1W9-12A72B120_8h60s_dec-30_800MHz_1ch-natural-psf.fits}.png} &\includegraphics[width=0.180000\textwidth,trim= 0 .05cm 0 0.05cm]{{images/SKA1W9-0A72B120_8h60s_dec-30_800MHz_1ch-natural-psf.fits}.png} &\includegraphics[width=0.180000\textwidth,trim= 0 .05cm 0 0.05cm]{{images/SKASUR1_8h60s_dec-30_800MHz_1ch-natural-psf.fits}.png} &\includegraphics[width=0.180000\textwidth,trim= 0 .05cm 0 0.05cm]{{images/SKASUR_8h60s_dec-30_800MHz_1ch-natural-psf.fits}.png} 
 \\ \hfill\includegraphics[width=0.180000\textwidth,trim= 0 .05cm 0 0.05cm]{{images/SKA1REF2_8h60s_dec-30_800MHz_1ch-briggs,robust=-2,fov=10-psf.fits}.png} &\includegraphics[width=0.180000\textwidth,trim= 0 .05cm 0 0.05cm]{{images/SKA1W9-12A72B120_8h60s_dec-30_800MHz_1ch-briggs,robust=-2,fov=10-psf.fits}.png} &\includegraphics[width=0.180000\textwidth,trim= 0 .05cm 0 0.05cm]{{images/SKA1W9-0A72B120_8h60s_dec-30_800MHz_1ch-briggs,robust=-2,fov=10-psf.fits}.png} &\includegraphics[width=0.180000\textwidth,trim= 0 .05cm 0 0.05cm]{{images/SKASUR1_8h60s_dec-30_800MHz_1ch-briggs,robust=-2,fov=10-psf.fits}.png} &\includegraphics[width=0.180000\textwidth,trim= 0 .05cm 0 0.05cm]{{images/SKASUR_8h60s_dec-30_800MHz_1ch-briggs,robust=-2,fov=10-psf.fits}.png} 
 \\ \hfill\includegraphics[width=0.180000\textwidth,trim= 0 .05cm 0 0.05cm]{{images/SKA1REF2_8h60s_dec-30_800MHz_1ch-briggs,robust=-2,fov=10,taper=1-psf.fits}.png} &\includegraphics[width=0.180000\textwidth,trim= 0 .05cm 0 0.05cm]{{images/SKA1W9-12A72B120_8h60s_dec-30_800MHz_1ch-briggs,robust=-2,fov=10,taper=1-psf.fits}.png} &\includegraphics[width=0.180000\textwidth,trim= 0 .05cm 0 0.05cm]{{images/SKA1W9-0A72B120_8h60s_dec-30_800MHz_1ch-briggs,robust=-2,fov=10,taper=1-psf.fits}.png} &\includegraphics[width=0.180000\textwidth,trim= 0 .05cm 0 0.05cm]{{images/SKASUR1_8h60s_dec-30_800MHz_1ch-briggs,robust=-2,fov=10,taper=1-psf.fits}.png} &\includegraphics[width=0.180000\textwidth,trim= 0 .05cm 0 0.05cm]{{images/SKASUR_8h60s_dec-30_800MHz_1ch-briggs,robust=-2,fov=10,taper=1-psf.fits}.png} 
 \\ \hfill\end{tabular}}
 \caption{PSF cross-sections at Dec=-30 deg, Freq=800MHz. Row 1 and 2 are for natural and uniform weighting respectively, and  row
3 is for uniform weighting with a 1 arcsec Gaussian taper. The blue and green curves are cross-sections along $l$ and $m$
respectively, and the horizontal line marks the FWHM. FWHM parameters are included in the plot.}
\end{figure}
\begin{figure}[H]
 \tiny{\include{dec-30_1100}}
 \caption{PSF cross-sections at Dec=-30 deg, Freq=1100MHz. Row 1 and 2 are for natural and uniform weighting respectively, and row
3 is for uniform weighting with a 1 arcsec Gaussian taper. The blue and green curves are cross-sections along $l$ and $m$
respectively, and the horizontal line marks the FWHM. FWHM parameters are included in the plot.}
\end{figure}

\end{document}

